% Chapter 1 - Pendahuluan
% Keep comments in English; content in Indonesian.
\chapter{Pendahuluan}
\label{chap:pendahuluan}

\section{Latar Belakang}
Perkembangan \textit{Large Language Model} (LLM) telah mendorong lahirnya beragam asisten pemrograman yang mampu membantu pengembang perangkat lunak dalam menulis, meninjau, dan memodifikasi kode. Meskipun demikian, sebagian besar asisten tersebut beroperasi sebagai ekstensi editor atau layanan berbasis \textit{cloud} yang menyimpan, memproses, atau melatih dari data pengguna. Kondisi ini menimbulkan kekhawatiran terkait privasi, kendali atas data, serta ketergantungan pada antarmuka tertentu.

Di sisi lain, \textit{Command Line Interface} (CLI) tetap menjadi lingkungan kerja yang penting bagi banyak pengembang karena sifatnya yang ringan, dapat diotomasi, dan mudah diintegrasikan dengan beragam alat. Integrasi kemampuan agen cerdas yang \textit{stateful} dan \textit{proactive} ke dalam CLI berpotensi mempercepat proses pengembangan perangkat lunak. Dalam konteks Paicode, sistem berjalan pada terminal lokal dan mengeksekusi tindakan langsung pada \textbf{berkas proyek di workspace}; namun, cuplikan kode/konteks \textbf{dikirim ke layanan LLM melalui API} untuk keperluan inferensi \cite{brown2020gpt3,openai2023gpt4,gemini2023}. Dengan demikian, aspek privasi/kerahasiaan kode \textbf{bergantung pada kebijakan penyedia API}, sementara pengamanan di sisi lokal difokuskan pada kebijakan \textit{path security} (keamanan \textit{path}) dan pembatasan perubahan berbasis \textit{diff}.

Penelitian ini menghadirkan \textbf{Paicode}, sebuah agen AI berbasis CLI yang dirancang untuk membantu proses pengembangan perangkat lunak secara interaktif dengan arsitektur 8-langkah (\textit{8-step workflow}). Paicode mampu: (i) mengobservasi struktur proyek (\texttt{TREE}, \texttt{LIST\_PATH}); (ii) membaca dan menulis berkas proyek (\texttt{READ}, \texttt{WRITE}); (iii) memodifikasi kode secara terarah dengan batasan perubahan berbasis diff (\texttt{MODIFY}) hingga maksimal 500 baris per modifikasi; (iv) menegakkan kebijakan keamanan path pada berkas proyek (memblokir akses ke direktori sensitif seperti \texttt{.git}, \texttt{venv}, dan \texttt{.env}); (v) melakukan klasifikasi intensi pengguna (\textit{chat} vs \textit{task}); (vi) menjalankan fase berpikir (\textit{thinking phase}) sebelum eksekusi; serta (vii) melakukan pemeriksaan integritas (\textit{integrity check}) pasca-eksekusi dengan skor kualitas 1-10. Sistem diimplementasikan pada lingkungan Ubuntu dengan bahasa pemrograman Python, pengelolaan dependensi melalui Poetry, mendukung manajemen multi-API key dengan \textit{round-robin load balancing}, dan menggunakan API Gemini sebagai LLM.

\section{Rumusan Masalah}
Berdasarkan latar belakang tersebut, rumusan masalah yang diajukan adalah sebagai berikut:

\begin{enumerate}
  \item Bagaimana merancang arsitektur agen AI berbasis CLI dengan \textit{8-step workflow} yang mencakup klasifikasi intensi, perencanaan tugas, fase berpikir, eksekusi aksi, dan pemeriksaan integritas, disertai integrasi LLM melalui API dan manajemen multi-API key dengan \textit{round-robin load balancing}?
  \item Bagaimana mengimplementasikan kebijakan keamanan \textit{path} dan pembatasan perubahan berbasis \textit{diff} dengan threshold maksimal 500 baris per modifikasi untuk mencegah penimpaan berkas yang tidak diinginkan?
  \item Bagaimana mengintegrasikan kemampuan observasi proyek, manipulasi berkas, serta modifikasi kode terarah dengan mekanisme \textit{interrupt handling} (Ctrl+C) dan fitur \textit{auto-continue} untuk pengalaman interaktif yang lebih baik?
  \item Bagaimana mengevaluasi efektivitas Paicode dalam membantu tugas-tugas pemrograman dengan metrik efisiensi, ketepatan hasil, dan skor kualitas dari sistem \textit{integrity check}?
\end{enumerate}

\section{Batasan Masalah}
Agar fokus penelitian terjaga dan implementasi dapat dilakukan secara terukur, batasan-batasan berikut ditetapkan:

\begin{itemize}
  \item Lingkungan target adalah sistem operasi Ubuntu (Linux) dengan antarmuka CLI.
  \item Bahasa pemrograman utama adalah Python; contoh dan skenario uji berfokus pada ekosistem Python/Unix.
  \item Layanan LLM eksternal menggunakan API Gemini; kualitas respons bergantung pada model dan tidak menjadi ruang lingkup untuk dioptimasi ulang.
  \item Dukungan multi-pengguna, kolaborasi real-time, dan integrasi langsung dengan editor tidak dibahas pada versi ini.
  \item Aspek visual seperti diagram dan ilustrasi antarmuka ditunda pada tahap akhir; fokus laporan adalah narasi dan hasil teknis.
\end{itemize}

\section{Tujuan Penelitian}
Tujuan penelitian ini adalah membangun dan mengevaluasi sebuah agen AI berbasis CLI yang dapat membantu pengembang dalam proses pemrograman secara interaktif dengan arsitektur 8-langkah. Secara khusus, penelitian menargetkan:

\begin{enumerate}
  \item Merancang arsitektur Paicode yang mencakup modul agen dengan \textit{8-step workflow} (klasifikasi intensi, perencanaan tugas, fase berpikir, eksekusi aksi, pemeriksaan integritas, dan ringkasan akhir), jembatan LLM dengan manajemen multi-API key, antarmuka CLI dengan \textit{interrupt handling}, lapisan keamanan path pada berkas proyek, serta komponen tampilan terminal.
  \item Mengimplementasikan kemampuan observasi proyek, manipulasi berkas, dan modifikasi kode terarah dengan mekanisme \textit{patch/diff} yang membatasi perubahan maksimal 500 baris per modifikasi (dapat dikonfigurasi melalui variabel lingkungan).
  \item Mengintegrasikan fitur-fitur interaktif seperti \textit{auto-continue} untuk melanjutkan saran otomatis, pencatatan sesi ke \texttt{.pai\_history}, dan sistem penilaian kualitas dengan skor 1-10 pada setiap eksekusi.
  \item Menyusun prosedur evaluasi dengan skenario tugas pemrograman yang representatif dan mengukur peningkatan produktivitas, kualitas hasil, serta efektivitas sistem \textit{integrity check}.
\end{enumerate}

\section{Manfaat Penelitian}
Manfaat yang diharapkan dari penelitian ini meliputi:

\begin{itemize}
  \item \textbf{Akademis:} menyediakan studi kasus dan arsitektur rujukan untuk pengembangan agen AI berbasis CLI dengan integrasi LLM melalui API, serta memperkaya literatur mengenai integrasi LLM dalam alur kerja rekayasa perangkat lunak.
  \item \textbf{Praktis:} menghadirkan alat bantu yang \textit{privacy-aware} dan mudah diintegrasikan dengan berbagai IDE karena beroperasi langsung pada berkas proyek di ruang kerja (workspace); memfasilitasi pembuatan struktur proyek, pembacaan, dan modifikasi kode secara cepat dan terarah.
\end{itemize}
