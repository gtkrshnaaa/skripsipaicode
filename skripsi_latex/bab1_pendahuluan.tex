% Chapter 1 - Pendahuluan
% Keep comments in English; content in Indonesian.
\chapter{PENDAHULUAN}
\label{chap:pendahuluan}

\section{Latar Belakang}
Perkembangan \textit{Large Language Model} (LLM) telah mendorong lahirnya beragam asisten pemrograman yang mampu membantu pengembang perangkat lunak dalam menulis, meninjau, dan memodifikasi kode. Meskipun demikian, sebagian besar asisten tersebut beroperasi sebagai ekstensi editor atau layanan berbasis \textit{cloud} yang menyimpan, memproses, atau melatih dari data pengguna. Kondisi ini menimbulkan kekhawatiran terkait privasi, kendali atas data, serta ketergantungan pada antarmuka tertentu.

Di sisi lain, \textit{Command Line Interface} (CLI) tetap menjadi lingkungan kerja yang penting bagi banyak pengembang karena sifatnya yang ringan, dapat diotomasi, dan mudah diintegrasikan dengan beragam alat. Integrasi kemampuan agen cerdas yang \textit{stateful} dan \textit{proactive} ke dalam CLI berpotensi mempercepat proses pengembangan perangkat lunak. Dalam konteks Paicode, sistem berjalan pada terminal lokal dan mengeksekusi tindakan langsung pada \textbf{berkas proyek di workspace}; namun, cuplikan kode/konteks \textbf{dikirim ke layanan LLM melalui API} untuk keperluan inferensi \cite{brown2020gpt3,openai2023gpt4,gemini2023}. Dengan demikian, aspek privasi/kerahasiaan kode \textbf{bergantung pada kebijakan penyedia API}, sementara pengamanan di sisi lokal difokuskan pada kebijakan \textit{path security} (keamanan \textit{path}) dan pembatasan perubahan berbasis \textit{diff}.

Penelitian ini menghadirkan \textbf{Paicode}, sebuah agen AI berbasis CLI yang dirancang untuk membantu proses pengembangan perangkat lunak secara interaktif dengan arsitektur \textit{Single-Shot Intelligence}. Paicode mampu: (i) mengobservasi struktur proyek (\texttt{TREE}, \texttt{LIST\_PATH}); (ii) membaca dan menulis berkas proyek (\texttt{READ}, \texttt{WRITE}); (iii) memodifikasi kode secara terarah dengan sistem perubahan berbasis \textit{diff} dengan threshold ganda: 500 baris absolut dan 50\% ratio maksimal (\texttt{MODIFY}); (iv) menegakkan kebijakan keamanan \textit{path} pada berkas proyek (memblokir akses ke direktori sensitif seperti \texttt{.git}, \texttt{venv}, dan \texttt{.env}); (v) melakukan klasifikasi intensi pengguna (\textit{chat} vs \textit{task}); (vi) mengoptimalkan efisiensi dengan sistem \textit{Single-Shot Intelligence} yang mencakup \textit{acknowledgment} dinamis, perencanaan JSON, dan eksekusi adaptif 1-3 subfase; serta (vii) menyediakan penanganan interupsi (\textit{interrupt handling}) untuk kontrol sesi yang lebih baik. Sistem diimplementasikan pada lingkungan Ubuntu dengan bahasa pemrograman Python, pengelolaan dependensi melalui pip dan virtual environment, manajemen API key tunggal, dan menggunakan API Gemini sebagai LLM.

\section{Rumusan Masalah}
Berdasarkan latar belakang tersebut, rumusan masalah yang diajukan adalah:

\vspace{0.3cm}

Bagaimana merancang, mengimplementasikan, dan mengevaluasi agen AI berbasis CLI dengan arsitektur Single-Shot Intelligence yang mampu mengotomasi aktivitas pemrograman secara aman melalui kebijakan path security dan pembatasan perubahan berbasis diff, serta terintegrasi dengan LLM melalui API?


\section{Ruang Lingkup}
Agar fokus penelitian terjaga dan implementasi dapat dilakukan secara terukur, batasan-batasan berikut ditetapkan:

\begin{itemize}
  \item Lingkungan target adalah sistem operasi Ubuntu (Linux) dengan antarmuka CLI.
  \item Bahasa pemrograman utama adalah Python; contoh dan skenario uji berfokus pada ekosistem Python/Unix.
  \item Layanan LLM eksternal menggunakan API Gemini; kualitas respons bergantung pada model dan tidak menjadi ruang lingkup untuk dioptimasi ulang.
  \item Dukungan multi-pengguna, kolaborasi real-time, dan integrasi langsung dengan editor tidak dibahas pada versi ini.
  \item Aspek visual seperti diagram dan ilustrasi antarmuka ditunda pada tahap akhir; fokus laporan adalah narasi dan hasil teknis.
\end{itemize}

\section{Tujuan Penelitian}
Tujuan penelitian ini adalah membangun dan menguji fungsionalitas sebuah agen AI berbasis CLI yang dapat membantu pengembang dalam proses pemrograman secara interaktif dengan arsitektur \textit{Single-Shot Intelligence}. Secara khusus, penelitian menargetkan:

\begin{enumerate}
  \item Merancang arsitektur Paicode yang mencakup modul agen dengan \textit{Single-Shot Intelligence} (klasifikasi intensi, fase perencanaan, dan fase eksekusi dalam 2 fase utama), jembatan LLM dengan manajemen API key tunggal, antarmuka CLI dengan \textit{interrupt handling}, lapisan keamanan \textit{path} pada berkas proyek, serta komponen tampilan terminal berbasis \texttt{rich}.
  \item Mengimplementasikan kemampuan observasi proyek, manipulasi berkas, dan modifikasi kode terarah dengan mekanisme \textit{diff}-aware yang mencegah penimpaan berkas tidak diinginkan dan memblokir akses ke direktori sensitif.
  \item Mengintegrasikan fitur-fitur interaktif seperti pencatatan sesi ke \texttt{.pai\_history}, penanganan interupsi (Ctrl+C), dan antarmuka terminal yang responsif dengan dukungan input multiline.
  \item Menyusun prosedur evaluasi dengan skenario tugas pemrograman yang representatif dan mengukur efisiensi panggilan API, ketepatan hasil, serta kepatuhan keamanan \textit{path}.
\end{enumerate}

\section{Manfaat Penelitian}
Manfaat yang diharapkan dari penelitian ini meliputi:

\begin{itemize}
  \item \textbf{Akademis:} menyediakan studi kasus dan arsitektur rujukan untuk pengembangan agen AI berbasis CLI dengan integrasi LLM melalui API, serta memperkaya literatur mengenai integrasi LLM dalam alur kerja rekayasa perangkat lunak.
  \item \textbf{Praktis:} menghadirkan alat bantu pengembangan perangkat lunak dengan kelebihan spesifik sebagai berikut:
    \begin{enumerate}
        \item \textbf{Efisiensi Biaya dan Token}: Menggunakan arsitektur \textit{Single-Shot Intelligence} yang memadatkan proses perencanaan dan eksekusi menjadi dua panggilan utama, mengurangi biaya API dibandingkan agen berbasis \textit{chat-loop} konvensional.
        \item \textbf{Keamanan Terkendali}: Menerapkan kebijakan keamanan \textit{path} (path security) yang memblokir akses ke direktori sensitif (seperti \texttt{.git}, \texttt{.env}) dan mekanisme modifikasi berbasis \textit{diff} untuk mencegah perubahan destruktif masif.
        \item \textbf{Fleksibilitas Lingkungan}: Beroperasi sebagai utilitas CLI yang ringan dan agnostik terhadap editor kode (IDE-agnostic), sehingga dapat digunakan di server tanpa antarmuka grafis (headless) maupun sebagai pendamping editor apa pun di OS berbasis Linux.
    \end{enumerate}
\end{itemize}

\section{Sistematika Penulisan}
Laporan tugas akhir ini disusun dalam lima bab yang saling berkaitan. Bab I (Pendahuluan) memaparkan latar belakang masalah, rumusan masalah yang akan diselesaikan, ruang lingkup penelitian, tujuan yang ingin dicapai, manfaat penelitian bagi aspek akademis maupun praktis, serta sistematika penulisan laporan ini.

Bab II (Tinjauan Pustaka dan Dasar Teori) menguraikan tinjauan pustaka dari penelitian-penelitian terdahulu yang relevan dengan pengembangan agen AI dan CLI, serta landasan teori yang mendukung penelitian, meliputi konsep \textit{Command Line Interface} (CLI), \textit{Artificial Intelligence} (AI) Agent, \textit{Large Language Model} (LLM), dan arsitektur perangkat lunak terkait.

Bab III (Metode Penelitian) menjelaskan bahan dan data yang digunakan, peralatan pendukung baik perangkat keras maupun lunak, prosedur dan pengumpulan data yang dilakukan selama penelitian, serta analisis dan perancangan sistem Paicode secara rinci.

Bab IV (Implementasi dan Pembahasan) menjabarkan proses lingkungan implementasi sistem, realisasi fitur-fitur utama Paicode, skenario pengujian yang dilakukan, serta pembahasan mendalam mengenai hasil uji coba dan evaluasi kinerja sistem.

Bab V (Penutup) berisi simpulan yang diperoleh dari seluruh rangkaian penelitian serta saran-saran konstruktif untuk pengembangan sistem Paicode di masa mendatang.
