% Front matter: Kata Pengantar (placeholder)
% Content in Indonesian; comments in English.
\chapter*{Kata Pengantar}
\addcontentsline{toc}{chapter}{Kata Pengantar}

Puji syukur ke hadirat Tuhan Yang Maha Esa atas limpahan rahmat dan karunia-Nya sehingga skripsi ini dapat diselesaikan. Naskah ini disusun untuk memenuhi salah satu syarat memperoleh gelar Sarjana pada Program Studi \Prodi, \Fakultas, \Universitas.

Ucapan terima kasih disampaikan kepada:
\begin{enumerate}
  \item Orang tua dan keluarga atas doa, dukungan, serta motivasi yang tiada henti.
  \item Dosen pembimbing I dan II atas bimbingan, arahan, dan waktu yang dicurahkan selama proses penyusunan.
  \item Para dosen dan staf di \Fakultas{} atas ilmu, kesempatan, dan fasilitas yang diberikan.
  \item Rekan-rekan mahasiswa yang telah memberikan bantuan, masukan, dan semangat selama penelitian ini berlangsung.
  \item Pihak-pihak lain yang tidak dapat disebutkan satu per satu, yang turut berkontribusi dalam bentuk apa pun.
\end{enumerate}

Penulis menyadari bahwa skripsi ini masih memiliki keterbatasan. Oleh karena itu, kritik dan saran yang membangun sangat diharapkan demi perbaikan pada penelitian selanjutnya.

\vspace{1cm}
\begin{flushright}
  \begin{minipage}{0.6\textwidth}
    \centering
    Yogyakarta, ................................ \\[1.6cm]
    {\small \resizebox{\linewidth}{!}{\textbf{\NamaMahasiswa}}} \\
    {\small NIM: \NIM}
  \end{minipage}
\end{flushright}
