% Front matter: Kata Pengantar
\thispagestyle{frontmatterstyle}
\addcontentsline{toc}{chapter}{PRAKATA}

\begin{center}
  \textbf{\fontsize{14pt}{17pt}\selectfont PRAKATA}
\end{center}

\vspace{1cm}

\indent Puji syukur ke hadirat Tuhan Yang Maha Esa atas segala limpahan rahmat, hidayah, dan karunia-Nya yang tak terhingga, sehingga penulis dapat menyelesaikan penyusunan Tugas Akhir yang berjudul \textbf{\JudulSkripsi} dengan baik dan tepat pada waktunya. Penulisan Tugas Akhir ini dimaksudkan untuk memenuhi salah satu syarat akademik guna memperoleh gelar Sarjana Komputer pada Program Studi \Prodi, \Fakultas, \Universitas. Penulis menyadari bahwa perjalanan dalam menyelesaikan studi ini bukanlah hal yang mudah, namun berkat penyertaan-Nya, segala rintangan dapat terlewati.

Dalam proses penyusunan tugas akhir ini, penulis menyadari bahwa tugas akhir ini tidak akan terselesaikan tanpa bantuan, bimbingan, dan dukungan dari berbagai pihak. Oleh karena itu, dengan segala kerendahan hati, penulis ingin menyampaikan terima kasih yang sebesar-besarnya kepada:

\begin{enumerate}
  \item Ibu Sri Redjeki, S.Si., M.Kom., Ph.D. selaku Rektor Universitas Teknologi Digital Indonesia.
  \item Ibu Dr. L.N. Harnaningrum, S.Si., M.T., Selaku Dekan Fakultas Teknologi Informasi.
  \item Ibu Dini Fakta Sari, S.T., M.T., selaku Ketua Program Studi Informatika di Universitas Teknologi Digital Indonesia.
  \item Bapak Dr. Bambang Purnomosidi Dwi Putranto, S.E., Akt., S.Kom., MMSI selaku dosen pembimbing yang telah dengan sabar memberikan bimbingan, arahan, serta dukungan dari awal hingga selesainya skripsi ini.
  \item Keluarga, serta teman-teman semuanya yang selalu menemani dimasa perkuliahan saya, yang senantiasa mendoakan, serta memotivasi saya dalam menyelesaikan naskah skripsi ini.
  \item Seluruh Bapak dan Ibu Dosen serta Staff Karyawan Universitas Teknologi Digital Indonesia.
  \item Semua pihak yang telah membantu dan tidak bisa disebutkan satu persatu.
\end{enumerate}

Penulis menyadari sepenuhnya bahwa dalam penyusunan Tugas Akhir ini masih terdapat banyak kekurangan dan keterbatasan, baik dari segi materi maupun tata bahasa, mengingat keterbatasan pengetahuan dan pengalaman yang penulis miliki. Oleh karena itu, dengan kerendahan hati, penulis sangat mengharapkan kritik dan saran yang membangun dari semua pihak demi penyempurnaan di masa yang akan datang. Besar harapan penulis, semoga karya sederhana ini dapat memberikan manfaat nyata, menjadi referensi yang berguna, serta memberikan kontribusi positif bagi pengembangan ilmu pengetahuan dan teknologi, khususnya di bidang Informatika.

Akhir kata, penulis mengucapkan terima kasih yang tulus kepada semua pihak yang telah memberikan dukungan morel maupun materiel. Semoga Tuhan Yang Maha Esa senantiasa melimpahkan rahmat, perlindungan, dan keberkahan-Nya kepada kita semua dalam setiap langkah kehidupan.

\vspace{1.5cm}
\begin{flushright}
  \begin{minipage}{0.5\textwidth}
    \centering
    Yogyakarta, 24 November 2025 \\[2cm]
    \textbf{Penulis}
  \end{minipage}
\end{flushright}
