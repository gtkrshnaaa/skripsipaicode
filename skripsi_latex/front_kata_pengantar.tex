% Front matter: Kata Pengantar
\thispagestyle{plain}
\addcontentsline{toc}{chapter}{PRAKATA}

\begin{center}
  \textbf{\fontsize{14pt}{17pt}\selectfont PRAKATA}
\end{center}

\vspace{1cm}

Puji syukur ke hadirat Tuhan Yang Maha Esa atas segala rahmat dan karunia-Nya sehingga penulis dapat menyelesaikan Tugas Akhir yang berjudul \textbf{\JudulSkripsi}. Tugas Akhir ini disusun sebagai salah satu syarat untuk memperoleh gelar Sarjana Komputer pada Program Studi \Prodi, \Fakultas, \Universitas.

Penulis menyadari bahwa penyelesaian Tugas Akhir ini tidak lepas dari bantuan, bimbingan, dan dukungan berbagai pihak. Oleh karena itu, penulis menyampaikan terima kasih kepada:

\begin{enumerate}
  \item Tuhan Yang Maha Esa atas segala rahmat, kesehatan, dan kemudahan yang diberikan selama proses penelitian.
  \item Orang tua dan keluarga yang senantiasa memberikan doa, dukungan moral, dan motivasi yang tiada henti.
  \item Bapak Dr. Bambang Purnomosidi Dwi Putranto, S.E., Akt., S.Kom., MMSI selaku dosen pembimbing yang telah memberikan bimbingan, arahan, dan masukan yang sangat berharga selama penyusunan Tugas Akhir ini.
  \item Seluruh dosen dan staf \Fakultas{} yang telah memberikan ilmu, fasilitas, dan dukungan selama masa perkuliahan.
  \item Rekan-rekan mahasiswa yang telah memberikan bantuan, diskusi, dan semangat selama proses penelitian.
  \item Semua pihak yang tidak dapat disebutkan satu per satu yang telah membantu dalam penyelesaian Tugas Akhir ini.
\end{enumerate}

Penulis menyadari bahwa Tugas Akhir ini masih jauh dari sempurna. Oleh karena itu, penulis mengharapkan kritik dan saran yang membangun untuk perbaikan di masa mendatang. Semoga Tugas Akhir ini dapat memberikan manfaat bagi pembaca dan perkembangan ilmu pengetahuan.

\vspace{1.5cm}
\begin{flushright}
  \begin{minipage}{0.5\textwidth}
    \centering
    Yogyakarta, 24 November 2025 \\[2cm]
    \textbf{Penulis}
  \end{minipage}
\end{flushright}
