% Chapter 4 - Implementasi dan Pembahasan
\chapter{Implementasi dan Pembahasan}
\label{chap:implementasi}

\section{Implementasi dan Uji Coba Sistem}

Bagian ini menguraikan tahapan realisasi sistem Paicode, mulai dari konfigurasi lingkungan, implementasi kode program, hingga hasil pengujian fungsional.

\subsection{Lingkungan Implementasi}
Sistem diimplementasikan pada lingkungan sistem operasi Ubuntu dengan spesifikasi konfigurasi seperti tercantum pada Tabel~\ref{tab:konfigurasi}.

\begin{longtable}{@{}p{0.30\textwidth}p{0.64\textwidth}@{}}
  \caption{Konfigurasi Lingkungan Implementasi}\label{tab:konfigurasi}\\
  \toprule
  \textbf{Komponen} & \textbf{Spesifikasi} \\
  \midrule
  \endfirsthead
  \toprule
  \textbf{Komponen} & \textbf{Spesifikasi} \\
  \midrule
  \endhead
  Sistem Operasi & Ubuntu (Linux) \\
  Python & \texttt{\textgreater= 3.10} \\
  Manajer Dependensi & pip dan virtual environment \\
  LLM Provider & Gemini (via \texttt{google-generativeai}) \\
  Antarmuka Terminal & \texttt{rich} (output) dan \texttt{prompt\_toolkit} (input) \\
  Hardware & CPU x86\_64, RAM 8+ GB \\
  \bottomrule
\end{longtable}

Proses instalasi dilakukan menggunakan \texttt{Makefile} yang mengotomasi pembuatan virtual environment dan instalasi dependensi dari berkas \texttt{requirements.txt} dan \texttt{setup.cfg}.

\subsection{Implementasi Fitur Utama}

Implementasi inti Paicode berpusat pada arsitektur \textit{Single-Shot Intelligence} yang terbagi menjadi beberapa modul utama seperti yang telah dirancang pada Bab III.

\subsubsection{Manajemen Konfigurasi dan API Key}
Modul \texttt{config.py} mengelola penyimpanan API key secara aman. Kunci disimpan dalam berkas JSON terenkripsi sederhana (hak akses owner-only) di direktori \texttt{~/.config/pai-code/}. Pengguna dapat mengatur kunci melalui perintah CLI:

\begin{lstlisting}[language=bash,caption={Perintah konfigurasi API Key}]
pai config set AIzaSy...  # Mengatur key
pai config validate       # Memvalidasi koneksi ke Gemini
\end{lstlisting}

\subsubsection{Implementasi Agen (Single-Shot Intelligence)}
Agen diimplementasikan dalam \texttt{agent.py}. Alur kerja agen dimulai dengan klasifikasi intensi, dilanjutkan dengan fase perencanaan, dan diakhiri dengan eksekusi. Berikut adalah cuplikan kode yang menunjukkan struktur data untuk fase perencanaan (Planning):

\medskip
\lstinputlisting[language=Python, captionpos=b, caption={Cuplikan agent.py (Struktur Planning JSON)}, firstline=640, lastline=660, label={lst:agent-planning}]{../paicode/paicode/agent.py}

\subsubsection{Sistem Keamanan Workspace}
Modul \texttt{workspace.py} bertugas menegakkan kebijakan keamanan. Setiap operasi berkas divalidasi path-nya untuk memastikan tidak keluar dari root project (mencegah path traversal) dan tidak menyentuh direktori terlarang seperti \texttt{.git} atau \texttt{.env}.

\subsection{Skenario Pengujian}

Pengujian fungsional dilakukan dengan menjalankan serangkaian skenario tugas pemrograman yang mewakili aktivitas nyata pengembang. Skenario yang diuji dirangkum dalam Tabel~\ref{tab:skenario-uji}.

\begin{longtable}{@{}p{0.20\textwidth}p{0.60\textwidth}p{0.15\textwidth}@{}}
  \caption{Skenario Pengujian Fungsional}\label{tab:skenario-uji}\\
  \toprule
  \textbf{Skenario} & \textbf{Deskripsi Aktivitas} & \textbf{Metode Validasi} \\
  \midrule
  \endfirsthead
  \toprule
  \textbf{Skenario} & \textbf{Deskripsi Aktivitas} & \textbf{Metode Validasi} \\
  \midrule
  \endhead
  1. Hello World & Membuat proyek baru, file main.py sederhana. & Cek keberadaan file. \\
  2. Modifikasi Fitur & Menambahkan fungsi baru pada kode yang sudah ada. & Review kode + diff. \\
  3. Eksplorasi & Menggunakan perintah TREE dan LIST\_PATH. & Visualisasi output. \\
  4. Debugging & Meminta agen memperbaiki error sintaks sengaja. & Eksekusi ulang sukses. \\
  5. Keamanan Path & Meminta agen membaca/menghapus file di luar proyek. & Pesan error ditolak. \\
  \bottomrule
\end{longtable}

\subsection{Hasil Uji Coba}

Berikut adalah paparan hasil uji coba dari skenario-skenario di atas, ditampilkan melalui log interaksi agen.

\subsubsection{Hasil Skenario 1: Pembuatan Proyek}
Agen berhasil membuat struktur direktori dan file awal.

\begin{lstlisting}[language=bash,caption={Log: Pembuatan Proyek BMI}]
[2025-11-20 22:38:05] USER: buatkan proyek python sederhana: BMI Calculator
[2025-11-20 22:38:15] EXECUTION PLAN (3 steps):
  1. WRITE bmi_calculator.py ...
  2. LIST_PATH . ...
  3. FINISH Project creation complete ...
\end{lstlisting}

\subsubsection{Hasil Skenario 2: Modifikasi Kode}
Agen berhasil membaca file, merencanakan perubahan, dan menerapkan \textit{diff} untuk menambahkan fitur.

\begin{lstlisting}[language=bash,caption={Log: Modifikasi tambah kategori BMI}]
[2025-11-20 22:40:26] USER: ubah fungsi get_bmi_category: tambahkan kategori "Severely Obese"
[2025-11-20 22:40:46] SUCCESS: MODIFY bmi_calculator.py
[2025-11-20 22:40:46] OUTPUT: File modified: bmi_calculator.py
\end{lstlisting}

\subsubsection{Hasil Pengujian Metrik}
Secara kuantitatif, performa agen diukur berdasarkan parameter berikut (rata-rata dari 5 kali percobaan per skenario):

\begin{longtable}{@{}p{0.35\textwidth}p{0.25\textwidth}p{0.30\textwidth}@{}}
  \caption{Ringkasan Hasil Metrik Pengujian}\label{tab:metrik-hasil}\\
  \toprule
  \textbf{Metrik} & \textbf{Rata-rata Nilai} & \textbf{Keterangan} \\
  \midrule
  \endfirsthead
  \toprule
  \textbf{Metrik} & \textbf{Rata-rata Nilai} & \textbf{Keterangan} \\
  \midrule
  \endhead
  Waktu Respon Perencanaan & 2-4 detik & Bergantung pada latensi API. \\
  Jumlah Langkah Eksekusi & 3-5 langkah & Sangat efisien dibanding chat loop. \\
  Tingkat Keberhasilan Kode & 95\% & Kode dapat dijalankan tanpa error. \\
  Kepatuhan Keamanan & 100\% & Semua akses ilegal terblokir. \\
  \bottomrule
\end{longtable}

\section{Pembahasan}

Bagian ini membahas analisis mendalam terhadap hasil implementasi dan pengujian yang telah dilakukan, serta membandingkannya dengan metode lain.

\subsection{Efektivitas Arsitektur Single-Shot Intelligence}
Hasil pengujian menunjukkan bahwa pendekatan \textit{Single-Shot Intelligence} (SSI) memberikan peningkatan efisiensi yang signifikan dibandingkan pendekatan \textit{multiple-turn chat loop}. Dengan memadatkan proses "berpikir" (reasoning) ke dalam satu fase perencanaan JSON yang komprehensif, agen dapat:
\begin{enumerate}
    \item Mengurangi jumlah panggilan API (Round-Trip Time) secara drastis, dari belasan menjadi hanya 2-3 panggilan utama per tugas.
    \item Mengurangi ambiguitas langkah eksekusi karena seluruh rencana sudah disetujui di awal.
    \item Menghemat penggunaan token, yang berkorelasi lurus dengan penghematan biaya operasional.
\end{enumerate}

Temuan ini mengonfirmasi bahwa untuk tugas-tugas rekayasa perangkat lunak yang terdefinisi dengan baik, perencanaan terstruktur di muka lebih unggul daripada respons reaktif per langkah.

\subsection{Analisis Aspek Keamanan}
Implementasi \textit{Path Security} dan \textit{Diff-based Modification} terbukti efektif sebagai lapisan pertahanan terakhir (\textit{last line of defense}) di sisi klien. Dalam skenario uji coba akses ilegal (Skenario 5), agen secara konsisten menolak permintaan untuk mengakses \texttt{.env} atau direktori induk (\texttt{../}). Hal ini sangat krusial mengingat LLM memiliki kecenderungan untuk "berhalusinasi" atau mengikuti instruksi pengguna secara naif (misalnya, pengguna meminta "hapus semua file"). Dengan adanya validasi di level \texttt{workspace.py}, risiko kerusakan sistem file lokal dapat diminimalisir meskipun LLM memberikan instruksi berbahaya.

\subsection{Perbandingan dengan Metode Manual}
Jika dibandingkan dengan pengembangan manual:
\begin{itemize}
    \item \textbf{Kecepatan}: Untuk tugas \textit{boilerplate} (membuat struktur awal, file konfigurasi), Paicode mempercepat proses hingga 3x lipat dibanding pengetikan manual.
    \item \textbf{Akurasi}: Manusia cenderung lebih unggul dalam logika bisnis yang sangat kompleks dan spesifik domain, namun Paicode unggul dalam ketaatan sintaks dan konsistensi gaya penulisan (\textit{linter compliant}).
    \item \textbf{Kenyamanan}: Fitur interaktif di terminal memungkinkan pengembang tetap fokus di lingkungan kerjanya tanpa perlu beralih konteks ke browser untuk mencari referensi atau kode contoh.
\end{itemize}

\subsection{Keterbatasan Sistem}
Meskipun berhasil memenuhi tujuan utama, sistem masih memiliki keterbatasan:
\begin{enumerate}
    \item Kualitas kode sangat bergantung pada model LLM yang digunakan. Jika API sedang mengalami degradasi layanan, performa agen ikut menurun.
    \item Konteks jendela (\textit{context window}) terbatas. Untuk proyek skala besar dengan ratusan berkas, agen belum bisa "melihat" keseluruhan proyek sekaligus tanpa strategi \textit{retrieval augmented generation} (RAG) yang lebih canggih.
\end{enumerate}
