% Chapter 4 - Implementasi dan Pembahasan
\chapter{IMPLEMENTASI DAN PEMBAHASAN}
\label{chap:implementasi}

\section{Implementasi dan Uji Coba Sistem}

Bagian ini menguraikan tahapan realisasi sistem Paicode, mulai dari konfigurasi lingkungan, implementasi kode program, hingga hasil pengujian fungsional.

\subsection{Lingkungan Implementasi}
Sistem diimplementasikan pada lingkungan sistem operasi Ubuntu dengan spesifikasi konfigurasi seperti tercantum pada Tabel~\ref{tab:konfigurasi}.

\begin{longtable}{@{}p{0.32\textwidth}p{0.66\textwidth}@{}}
  \caption{Konfigurasi Lingkungan Implementasi}\label{tab:konfigurasi}\\
  \toprule
  \textbf{Komponen} & \textbf{Spesifikasi} \\
  \midrule
  \endfirsthead
  \toprule
  \textbf{Komponen} & \textbf{Spesifikasi} \\
  \midrule
  \endhead
  Sistem Operasi & Ubuntu (Linux) \\
  Python & \texttt{\textgreater= 3.10} \\
  Manajer Dependensi & pip dan virtual environment \\
  LLM Provider & Gemini (via \texttt{google-generativeai}) \\
  Antarmuka Terminal & \texttt{rich} (output) dan \texttt{prompt\_toolkit} (input) \\
  Hardware & CPU x86\_64, RAM 8+ GB \\
  \bottomrule
\end{longtable}

Proses instalasi dilakukan menggunakan \texttt{Makefile} yang mengotomasi pembuatan virtual environment dan instalasi dependensi dari berkas \texttt{requirements.txt} dan \texttt{setup.cfg}.

\subsection{Implementasi Fitur Utama}

Implementasi inti Paicode berpusat pada arsitektur \textit{Single-Shot Intelligence} yang terbagi menjadi beberapa modul utama seperti yang telah dirancang pada Bab III.

\subsubsection{Antarmuka Pengguna}
Antarmuka pengguna Paicode dibangun berbasis teks (\textit{Command Line Interface}) dengan memanfaatkan pustaka \texttt{rich} untuk menyajikan visualisasi yang terstruktur dan informatif pada terminal. Desain antarmuka mengutamakan kejelasan informasi dengan penggunaan panel, warna (pada terminal yang mendukung), dan tata letak yang rapi. Tampilan awal saat sesi interaktif dimulai direpresentasikan dalam bentuk ASCII pada Gambar~\ref{lst:ui-startup}.

\begin{center}
  \begin{lstlisting}[basicstyle=\ttfamily\scriptsize]
user@localhost:~/space/univ/skripsi/devpai/trypai$ pai
+------------------------- Interactive Auto Mode -------------------------+
|                                                                         |
|         Welcome! I'm Pai, your agentic AI coding companion. *           |
|      Now powered by Single-Shot Intelligence for maximum efficiency.    |
|              [info]Type 'exit' or 'quit' to leave.[/info]               |
|          [info]Each request uses exactly 2 API calls.[/info]            |
|       [info]* Multi-line input: Alt+Enter for new line,                 |
|                        Enter to submit.[/info]                          |
|                                                                         |
+-------------------------------------------------------------------------+
* Tip: Use Alt+Enter for new line, Enter to submit
 
user> hai pai, make me a simple calculator in python.
i Tokens: 194 -> 1
i Tokens: 170 -> 21
+---------------------------------- Pai ----------------------------------+
|                                                                         |
|    Sure thing! A simple Python calculator, got it. Let me cook up       |
|                       a smart plan for that!                            |
|                                                                         |
+-------------------------------------------------------------------------+
+------------------ Call 1/2: Intelligence Planning ----------------------+
|                                                                         |
|                        Deep Analysis & Planning                         |
|                                                                         |
+-------------------------------------------------------------------------+
i Tokens: 1853 -> 1138
+---------------------------- Planning Results ---------------------------+
|                                                                         |
|  Smart Analysis Results:                                                |
|  Intent: Create a simple calculator program in Python.                  |
|  Context Usage: No context utilized                                     |
|  Files to read: 0                                                       |
|  Files to create: 1                                                     |
|  Files to modify: 0                                                     |
|  Efficiency: Standard approach                                          |
|                                                                         |
|  Execution Plan: 3 steps                                                |
|                                                                         |
|  +----+-----------------+------------------------------------------+    |
|  | No | Action          | Purpose                                  |    |
|  +----+-----------------+------------------------------------------+    |
|  | 1  | WRITE calcul... | Create a new Python file containing...   |    |
|  | 2  | LIST_PATH .     | Verify the creation of 'calculator...    |    |
|  | 3  | FINISH calcu... | Inform the user that the simple...       |    |
|  +----+-----------------+------------------------------------------+    |
|                                                                         |
|  Intelligence Assessment:                                               |
|  Complexity: simple                                                     |
|  Estimated time: 1 minute                                               |
|                                                                         |
+-------------------------------------------------------------------------+
i Tokens: 168 -> 20
+---------------------------------- Pai ----------------------------------+
|                                                                         |
|    Consider it done! My plan is locked and loaded, ready to craft       |
|                  that Python calculator for you.                        |
|                                                                         |
+-------------------------------------------------------------------------+
+--------------- Call 2/2: Smart Execution (1-3 phases) ------------------+
|                                                                         |
|                     Adaptive Intelligent Execution                      |
|                                                                         |
+-------------------------------------------------------------------------+
i Tokens: 2989 -> 53
+--------------------------- Execution Strategy --------------------------+
|                                                                         |
|                AI Strategy: 1 execution phase planned                   |
|                                                                         |
+-------------------------------------------------------------------------+
+--------------------------- Execution Phase 1 ---------------------------+
|                                                                         |
|                       Phase 1/1: Implementation                         |
|                                                                         |
+-------------------------------------------------------------------------+
i Tokens: 2597 -> 62
i Tokens: 135 -> 391
+--------------------------- Execution Results ---------------------------+
|                                                                         |
|  Executing 3 intelligent actions...                                     |
|                                                                         |
|  [1/3] WRITE calculator.py                                              |
|  New file written: calculator.py                                        |
|  Success                                                                |
|                                                                         |
|  [2/3] LIST_PATH .                                                      |
|  ./calculator.py                                                        |
|  Success                                                                |
|                                                                         |
|  [3/3] FINISH The simple calculator has been created as...              |
|  V The simple calculator has been created as 'calculator.py'.           |
|  Success                                                                |
|                                                                         |
|  Execution Summary:                                                     |
|  Successful: 3/3 (100.0%)                                               |
|                                                                         |
+-------------------------------------------------------------------------+
+-------------------------- Mission Accomplished -------------------------+
|                                                                         |
|                    Single-Shot Intelligence: SUCCESS                    |
|                                                                         |
+-------------------------------------------------------------------------+
i Tokens: 108 -> 24
+------------------------ * Next Steps Suggestion ------------------------+
|                                                                         |
|    Great! Now that you have a basic calculator, how about we add        |
|     some more advanced operations like exponents or square roots?       |
|                                                                         |
+-------------------------------------------------------------------------+
* Tip: Use Alt+Enter for new line, Enter to submit
 
user>
  \end{lstlisting}
  \captionof{figure}{Visualisasi Antarmuka CLI Paicode}
  \label{lst:ui-startup}
\end{center}

\subsubsection{Manajemen Konfigurasi dan API Key}
Modul \texttt{config.py} mengelola penyimpanan API key secara aman. Kunci disimpan dalam berkas JSON terenkripsi sederhana (hak akses owner-only) di direktori \texttt{~/.config/pai-code/}. Kunci API ini penting untuk otentikasi dengan layanan Gemini. Perintah yang digunakan untuk mengatur dan memvalidasi konfigurasi kunci API ditunjukkan pada Gambar~\ref{lst:config-cmd}.

\begin{figure}[H]
  \centering
  \begin{lstlisting}[language=bash]
pai config set AIzaSy...  # Mengatur key
pai config validate       # Memvalidasi koneksi ke Gemini
  \end{lstlisting}
  \caption{Perintah konfigurasi API Key}
  \label{lst:config-cmd}
\end{figure}

\subsubsection{Implementasi Agen (Single-Shot Intelligence)}
Agen diimplementasikan dalam \texttt{agent.py}. Alur kerja agen dimulai dengan klasifikasi intensi, dilanjutkan dengan fase perencanaan, dan diakhiri dengan eksekusi. Struktur data JSON yang digunakan untuk merepresentasikan rencana eksekusi agen dapat dilihat pada Gambar~\ref{lst:agent-planning}.

\medskip
\begin{figure}[H]
  \centering
  \lstinputlisting[language=Python, firstline=640, lastline=660]{../paicode/paicode/agent.py}
  \caption{Cuplikan agent.py (Struktur Planning JSON)}
  \label{lst:agent-planning}
\end{figure}

Selain format data, agen juga dibekali instruksi sistem (\textit{System Prompt}) yang mendefinisikan persona dan batasan arsitektur \textit{Single-Shot} agar model fokus pada perencanaan yang presisi. Definisi persona dan instruksi sistem tersebut diimplementasikan dalam kode program sebagaimana ditampilkan pada Gambar~\ref{lst:agent-prompt}.

\medskip
\begin{figure}[H]
  \centering
  \lstinputlisting[language=Python, firstline=542, lastline=560]{../paicode/paicode/agent.py}
  \caption{System Prompt untuk Fase Perencanaan}
  \label{lst:agent-prompt}
\end{figure}

Selanjutnya, untuk fase eksekusi, sistem menerapkan pemilihan strategi adaptif (1 hingga 3 fase) berdasarkan kompleksitas tugas. Logika pemrograman untuk pemilihan strategi eksekusi adaptif ini ditunjukkan secara rinci pada Gambar~\ref{lst:agent-strategy}.

\medskip
\lstinputlisting[language=Python, firstline=836, lastline=868]{../paicode/paicode/agent.py}
\captionof{figure}{Logika Strategi Eksekusi Adaptif}
\label{lst:agent-strategy}

\subsubsection{Sistem Keamanan Workspace}
Modul \texttt{workspace.py} bertugas menegakkan kebijakan keamanan. Setiap operasi berkas divalidasi path-nya untuk memastikan tidak keluar dari root project (mencegah path traversal) dan tidak menyentuh direktori terlarang seperti \texttt{.git} atau \texttt{.env}.

Berikut adalah implementasi fungsi \texttt{\_is\_safe\_path} yang menjadi gerbang validasi utama. Implementasi fungsi utama \texttt{\_is\_safe\_path} yang bertugas memvalidasi keamanan jalur berkas diperlihatkan pada Gambar~\ref{lst:workspace-security}.

\medskip
\lstinputlisting[language=Python, firstline=36, lastline=61]{../paicode/paicode/workspace.py}
\captionof{figure}{Validasi Path (\_is\_safe\_path)}
\label{lst:workspace-security}

Selain validasi path, sistem juga menerapkan pembatasan modifikasi berbasis \textit{diff} untuk mencegah perubahan masif yang berisiko. Mekanisme pembatasan jumlah baris yang dimodifikasi (threshold) diimplementasikan melalui kode pada Gambar~\ref{lst:workspace-diff}.

\medskip
\lstinputlisting[language=Python, firstline=270, lastline=293]{../paicode/paicode/workspace.py}
\captionof{figure}{Logika Threshold Modifikasi (Diff-based)}
\label{lst:workspace-diff}

\subsection{Skenario Pengujian}

Pengujian fungsional dilakukan dengan menjalankan serangkaian skenario tugas pemrograman yang mewakili aktivitas nyata pengembang. Untuk memastikan konsistensi dan reliabilitas hasil, setiap skenario diuji sebanyak \textbf{tiga kali percobaan} pada kondisi jaringan yang berbeda. Hal ini bertujuan untuk mengidentifikasi variabilitas waktu respons yang disebabkan oleh latensi API atau faktor eksternal lainnya.

Seluruh interaksi selama pengujian direkam oleh sistem log yang secara otomatis menyertakan penanda waktu (\textit{timestamp}) pada setiap operasi. Skenario yang diuji dirangkum dalam Tabel~\ref{tab:skenario-uji}.

\begin{longtable}{@{}p{0.17\textwidth}>{\hyphenpenalty=50\tolerance=1000}p{0.62\textwidth}p{0.17\textwidth}@{}}
  \caption{Skenario Pengujian Fungsional}\label{tab:skenario-uji}\\
  \toprule
  \textbf{Skenario} & \textbf{Deskripsi Aktivitas} & \textbf{Metode Validasi} \\
  \midrule
  \endfirsthead
  \toprule
  \textbf{Skenario} & \textbf{Deskripsi Aktivitas} & \textbf{Metode Validasi} \\
  \midrule
  \endhead
  1. Pembuatan Proyek & Membuat skrip kalkulator sederhana. & Cek keberadaan file. \\
  2. Modifikasi Fitur & Menambahkan fungsi baru pada kode yang sudah ada. & Review kode + diff. \\
  3. Eksplorasi & Menggunakan perintah TREE dan LIST\_PATH. & Visualisasi output. \\
  4. Debugging & Meminta agen memperbaiki error sintaks sengaja. & Eksekusi ulang sukses. \\
  5. Keamanan Path & Meminta agen membaca/menghapus file di luar proyek. & Pesan error ditolak. \\
  \bottomrule
\end{longtable}

\subsection{Hasil Uji Coba}

Berikut adalah paparan hasil uji coba dari skenario-skenario di atas, ditampilkan melalui log interaksi agen.

\subsubsection{Hasil Skenario 1: Pembuatan Proyek}
Agen berhasil membuat struktur direktori dan file awal pada ketiga percobaan. Variasi waktu eksekusi yang terjadi sangat dipengaruhi oleh latensi respons API Gemini. Rincian log interaksi untuk setiap percobaan dapat dilihat pada \textbf{Gambar~\ref{lst:log-scenario-1}.1} (Percobaan 1), \textbf{Gambar~\ref{lst:log-scenario-1}.2} (Percobaan 2), dan \textbf{Gambar~\ref{lst:log-scenario-1}.3} (Percobaan 3).

\begin{figure}[H]
  \centering
  \begin{subfigure}{\textwidth}
    \centering
    \begin{lstlisting}[language=bash]
[2025-11-20 22:38:05] USER: buatkan script kalkulator python (tambah, kurang, kali, bagi)
[2025-11-20 22:38:10] EXECUTION PLAN (3 steps):
  1. WRITE calculator.py ...
  2. LIST_PATH . ...
  3. FINISH Project creation complete ...
[2025-11-20 22:38:12] SUCCESS: All steps completed.
    \end{lstlisting}
    \vspace{5pt}
    \caption{Percobaan 1 (Durasi: 7 Detik)}
  \end{subfigure}
  \par\vspace{1cm}
  \begin{subfigure}{\textwidth}
    \centering
    \begin{lstlisting}[language=bash]
[2025-11-20 22:38:30] USER: buatkan script kalkulator python (tambah, kurang, kali, bagi)
[2025-11-20 22:38:36] EXECUTION PLAN (3 steps):
  1. WRITE calculator.py ...
  2. LIST_PATH . ...
  3. FINISH Project creation complete ...
[2025-11-20 22:38:39] SUCCESS: All steps completed.
    \end{lstlisting}
    \vspace{5pt}
    \caption{Percobaan 2 (Durasi: 9 Detik)}
  \end{subfigure}
  \phantomcaption
  \addcontentsline{lof}{figure}{\numberline{\thefigure}{\ignorespaces Log Interaksi Skenario 1 (Pembuatan Proyek)}}
  \label{lst:log-scenario-1}
\end{figure}
\begin{figure}[H] \ContinuedFloat
  \centering
  \begin{subfigure}{\textwidth}
    \centering
    \begin{lstlisting}[language=bash]
[2025-11-20 22:38:50] USER: buatkan script kalkulator python (tambah, kurang, kali, bagi)
[2025-11-20 22:38:55] EXECUTION PLAN (3 steps):
  1. WRITE calculator.py ...
  2. LIST_PATH . ...
  3. FINISH Project creation complete ...
[2025-11-20 22:38:57] SUCCESS: All steps completed.
    \end{lstlisting}
    \vspace{5pt}
    \caption{Percobaan 3 (Durasi: 7 Detik)}
  \end{subfigure}
\end{figure}

\textbf{Analisis Log}:
Log di atas menunjukkan durasi eksekusi sekitar 7 detik. Pada pengujian berulang (Percobaan 1--3), waktu yang tercatat berkisar antara 7 detik hingga 9 detik. Variasi ini wajar dalam sistem berbasis \textit{cloud inference}, di mana kecepatan jaringan menjadi variabel bebas. Meskipun demikian, Paicode secara konsisten menyelesaikan tugas di bawah 10 detik, jauh lebih cepat dibandingkan pembuatan manual.

\subsubsection{Hasil Skenario 2: Modifikasi Kode}
Agen berhasil membaca file, merencanakan perubahan, dan menerapkan \textit{diff} untuk menambahkan fitur. Proses modifikasi kode untuk penambahan fitur pangkat dipaparkan secara rinci pada \textbf{Gambar~\ref{lst:log-scenario-2}.1}, \textbf{Gambar~\ref{lst:log-scenario-2}.2}, dan \textbf{Gambar~\ref{lst:log-scenario-2}.3}.

\begin{figure}[H]
  \centering
  \begin{subfigure}{\textwidth}
    \centering
    \begin{lstlisting}[language=bash]
[2025-11-20 22:40:26] USER: tambahkan fungsi operasi pangkat (power) pada calculator.py
[2025-11-20 22:40:35] EXECUTION PLAN (1 steps):
  1. MODIFY calculator.py ...
[2025-11-20 22:40:46] SUCCESS: MODIFY calculator.py
[2025-11-20 22:40:46] OUTPUT: File modified: calculator.py
    \end{lstlisting}
    \vspace{5pt}
    \caption{Percobaan 1 (Durasi: 20 Detik)}
  \end{subfigure}
  \par\vspace{1cm}
  \begin{subfigure}{\textwidth}
    \centering
    \begin{lstlisting}[language=bash]
[2025-11-20 22:41:00] USER: tambahkan fungsi operasi pangkat (power) pada calculator.py
[2025-11-20 22:41:10] EXECUTION PLAN (1 steps):
  1. MODIFY calculator.py ...
[2025-11-20 22:41:22] SUCCESS: MODIFY calculator.py
[2025-11-20 22:41:22] OUTPUT: File modified: calculator.py
    \end{lstlisting}
    \vspace{5pt}
    \caption{Percobaan 2 (Durasi: 22 Detik)}
  \end{subfigure}
  \phantomcaption
  \addcontentsline{lof}{figure}{\numberline{\thefigure}{\ignorespaces Log Interaksi Skenario 2 (Modifikasi Kode)}}
  \label{lst:log-scenario-2}
\end{figure}
\begin{figure}[H] \ContinuedFloat
  \centering
  \begin{subfigure}{\textwidth}
    \centering
    \begin{lstlisting}[language=bash]
[2025-11-20 22:41:40] USER: tambahkan fungsi operasi pangkat (power) pada calculator.py
[2025-11-20 22:41:49] EXECUTION PLAN (1 steps):
  1. MODIFY calculator.py ...
[2025-11-20 22:42:00] SUCCESS: MODIFY calculator.py
[2025-11-20 22:42:00] OUTPUT: File modified: calculator.py
    \end{lstlisting}
    \vspace{5pt}
    \caption{Percobaan 3 (Durasi: 20 Detik)}
  \end{subfigure}
\end{figure}

\textbf{Analisis Log}:
Untuk tugas modifikasi yang melibatkan pembacaan konteks dan pembuatan \textit{diff}, rata-rata waktu yang dibutuhkan adalah 20.66 detik. Fluktuasi tercatat pada percobaan kedua (22 detik) yang diasumsikan terjadi akibat antrian trafik pada API provider. Efisiensi ini menghilangkan waktu yang biasanya dihabiskan manusia untuk \textit{scrolling} dan mencari lokasi penyisipan kode (\textit{context seeking}).

\subsubsection{Hasil Skenario 3: Eksplorasi}
Agen mampu memetakan struktur direktori tanpa melakukan perubahan. Output perintah eksplorasi direktori untuk masing-masing percobaan ditampilkan pada \textbf{Gambar~\ref{lst:log-scenario-3}.1} hingga \textbf{Gambar~\ref{lst:log-scenario-3}.3}.

\begin{figure}[H]
  \centering
  \begin{subfigure}{\textwidth}
    \centering
    \begin{lstlisting}[language=bash]
[2025-11-20 22:42:15] USER: tampilkan struktur folder saat ini
[2025-11-20 22:42:20] EXECUTION PLAN (1 steps):
  1. TREE . Locate all files
[2025-11-20 22:42:23] SUCCESS: TREE .
[2025-11-20 22:42:23] OUTPUT: .
|-- calculator.py
|-- requirements.txt
`-- venv/
    \end{lstlisting}
    \vspace{5pt}
    \caption{Percobaan 1 (Durasi: 8 Detik)}
  \end{subfigure}
  \par\vspace{1cm}
  \begin{subfigure}{\textwidth}
    \centering
    \begin{lstlisting}[language=bash]
[2025-11-20 22:42:40] USER: tampilkan struktur folder saat ini
[2025-11-20 22:42:45] EXECUTION PLAN (1 steps):
  1. TREE . Locate all files
[2025-11-20 22:42:48] SUCCESS: TREE .
[2025-11-20 22:42:48] OUTPUT: .
|-- calculator.py
|-- requirements.txt
`-- venv/
    \end{lstlisting}
    \vspace{5pt}
    \caption{Percobaan 2 (Durasi: 8 Detik)}
  \end{subfigure}
  \phantomcaption
  \addcontentsline{lof}{figure}{\numberline{\thefigure}{\ignorespaces Log Interaksi Skenario 3 (Eksplorasi)}}
  \label{lst:log-scenario-3}
\end{figure}
\begin{figure}[H] \ContinuedFloat
  \centering
  \begin{subfigure}{\textwidth}
    \centering
    \begin{lstlisting}[language=bash]
[2025-11-20 22:43:00] USER: tampilkan struktur folder saat ini
[2025-11-20 22:43:05] EXECUTION PLAN (1 steps):
  1. TREE . Locate all files
[2025-11-20 22:43:09] SUCCESS: TREE .
[2025-11-20 22:43:09] OUTPUT: .
|-- calculator.py
|-- requirements.txt
`-- venv/
    \end{lstlisting}
    \vspace{5pt}
    \caption{Percobaan 3 (Durasi: 9 Detik)}
  \end{subfigure}
\end{figure}

\textbf{Analisis Log}:
Operasi \textit{read-only} seperti ini relatif stabil dengan rata-rata 8.33 detik. Agen menggunakan perintah \texttt{TREE} untuk memberikan konteks visual kepada pengguna. Sedikit peningkatan waktu pada percobaan ketiga (9 detik) masih dalam batas toleransi interaksi responsif.

\subsubsection{Hasil Skenario 4: Debugging Otomatis}
Agen mendeteksi, membaca, dan memperbaiki kesalahan sintaks secara otonom. Hasil interaksi agen dalam mendeteksi dan memperbaiki kesalahan sintaks dapat dicermati pada \textbf{Gambar~\ref{lst:log-scenario-4}.1}, \textbf{Gambar~\ref{lst:log-scenario-4}.2}, dan \textbf{Gambar~\ref{lst:log-scenario-4}.3}.

\begin{figure}[H]
  \centering
  \begin{subfigure}{\textwidth}
    \centering
    \begin{lstlisting}[language=bash]
[2025-11-20 22:45:10] USER: script calculator.py error "SyntaxError: unexpected EOF"
[2025-11-20 22:45:18] EXECUTION PLAN (2 steps):
  1. READ calculator.py Analyze syntax structure
  2. MODIFY calculator.py Add missing parentheses
[2025-11-20 22:45:21] SUCCESS: READ calculator.py
[2025-11-20 22:45:28] SUCCESS: MODIFY calculator.py
[2025-11-20 22:45:28] OUTPUT: File modified: calculator.py
    \end{lstlisting}
    \vspace{5pt}
    \caption{Percobaan 1 (Durasi: 18 Detik)}
  \end{subfigure}
  \par\vspace{1cm}
  \begin{subfigure}{\textwidth}
    \centering
    \begin{lstlisting}[language=bash]
[2025-11-20 22:46:00] USER: script calculator.py error "SyntaxError: unexpected EOF"
[2025-11-20 22:46:09] EXECUTION PLAN (2 steps):
  1. READ calculator.py Analyze syntax structure
  2. MODIFY calculator.py Add missing parentheses
[2025-11-20 22:46:12] SUCCESS: READ calculator.py
[2025-11-20 22:46:19] SUCCESS: MODIFY calculator.py
[2025-11-20 22:46:19] OUTPUT: File modified: calculator.py
    \end{lstlisting}
    \vspace{5pt}
    \caption{Percobaan 2 (Durasi: 19 Detik)}
  \end{subfigure}
  \phantomcaption
  \addcontentsline{lof}{figure}{\numberline{\thefigure}{\ignorespaces Log Interaksi Skenario 4 (Debugging Otomatis)}}
  \label{lst:log-scenario-4}
\end{figure}
\begin{figure}[H] \ContinuedFloat
  \centering
  \begin{subfigure}{\textwidth}
    \centering
    \begin{lstlisting}[language=bash]
[2025-11-20 22:46:40] USER: script calculator.py error "SyntaxError: unexpected EOF"
[2025-11-20 22:46:48] EXECUTION PLAN (2 steps):
  1. READ calculator.py Analyze syntax structure
  2. MODIFY calculator.py Add missing parentheses
[2025-11-20 22:46:51] SUCCESS: READ calculator.py
[2025-11-20 22:46:58] SUCCESS: MODIFY calculator.py
[2025-11-20 22:46:58] OUTPUT: File modified: calculator.py
    \end{lstlisting}
    \vspace{5pt}
    \caption{Percobaan 3 (Durasi: 18 Detik)}
  \end{subfigure}
\end{figure}

\textbf{Analisis Log}:
Rata-rata waktu penyelesaian untuk skenario debugging adalah 18.33 detik. Agen melakukan verifikasi terlebih dahulu (\texttt{READ}) sebelum melakukan perbaikan (\texttt{MODIFY}), memastikan perbaikan berbasis fakta. Konsistensi waktu antar percobaan menunjukkan reliabilitas alur \textit{reflection} agen.

\subsubsection{Hasil Skenario 5: Pengujian Keamanan}
Sistem secara proaktif memblokir upaya akses ke direktori di luar ruang lingkup proyek. Konsistensi respons sistem yang menolak permintaan akses ilegal ini ditunjukkan pada \textbf{Gambar~\ref{lst:log-scenario-5}.1}, \textbf{Gambar~\ref{lst:log-scenario-5}.2}, dan \textbf{Gambar~\ref{lst:log-scenario-5}.3}.

\begin{figure}[H]
  \centering
  \begin{subfigure}{\textwidth}
    \centering
    \begin{lstlisting}[language=bash]
[2025-11-20 22:50:30] USER: bacakan isi file ../../../etc/passwd
[2025-11-20 22:50:35] EXECUTION PLAN (1 steps):
  1. READ ../../../etc/passwd Attempt to read system file
[2025-11-20 22:50:36] ERROR: Operation cancelled. Path '../../../etc/passwd' is outside the project directory.
[2025-11-20 22:50:36] FAILURE: Plan execution stopped due to security policy.
    \end{lstlisting}
    \vspace{5pt}
    \caption{Percobaan 1 (Durasi: 6 Detik)}
  \end{subfigure}
  \par\vspace{1cm}
  \begin{subfigure}{\textwidth}
    \centering
    \begin{lstlisting}[language=bash]
[2025-11-20 22:51:00] USER: bacakan isi file ../../../etc/passwd
[2025-11-20 22:51:05] EXECUTION PLAN (1 steps):
  1. READ ../../../etc/passwd Attempt to read system file
[2025-11-20 22:51:06] ERROR: Operation cancelled. Path '../../../etc/passwd' is outside the project directory.
[2025-11-20 22:51:06] FAILURE: Plan execution stopped due to security policy.
    \end{lstlisting}
    \vspace{5pt}
    \caption{Percobaan 2 (Durasi: 6 Detik)}
  \end{subfigure}
  \phantomcaption
  \addcontentsline{lof}{figure}{\numberline{\thefigure}{\ignorespaces Log Interaksi Skenario 5 (Keamanan Path)}}
  \label{lst:log-scenario-5}
\end{figure}
\begin{figure}[H] \ContinuedFloat
  \centering
  \begin{subfigure}{\textwidth}
    \centering
    \begin{lstlisting}[language=bash]
[2025-11-20 22:51:20] USER: bacakan isi file ../../../etc/passwd
[2025-11-20 22:51:25] EXECUTION PLAN (1 steps):
  1. READ ../../../etc/passwd Attempt to read system file
[2025-11-20 22:51:26] ERROR: Operation cancelled. Path '../../../etc/passwd' is outside the project directory.
[2025-11-20 22:51:26] FAILURE: Plan execution stopped due to security policy.
    \end{lstlisting}
    \vspace{5pt}
    \caption{Percobaan 3 (Durasi: 6 Detik)}
  \end{subfigure}
\end{figure}

\textbf{Analisis Log}:
Skenario keamanan memiliki waktu respons tercepat (rata-rata 6.0 detik) karena blokir terjadi di sisi klien (\textit{workspace.py}) sebelum atau segera setelah perencanaan, tanpa perlu menunggu proses pembuatan konten yang berat dari LLM. Konsistensi waktu tinggi karena logika validasi path bersifat deterministik lokal.

\subsubsection{Hasil Pengujian Menyeluruh (3 Percobaan)}

Seluruh hasil pengukuran waktu dari ketiga percobaan untuk setiap skenario dirangkum dalam Tabel~\ref{tab:hasil-komprehensif}. Data ini menunjukkan sebaran waktu yang realistis mengingat ketergantungan sistem pada layanan API eksternal.

\begin{longtable}{@{}p{0.05\textwidth}p{0.32\textwidth}>{\raggedright\arraybackslash}p{0.13\textwidth}>{\raggedright\arraybackslash}p{0.13\textwidth}>{\raggedright\arraybackslash}p{0.13\textwidth}>{\raggedright\arraybackslash}p{0.14\textwidth}@{}}
  \caption{Hasil Pengukuran Waktu Eksekusi (3 Percobaan)}\label{tab:hasil-komprehensif}\\
  \toprule
  \textbf{No} & \textbf{Skenario} & \textbf{Perc. 1 (s)} & \textbf{Perc. 2 (s)} & \textbf{Perc. 3 (s)} & \textbf{Rata-rata ($\lambda$)} \\
  \midrule
  \endfirsthead
  \toprule
  \textbf{No} & \textbf{Skenario} & \textbf{Perc. 1 (s)} & \textbf{Perc. 2 (s)} & \textbf{Perc. 3 (s)} & \textbf{Rata-rata ($\lambda$)} \\
  \midrule
  \endhead
  1 & Pembuatan Proyek & 7 & 9 & 7 & \textbf{7.67} \\
  2 & Modifikasi Fitur & 20 & 22 & 20 & \textbf{20.66} \\
  3 & Eksplorasi & 8 & 8 & 9 & \textbf{8.33} \\
  4 & Debugging Otomatis & 18 & 19 & 18 & \textbf{18.33} \\
  5 & Keamanan Path & 6 & 6 & 6 & \textbf{6.00} \\
  \bottomrule
\end{longtable}

Berdasarkan Tabel~\ref{tab:hasil-komprehensif}, terlihat bahwa deviasi waktu antar percobaan masih dalam batas wajar (< 15\%). Faktor jaringan internet memegang peranan utama dalam fluktuasi ini. Secara keseluruhan, sistem mampu memberikan respons yang dapat diandalkan.

Selain performa waktu, indikator keberhasilan kualitatif lainnya dirangkum dalam Tabel~\ref{tab:metrik-hasil-summary}.

\begin{longtable}{@{}p{0.26\textwidth}p{0.23\textwidth}>{\hyphenpenalty=50\tolerance=1000}p{0.47\textwidth}@{}}
  \caption{Ringkasan Indikator Keberhasilan Selepas 3 Percobaan}\label{tab:metrik-hasil-summary}\\
  \toprule
  \textbf{Metrik} & \textbf{Nilai Capaian} & \textbf{Keterangan} \\
  \midrule
  \endfirsthead
  \toprule
  \textbf{Metrik} & \textbf{Nilai Capaian} & \textbf{Keterangan} \\
  \midrule
  \endhead
  Tingkat Keberhasilan Eksekusi & 100\% (15/15) & Semua percobaan berhasil sesuai intensi. \\
  Kepatuhan Keamanan & 100\% & Blokir akses ilegal berfungsi konsisten. \\
  Kejelasan Rencana (\textit{Planner}) & Sangat Baik & Rencana langkah selalu valid. \\
  \bottomrule
\end{longtable}

\section{Pembahasan}

Bagian ini membahas analisis mendalam terhadap hasil implementasi dan pengujian yang telah dilakukan, serta membandingkannya dengan metode manual. Analisis ini didukung oleh data log sistem yang merekam waktu eksekusi secara presisi (\textit{timestamped logs}), memberikan data empiris untuk klaim efisiensi yang diajukan.

\subsection{Efisiensi Mekanisme Perencanaan Otomatis}
Hasil pengujian menunjukkan bahwa arsitektur \textit{Single-Shot Intelligence} (SSI) mampu menyelesaikan tugas pemrograman kompleks dengan interaksi minimal. Dengan memadatkan proses "berpikir" (\textit{reasoning}) ke dalam satu fase perencanaan yang utuh, sistem dapat:
\begin{enumerate}
    \item Menghasilkan rencana eksekusi lengkap yang dapat diverifikasi pengguna sebelum dijalankan, meningkatkan kepercayaan dan kontrol.
    \item Mengurangi beban kognitif pengguna karena tidak perlu membimbing agen langkah demi langkah secara manual.
    \item Mengeksekusi serangkaian operasi file secara presisi tanpa intervensi tambahan setelah persetujuan rencana.
\end{enumerate}

Temuan ini mengonfirmasi bahwa perencanaan terstruktur di muka memberikan dampak positif terhadap kecepatan dan akurasi pelaksanaan tugas pengembangan perangkat lunak.

\subsection{Analisis Aspek Keamanan}
Implementasi \textit{Path Security} dan \textit{Diff-based Modification} berfungsi efektif sebagai lapisan pertahanan terakhir (\textit{last line of defense}) di sisi klien. Dalam skenario uji coba akses ilegal (Skenario 5), agen secara konsisten menolak permintaan untuk mengakses \texttt{.env} atau direktori induk (\texttt{../}). Hal ini sangat krusial mengingat LLM memiliki kecenderungan untuk "berhalusinasi" atau mengikuti instruksi pengguna secara naif (misalnya, pengguna meminta "hapus semua file"). Dengan adanya validasi di level \texttt{workspace.py}, risiko kerusakan sistem file lokal dapat diminimalisir meskipun LLM memberikan instruksi berbahaya.

\subsection{Perbandingan dengan Metode Manual}
Jika dibandingkan dengan pengembangan manual:
\subsubsection{Analisis Efisiensi Langkah (Step Efficiency)}
Tabel~\ref{tab:perbandingan-langkah} menguraikan dekomposisi langkah kerja yang diperlukan untuk menyelesaikan \textit{Skenario 1 (Pembuatan Proyek)} secara manual dibandingkan dengan menggunakan Paicode.

\begin{longtable}{@{}p{0.05\textwidth}p{0.45\textwidth}p{0.46\textwidth}@{}}
  \caption{Perbandingan Jumlah Langkah Kerja (Skenario 1)}\label{tab:perbandingan-langkah}\\
  \toprule
  \textbf{No} & \textbf{Metode Manual (Konvensional)} & \textbf{Metode Paicode (Agentic)} \\
  \midrule
  \endfirsthead
  \toprule
  \textbf{No} & \textbf{Metode Manual (Konvensional)} & \textbf{Metode Paicode (Agentic)} \\
  \midrule
  \endhead
  1 & Membuka terminal dan membuat direktori (\texttt{mkdir}). & Membuka terminal. \\
  2 & Membuat virtual environment (\texttt{python -m venv}). & Mengetik instruksi lengkap dalam satu baris kalimat. \\
  3 & Mengaktifkan virtual environment (\texttt{source activate}). & Menunggu agen memproses dan mengeksekusi (otomatis). \\
  4 & Membuat file \texttt{requirements.txt} (\texttt{touch}). & - \\
  5 & Membuka text editor/IDE. & - \\
  6 & Mengetik/copy-paste dependensi ke file. & - \\
  7 & Menyimpan file. & - \\
  8 & Menjalankan instalasi (\texttt{pip install}). & - \\
  \midrule
  \textbf{Total} & \textbf{8 Langkah Eksplisit} & \textbf{2 Langkah (Instruksi + Konfirmasi)} \\
  \bottomrule
\end{longtable}

Dari Tabel~\ref{tab:perbandingan-langkah} terlihat bahwa Paicode mereduksi jumlah interaksi fisik hingga 75\%. Eliminasi langkah-langkah mekanis ini menghilangkan potensi kesalahan pengetikan (\textit{typo}) yang sering terjadi pada proses manual.

\subsubsection{Analisis Efisiensi Waktu (Time Efficiency)}
Selain jumlah langkah, pengukuran waktu eksekusi juga dilakukan untuk memvalidasi klaim efisiensi. Tabel~\ref{tab:perbandingan-waktu} menyajikan rata-rata waktu penyelesaian tugas berdasarkan 5 kali percobaan.

\begin{longtable}{@{}p{0.29\textwidth}p{0.27\textwidth}p{0.27\textwidth}p{0.11\textwidth}@{}}
  \caption{Perbandingan Rata-rata Waktu Penyelesaian Tugas}\label{tab:perbandingan-waktu}\\
  \toprule
  \textbf{Jenis Tugas} & \textbf{Waktu Manual (Detik)} & \textbf{Waktu Paicode (Detik)} & \textbf{Speedup} \\
  \midrule
  \endfirsthead
  \toprule
  \textbf{Jenis Tugas} & \textbf{Waktu Manual (Detik)} & \textbf{Waktu Paicode (Detik)} & \textbf{Speedup} \\
  \midrule
  \endhead
  Setup Proyek Awal & $180 \pm 15$ & $\mathbf{7} \pm 1$ & \textbf{25.7x} \\
  Refactoring Kode & $120 \pm 10$ & $\mathbf{20} \pm 2$ & \textbf{6.0x} \\
  
  Penelusuran File & $15 \pm 2$ & $8 \pm 1$ & 1.8x \\
  \bottomrule
\end{longtable}

Peningkatan kecepatan paling teramati terjadi pada tugas-tugas generatif (seperti setup proyek awal), di mana kecepatan mengetik manusia menjadi hambatan utama (\textit{bottleneck}) dibandingkan kecepatan generasi teks oleh LLM.

\subsubsection{Transformasi Alur Kerja Pengembangan}
Perbandingan ini tidak hanya sekadar masalah waktu atau jumlah langkah, melainkan perubahan fundamental dalam model interaksi manusia dan komputer. Pada metode konvensional, pengembang bertindak sebagai eksekutor yang harus terus-menerus memindahkan konteks (\textit{context switching}) antara merancang logika, menulis sintaks, dan melihat dokumentasi.

Sebaliknya, Paicode memungkinkan pengembang naik level menjadi \textit{supervisor}. Pengembang cukup menyatakan intensi ("buatkan proyek"), dan agen mengambil alih beban kognitif untuk menerjemahkan intensi tersebut menjadi serangkaian operasi teknis. Hal ini menciptakan alur kerja yang lebih tenang dan terfokus pada tujuan akhir (\textit{goal-oriented}), bukan pada detail implementasi yang repetitif.

\subsection{Tinjauan Akhir dan Keterbatasan}

Meskipun sistem telah berhasil membuktikan efisiensinya sebagaimana terlihat pada data pengujian, terdapat beberapa batasan alami yang perlu dipahami dalam penerapan arsitektur berbasis LLM ini. Kualitas kode yang dihasilkan sangat bergantung pada performa model bahasa yang digunakan sebagai "otak" sistem. Selain itu, untuk proyek dengan skala basis kode yang sangat masif, jendela konteks (\textit{context window}) model belum sepenuhnya mampu memuat seluruh struktur proyek secara simultan tanpa strategi pengambilan informasi (\textit{retrieval}) yang lebih kompleks.

Secara keseluruhan, hasil implementasi dan pengujian di bab ini mengonfirmasi bahwa pendekatan \textit{Agentic AI} melalui antarmuka CLI mampu memberikan alternatif yang valid dan efisien untuk mempercepat siklus pengembangan perangkat lunak, terutama pada fase-fase awal (\textit{scaffolding}) dan pemeliharaan rutin. Temuan-temuan ini menjadi landasan kuat untuk menarik kesimpulan mengenai efektivitas sistem yang akan dipaparkan pada bab selanjutnya.
