% Chapter 4 - Implementasi dan Pembahasan
\chapter{IMPLEMENTASI DAN PEMBAHASAN}
\label{chap:implementasi}
\setlength{\parindent}{0pt} % Disable standard indentation locally to avoid double-indenting with \customindent

\section{Implementasi dan Uji Coba Sistem}
\customindent Bagian ini menguraikan tahapan realisasi sistem Paicode, mulai dari konfigurasi lingkungan, implementasi kode program, hingga hasil pengujian fungsional.

\subsection{Lingkungan Implementasi}
\customindent Sistem diimplementasikan pada lingkungan sistem operasi Ubuntu dengan spesifikasi konfigurasi seperti tercantum pada Tabel~\ref{tab:konfigurasi}.

\begin{longtable}{|p{0.30\textwidth}|p{0.64\textwidth}|}
  \caption{Konfigurasi Lingkungan Implementasi}\label{tab:konfigurasi}\\
  \hline
  \textbf{Komponen} & \textbf{Spesifikasi} \\
  \hline
  \endfirsthead
  \hline
  \textbf{Komponen} & \textbf{Spesifikasi} \\
  \hline
  \endhead
  Sistem Operasi & Ubuntu (Linux) \\ \hline
  Python & \texttt{\textgreater= 3.10} \\ \hline
  Manajer Dependensi & pip dan virtual environment \\ \hline
  LLM Provider & Gemini (via \texttt{google-generativeai}) \\ \hline
  Antarmuka Terminal & \texttt{rich} (output) dan \texttt{prompt\_toolkit} (input) \\ \hline
  Hardware & CPU x86\_64, RAM 8+ GB \\ \hline
\end{longtable}

\customindent Proses instalasi dilakukan menggunakan \texttt{Makefile} yang mengotomasi pembuatan virtual environment dan instalasi dependensi dari berkas \texttt{requirements.txt} dan \texttt{setup.cfg}.

\subsection{Implementasi Fitur Utama}
\customindent Implementasi inti Paicode berpusat pada arsitektur \textit{Single-Shot Intelligence} yang terbagi menjadi beberapa modul utama seperti yang telah dirancang pada Bab III.

\subsubsection{Antarmuka Pengguna}
\customindent Antarmuka pengguna Paicode dibangun berbasis teks (\textit{Command Line Interface}) dengan memanfaatkan pustaka \texttt{rich} untuk menyajikan visualisasi yang terstruktur dan informatif pada terminal. Desain antarmuka mengutamakan kejelasan informasi dengan penggunaan panel, warna (pada terminal yang mendukung), dan tata letak yang rapi. Tampilan awal saat sesi interaktif dimulai direpresentasikan dalam bentuk ASCII pada Gambar~\ref{lst:ui-startup}.

\begin{center}
  \begin{lstlisting}[basicstyle=\ttfamily\scriptsize]
user@localhost:~/space/univ/skripsi/devpai/trypai$ pai
+------------------------- Interactive Auto Mode -------------------------+
|                                                                         |
|         Welcome! I'm Pai, your agentic AI coding companion. *           |
|      Now powered by Single-Shot Intelligence for maximum efficiency.    |
|              [info]Type 'exit' or 'quit' to leave.[/info]               |
|          [info]Each request uses exactly 2 API calls.[/info]            |
|       [info]* Multi-line input: Alt+Enter for new line,                 |
|                        Enter to submit.[/info]                          |
+-------------------------------------------------------------------------+
* Tip: Use Alt+Enter for new line, Enter to submit
 
user> hai pai, make me a simple calculator in python.
i Tokens: 194 -> 1
i Tokens: 170 -> 21
+---------------------------------- Pai ----------------------------------+
|    Sure thing! A simple Python calculator, got it. Let me cook up       |
|                       a smart plan for that!                            |
+-------------------------------------------------------------------------+
+------------------ Call 1/2: Intelligence Planning ----------------------+
|                        Deep Analysis & Planning                         |
|                                                                         |
|                                                                         |
+-------------------------------------------------------------------------+
i Tokens: 1853 -> 1138
+---------------------------- Planning Results ---------------------------+
|  Smart Analysis Results:                                                |
|  Intent: Create a simple calculator program in Python.                  |
|  Context Usage: No context utilized                                     |
|  Files to read: 0                                                       |
|  Files to create: 1                                                     |
|  Files to modify: 0                                                     |
|  Efficiency: Standard approach                                          |
|                                                                         |
|  ((Planning analysis in progress...))                                   |
|  Execution Plan: 3 steps                                                |
|  +----+-----------------+------------------------------------------+    |
|  | No | Action          | Purpose                                  |    |
|  +----+-----------------+------------------------------------------+    |
|  | 1  | WRITE calcul... | Create a new Python file containing...   |    |
|  | 2  | LIST_PATH .     | Verify the creation of 'calculator...    |    |
|  | 3  | FINISH calcu... | Inform the user that the simple...       |    |
|  +----+-----------------+------------------------------------------+    |
|  Intelligence Assessment:                                               |
|  Complexity: simple                                                     |
|  Estimated time: 1 minute                                               |
+-------------------------------------------------------------------------+
i Tokens: 168 -> 20
+---------------------------------- Pai ----------------------------------+
|    Consider it done! My plan is locked and loaded, ready to craft       |
|                  that Python calculator for you.                        |
+-------------------------------------------------------------------------+
+--------------- Call 2/2: Smart Execution (1-3 phases) ------------------+
|                     Adaptive Intelligent Execution                      |
+-------------------------------------------------------------------------+
i Tokens: 2989 -> 53
+--------------------------- Execution Strategy --------------------------+
|                AI Strategy: 1 execution phase planned                   |
|                                                                         |
+-------------------------------------------------------------------------+
+--------------------------- Execution Phase 1 ---------------------------+
|                       Phase 1/1: Implementation                         |
+-------------------------------------------------------------------------+
i Tokens: 2597 -> 62
i Tokens: 135 -> 391
+--------------------------- Execution Results ---------------------------+
|  Executing 3 intelligent actions...                                     |
|  [1/3] WRITE calculator.py                                              |
|  New file written: calculator.py                                        |
|  Success                                                                |
|                                                                         |
|  ((Executing detailed implementation steps...))                         |
|  [2/3] LIST_PATH .                                                      |
|  ./calculator.py                                                        |
|  Success                                                                |
|  [3/3] FINISH The simple calculator has been created as...              |
|  V The simple calculator has been created as 'calculator.py'.           |
|  Success                                                                |
|  Execution Summary:                                                     |
|  Successful: 3/3 (100.0%)                                               |
+-------------------------------------------------------------------------+
+-------------------------- Mission Accomplished -------------------------+
|                    Single-Shot Intelligence: SUCCESS                    |
+-------------------------------------------------------------------------+
i Tokens: 108 -> 24
+------------------------ * Next Steps Suggestion ------------------------+
|    Great! Now that you have a basic calculator, how about we add        |
|     some more advanced operations like exponents or square roots?       |
+-------------------------------------------------------------------------+
* Tip: Use Alt+Enter for new line, Enter to submit
 
user>
  \end{lstlisting}
  \captionof{figure}{Visualisasi Antarmuka CLI Paicode}
  \label{lst:ui-startup}
\end{center}

\subsubsection{Manajemen Konfigurasi dan API Key}
\customindent Modul \texttt{config.py} mengelola penyimpanan API key secara aman. Kunci disimpan dalam berkas JSON terenkripsi sederhana (hak akses owner-only) di direktori \texttt{~/.config/pai-code/}. Kunci API ini penting untuk otentikasi dengan layanan Gemini. Perintah yang digunakan untuk mengatur dan memvalidasi konfigurasi kunci API ditunjukkan pada Gambar~\ref{lst:config-cmd}.

\begin{figure}[H]
  \centering
  \begin{lstlisting}[language=bash, basicstyle=\ttfamily\scriptsize]
pai config set AIzaSy...  # Mengatur key
pai config validate       # Memvalidasi koneksi ke Gemini
  \end{lstlisting}
  \caption{Perintah konfigurasi API Key}
  \label{lst:config-cmd}
\end{figure}

\subsubsection{Implementasi Agen (Single-Shot Intelligence)}
\customindent Agen diimplementasikan dalam \texttt{agent.py}. Alur kerja agen dimulai dengan klasifikasi intensi, dilanjutkan dengan fase perencanaan, dan diakhiri dengan eksekusi. Struktur data JSON yang digunakan untuk merepresentasikan rencana eksekusi agen dapat dilihat pada Gambar~\ref{lst:agent-planning}.

\medskip
\medskip
\lstinputlisting[language=Python, basicstyle=\ttfamily\scriptsize, firstline=640, lastline=660]{../paicode/paicode/agent.py}
\captionof{figure}{Cuplikan agent.py (Struktur Planning JSON)}
\label{lst:agent-planning}

\customindent Selain format data, agen juga dibekali instruksi sistem (\textit{System Prompt}) yang mendefinisikan persona dan batasan arsitektur \textit{Single-Shot} agar model fokus pada perencanaan yang presisi. Definisi persona dan instruksi sistem tersebut diimplementasikan dalam kode program sebagaimana ditampilkan pada Gambar~\ref{lst:agent-prompt}.

\medskip
\lstinputlisting[language=Python, basicstyle=\ttfamily\scriptsize, firstline=542, lastline=560]{../paicode/paicode/agent.py}
\captionof{figure}{System Prompt untuk Fase Perencanaan}
\label{lst:agent-prompt}

\customindent Selanjutnya, untuk fase eksekusi, sistem menerapkan pemilihan strategi adaptif (1 hingga 3 fase) berdasarkan kompleksitas tugas. Logika pemrograman untuk pemilihan strategi eksekusi adaptif ini ditunjukkan secara rinci pada Gambar~\ref{lst:agent-strategy}.

\medskip
\lstinputlisting[language=Python, basicstyle=\ttfamily\scriptsize, firstline=836, lastline=868]{../paicode/paicode/agent.py}
\captionof{figure}{Logika Strategi Eksekusi Adaptif}
\label{lst:agent-strategy}

\subsubsection{Sistem Keamanan Workspace}
\customindent Modul \texttt{workspace.py} bertugas menegakkan kebijakan keamanan. Setiap operasi berkas divalidasi path-nya untuk memastikan tidak keluar dari root project (mencegah path traversal) dan tidak menyentuh direktori terlarang seperti \texttt{.git} atau \texttt{.env}.

\customindent Berikut adalah implementasi fungsi \texttt{\_is\_safe\_path} yang menjadi gerbang validasi utama. Implementasi fungsi utama \texttt{\_is\_safe\_path} yang bertugas memvalidasi keamanan jalur berkas diperlihatkan pada Gambar~\ref{lst:workspace-security}.

\medskip
\lstinputlisting[language=Python, basicstyle=\ttfamily\scriptsize, firstline=36, lastline=61]{../paicode/paicode/workspace.py}
\captionof{figure}{Validasi Path (\_is\_safe\_path)}
\label{lst:workspace-security}

\customindent Selain validasi path, sistem juga menerapkan pembatasan modifikasi berbasis \textit{diff} untuk mencegah perubahan masif yang berisiko. Mekanisme pembatasan jumlah baris yang dimodifikasi (threshold) diimplementasikan melalui kode pada Gambar~\ref{lst:workspace-diff}.

\medskip
\lstinputlisting[language=Python, basicstyle=\ttfamily\scriptsize, firstline=270, lastline=293]{../paicode/paicode/workspace.py}
\captionof{figure}{Logika Threshold Modifikasi (Diff-based)}
\label{lst:workspace-diff}

\subsection{Skenario Pengujian}
\customindent Pengujian fungsional dilakukan dengan menjalankan serangkaian skenario tugas pemrograman yang mewakili aktivitas nyata pengembang. Untuk memastikan konsistensi dan reliabilitas hasil, setiap skenario diuji sebanyak \textbf{tiga kali percobaan} pada kondisi jaringan yang berbeda. Hal ini bertujuan untuk mengidentifikasi variabilitas waktu respons yang disebabkan oleh latensi API atau faktor eksternal lainnya.

\customindent Seluruh interaksi selama pengujian direkam oleh sistem log yang secara otomatis menyertakan penanda waktu (\textit{timestamp}) pada setiap operasi. Skenario yang diuji dirangkum dalam Tabel~\ref{tab:skenario-uji}.

\begin{longtable}{|p{0.16\textwidth}|>{\hyphenpenalty=50\tolerance=1000}p{0.60\textwidth}|p{0.16\textwidth}|}
  \caption{Skenario Pengujian Fungsional}\label{tab:skenario-uji}\\
  \hline
  \textbf{Skenario} & \textbf{Deskripsi Aktivitas} & \textbf{Metode Validasi} \\
  \hline
  \endfirsthead
  \hline
  \textbf{Skenario} & \textbf{Deskripsi Aktivitas} & \textbf{Metode Validasi} \\
  \hline
  \endhead
  1. Pembuatan Proyek & Membuat skrip kalkulator sederhana. & Cek keberadaan file. \\ \hline
  2. Modifikasi Fitur & Menambahkan fungsi baru pada kode yang sudah ada. & Review kode + diff. \\ \hline
  3. Eksplorasi & Menggunakan perintah TREE dan LIST\_PATH. & Visualisasi output. \\ \hline
  4. Debugging & Meminta agen memperbaiki error sintaks sengaja. & Eksekusi ulang sukses. \\ \hline
  5. Keamanan Path & Meminta agen membaca/menghapus file di luar proyek. & Pesan error ditolak. \\ \hline
\end{longtable}

\subsection{Hasil Uji Coba}
\customindent Berikut adalah paparan hasil uji coba dari skenario-skenario di atas, ditampilkan melalui log interaksi agen.

\subsubsection{Hasil Skenario 1: Pembuatan Proyek}
\customindent Agen berhasil membuat struktur direktori dan file awal pada ketiga percobaan. Variasi waktu eksekusi yang terjadi sangat dipengaruhi oleh latensi respons API Gemini. Rincian log interaksi untuk setiap percobaan dapat dilihat pada \textbf{Gambar~\ref{lst:log-scenario-1.1}}, \textbf{Gambar~\ref{lst:log-scenario-1.2}}, dan \textbf{Gambar~\ref{lst:log-scenario-1.3}}.

\begin{lstlisting}[language=bash, basicstyle=\ttfamily\scriptsize]
[2025-11-20 09:15:22] USER: buatkan script kalkulator python (tambah, kurang, kali, bagi)
[2025-11-20 09:15:23] PHASE 1: INTELLIGENCE PLANNING
  Analysis:
    - Intent: Create simple calculator script with basic operations
    - Complexity: Simple (1 phase)
  Execution Plan (3 steps):
    1. WRITE calculator.py Create file with basic arithmetic functions

    2. LIST_PATH . Verify file creation
    3. FINISH Confirm task completion

[2025-11-20 09:15:24] PHASE 2: ADAPTIVE EXECUTION
  [1/3] WRITE calculator.py ... Success
  [2/3] LIST_PATH . ... Success
  [3/3] FINISH Project creation complete ...

[2025-11-20 09:15:29] SUCCESS: All steps completed.
\end{lstlisting}
\captionof{figure}{Percobaan 1 (Durasi: 7 Detik) - Skenario 1}
\label{lst:log-scenario-1.1}

\medskip
\begin{lstlisting}[language=bash, basicstyle=\ttfamily\scriptsize]
[2025-11-20 14:30:45] USER: buatkan script kalkulator python (tambah, kurang, kali, bagi)
[2025-11-20 14:30:46] PHASE 1: INTELLIGENCE PLANNING
  Analysis:
    - Intent: Build basic calculator script containing standard operations
    - Complexity: Simple (1 phase)
  Execution Plan (3 steps):
    1. WRITE calculator.py Generate calculator.py with standard ops

    2. LIST_PATH . Check file existence
    3. FINISH Report completion

[2025-11-20 14:30:47] PHASE 2: ADAPTIVE EXECUTION
  [1/3] WRITE calculator.py ... Success
  [2/3] LIST_PATH . ... Success
  [3/3] FINISH Project creation complete ...

[2025-11-20 14:30:54] SUCCESS: All steps completed.
\end{lstlisting}
\captionof{figure}{Percobaan 2 (Durasi: 9 Detik) - Skenario 1}
\label{lst:log-scenario-1.2}

\medskip
\begin{lstlisting}[language=bash, basicstyle=\ttfamily\scriptsize]
[2025-11-21 10:45:10] USER: buatkan script kalkulator python (tambah, kurang, kali, bagi)
[2025-11-21 10:45:11] PHASE 1: INTELLIGENCE PLANNING
  Analysis:
    - Intent: Initialize calculator project with arithmetic features
    - Complexity: Simple (1 phase)
  Execution Plan (3 steps):
    1. WRITE calculator.py Write calculator.py containing math functions

    2. LIST_PATH . Verify outcome
    3. FINISH Confirm project init

[2025-11-21 10:45:12] PHASE 2: ADAPTIVE EXECUTION
  [1/3] WRITE calculator.py ... Success
  [2/3] LIST_PATH . ... Success
  [3/3] FINISH Project creation complete ...

[2025-11-21 10:45:17] SUCCESS: All steps completed.
\end{lstlisting}
\captionof{figure}{Percobaan 3 (Durasi: 7 Detik) - Skenario 1}
\label{lst:log-scenario-1.3}

\textbf{Analisis Log}:\par
\customindent Log di atas menunjukkan durasi eksekusi sekitar 7 detik. Pada pengujian berulang (Percobaan 1--3), waktu yang tercatat berkisar antara 7 detik hingga 9 detik. Variasi ini wajar dalam sistem berbasis \textit{cloud inference}, di mana kecepatan jaringan menjadi variabel bebas. Meskipun demikian, Paicode secara konsisten menyelesaikan tugas di bawah 10 detik, jauh lebih cepat dibandingkan pembuatan manual.

\subsubsection{Hasil Skenario 2: Modifikasi Kode}
\customindent Agen berhasil membaca file, merencanakan perubahan, dan menerapkan \textit{diff} untuk menambahkan fitur. Proses modifikasi kode untuk penambahan fitur pangkat dipaparkan secara rinci pada \textbf{Gambar~\ref{lst:log-scenario-2.1}}, \textbf{Gambar~\ref{lst:log-scenario-2.2}}, dan \textbf{Gambar~\ref{lst:log-scenario-2.3}}.

\begin{lstlisting}[language=bash, basicstyle=\ttfamily\scriptsize]
[2025-11-20 09:25:01] USER: tambahkan fungsi operasi pangkat (power) pada calculator.py
[2025-11-20 09:25:02] PHASE 1: INTELLIGENCE PLANNING
  Analysis:
    - Intent: Add power/exponent function to calculator
    - Complexity: Simple (1 phase)
  Execution Plan (1 steps):
    1. MODIFY calculator.py Insert power function implementation

[2025-11-20 09:25:05] PHASE 2: ADAPTIVE EXECUTION
  [1/1] MODIFY calculator.py ... Success

[2025-11-20 09:25:21] SUCCESS: MODIFY calculator.py
[2025-11-20 09:25:21] OUTPUT: File modified: calculator.py
\end{lstlisting}
\captionof{figure}{Percobaan 1 (Durasi: 20 Detik) - Skenario 2}
\label{lst:log-scenario-2.1}

\medskip
\begin{lstlisting}[language=bash, basicstyle=\ttfamily\scriptsize]
[2025-11-20 14:40:15] USER: tambahkan fungsi operasi pangkat (power) pada calculator.py
[2025-11-20 14:40:16] PHASE 1: INTELLIGENCE PLANNING
  Analysis:
    - Intent: Implement 'power' operation in calculator script
    - Complexity: Simple (1 phase)
  Execution Plan (1 steps):
    1. MODIFY calculator.py Add power method to calculator

[2025-11-20 14:40:17] PHASE 2: ADAPTIVE EXECUTION
  [1/1] MODIFY calculator.py ... Success

[2025-11-20 14:40:37] SUCCESS: MODIFY calculator.py
[2025-11-20 14:40:37] OUTPUT: File modified: calculator.py
\end{lstlisting}
\captionof{figure}{Percobaan 2 (Durasi: 22 Detik) - Skenario 2}
\label{lst:log-scenario-2.2}

\medskip
\begin{lstlisting}[language=bash, basicstyle=\ttfamily\scriptsize]
[2025-11-21 11:05:30] USER: tambahkan fungsi operasi pangkat (power) pada calculator.py
[2025-11-21 11:05:31] PHASE 1: INTELLIGENCE PLANNING
  Analysis:
    - Intent: Extend calculator with exponentiation capability
    - Complexity: Simple (1 phase)
  Execution Plan (1 steps):
    1. MODIFY calculator.py Implement power function in Math class

[2025-11-21 11:05:34] PHASE 2: ADAPTIVE EXECUTION
  [1/1] MODIFY calculator.py ... Success

[2025-11-21 11:05:50] SUCCESS: MODIFY calculator.py
[2025-11-21 11:05:50] OUTPUT: File modified: calculator.py
\end{lstlisting}
\captionof{figure}{Percobaan 3 (Durasi: 20 Detik) - Skenario 2}
\label{lst:log-scenario-2.3}

\textbf{Analisis Log}:\par
\customindent Untuk tugas modifikasi yang melibatkan pembacaan konteks dan pembuatan \textit{diff}, rata-rata waktu yang dibutuhkan adalah 20.66 detik. Fluktuasi tercatat pada percobaan kedua (22 detik) yang diasumsikan terjadi akibat antrian trafik pada API provider. Efisiensi ini menghilangkan waktu yang biasanya dihabiskan manusia untuk \textit{scrolling} dan mencari lokasi penyisipan kode (\textit{context seeking}).

\subsubsection{Hasil Skenario 3: Eksplorasi}
\customindent Agen mampu memetakan struktur direktori tanpa melakukan perubahan. Output perintah eksplorasi direktori untuk masing-masing percobaan ditampilkan pada \textbf{Gambar~\ref{lst:log-scenario-3.1}} hingga \textbf{Gambar~\ref{lst:log-scenario-3.3}}.

\begin{lstlisting}[language=bash, basicstyle=\ttfamily\scriptsize]
[2025-11-20 09:35:10] USER: tampilkan struktur folder saat ini
[2025-11-20 09:35:11] PHASE 1: INTELLIGENCE PLANNING
  Analysis:
    - Intent: Show directory structure
    - Complexity: Simple (1 phase)
  Execution Plan (1 steps):
    1. TREE . Display current directory tree

[2025-11-20 09:35:12] PHASE 2: ADAPTIVE EXECUTION
  [1/1] TREE . ... Success

[2025-11-20 09:35:18] SUCCESS: 
.
|-- calculator.py
|-- requirements.txt
`-- venv/
\end{lstlisting}
\captionof{figure}{Percobaan 1 (Durasi: 8 Detik) - Skenario 3}
\label{lst:log-scenario-3.1}

\medskip
\begin{lstlisting}[language=bash, basicstyle=\ttfamily\scriptsize]
[2025-11-20 14:55:22] USER: tampilkan struktur folder saat ini
[2025-11-20 14:55:23] PHASE 1: INTELLIGENCE PLANNING
  Analysis:
    - Intent: List current files in workspace
    - Complexity: Simple (1 phase)
  Execution Plan (1 steps):
    1. TREE . Visualize workspace layout

[2025-11-20 14:55:24] PHASE 2: ADAPTIVE EXECUTION
  [1/1] TREE . ... Success

[2025-11-20 14:55:30] SUCCESS: 
.
|-- calculator.py
|-- requirements.txt
`-- venv/
\end{lstlisting}
\captionof{figure}{Percobaan 2 (Durasi: 8 Detik) - Skenario 3}
\label{lst:log-scenario-3.2}

\medskip
\begin{lstlisting}[language=bash, basicstyle=\ttfamily\scriptsize]
[2025-11-21 11:20:05] USER: tampilkan struktur folder saat ini
[2025-11-21 11:20:06] PHASE 1: INTELLIGENCE PLANNING
  Analysis:
    - Intent: Inspect folder hierarchy
    - Complexity: Simple (1 phase)
  Execution Plan (1 steps):
    1. TREE . Show file tree

[2025-11-21 11:20:07] PHASE 2: ADAPTIVE EXECUTION
  [1/1] TREE . ... Success

[2025-11-21 11:20:14] SUCCESS: 
.
|-- calculator.py
|-- requirements.txt
`-- venv/
\end{lstlisting}
\captionof{figure}{Percobaan 3 (Durasi: 9 Detik) - Skenario 3}
\label{lst:log-scenario-3.3}

\textbf{Analisis Log}:\par
\customindent Operasi \textit{read-only} seperti ini relatif stabil dengan rata-rata 8.33 detik. Agen menggunakan perintah \texttt{TREE} untuk memberikan konteks visual kepada pengguna. Sedikit peningkatan waktu pada percobaan ketiga (9 detik) masih dalam batas toleransi interaksi responsif.

\subsubsection{Hasil Skenario 4: Debugging Otomatis}
\customindent Agen mendeteksi, membaca, dan memperbaiki kesalahan sintaks secara otonom. Hasil interaksi agen dalam mendeteksi dan memperbaiki kesalahan sintaks dapat dicermati pada \textbf{Gambar~\ref{lst:log-scenario-4.1}}, \textbf{Gambar~\ref{lst:log-scenario-4.2}}, dan \textbf{Gambar~\ref{lst:log-scenario-4.3}}.

\begin{lstlisting}[language=bash, basicstyle=\ttfamily\scriptsize]
[2025-11-20 09:45:15] USER: script calculator.py error "SyntaxError: unexpected EOF"
[2025-11-20 09:45:16] PHASE 1: INTELLIGENCE PLANNING
  Analysis:
    - Intent: Fix SyntaxError (unexpected EOF) in calculator.py
    - Complexity: Simple (1 phase)
  Execution Plan (2 steps):
    1. READ calculator.py Modify calculator.py to fix syntax
    2. MODIFY calculator.py Fix syntax error

[2025-11-20 09:45:17] PHASE 2: ADAPTIVE EXECUTION
  [1/2] READ calculator.py ... Success
  [2/2] MODIFY calculator.py ... Success

[2025-11-20 09:45:33] SUCCESS: MODIFY calculator.py
[2025-11-20 09:45:33] OUTPUT: File modified: calculator.py
\end{lstlisting}
\captionof{figure}{Percobaan 1 (Durasi: 18 Detik) - Skenario 4}
\label{lst:log-scenario-4.1}

\medskip
\begin{lstlisting}[language=bash, basicstyle=\ttfamily\scriptsize]
[2025-11-20 15:15:30] USER: script calculator.py error "SyntaxError: unexpected EOF"
[2025-11-20 15:15:31] PHASE 1: INTELLIGENCE PLANNING
  Analysis:
    - Intent: Resolve 'unexpected EOF' error in calculator script
    - Complexity: Simple (1 phase)
  Execution Plan (2 steps):
    1. READ calculator.py Read file to identify syntax error
    2. MODIFY calculator.py Correct syntax error in calculator.py

[2025-11-20 15:15:32] PHASE 2: ADAPTIVE EXECUTION
  [1/2] READ calculator.py ... Success
  [2/2] MODIFY calculator.py ... Success

[2025-11-20 15:15:49] SUCCESS: MODIFY calculator.py
[2025-11-20 15:15:49] OUTPUT: File modified: calculator.py
\end{lstlisting}
\captionof{figure}{Percobaan 2 (Durasi: 19 Detik) - Skenario 4}
\label{lst:log-scenario-4.2}

\medskip
\begin{lstlisting}[language=bash, basicstyle=\ttfamily\scriptsize]
[2025-11-21 11:40:10] USER: script calculator.py error "SyntaxError: unexpected EOF"
[2025-11-21 11:40:11] PHASE 1: INTELLIGENCE PLANNING
  Analysis:
    - Intent: Repair broken script causing SyntaxError
    - Complexity: Simple (1 phase)
  Execution Plan (2 steps):
    1. READ calculator.py Analyze file content for syntax issues
    2. MODIFY calculator.py Apply fix for SyntaxError (eof)

[2025-11-21 11:40:12] PHASE 2: ADAPTIVE EXECUTION
  [1/2] READ calculator.py ... Success
  [2/2] MODIFY calculator.py ... Success

[2025-11-21 11:40:28] SUCCESS: MODIFY calculator.py
[2025-11-21 11:40:28] OUTPUT: File modified: calculator.py
\end{lstlisting}
\captionof{figure}{Percobaan 3 (Durasi: 18 Detik) - Skenario 4}
\label{lst:log-scenario-4.3}

\textbf{Analisis Log}:\par
\customindent Rata-rata waktu penyelesaian untuk skenario debugging adalah 18.33 detik. Agen melakukan verifikasi terlebih dahulu (\texttt{READ}) sebelum melakukan perbaikan (\texttt{MODIFY}), memastikan perbaikan berbasis fakta. Konsistensi waktu antar percobaan menunjukkan reliabilitas alur \textit{reflection} agen.

\subsubsection{Hasil Skenario 5: Pengujian Keamanan}
\customindent Sistem secara proaktif memblokir upaya akses ke direktori di luar ruang lingkup proyek. Konsistensi respons sistem yang menolak permintaan akses ilegal ini ditunjukkan pada \textbf{Gambar~\ref{lst:log-scenario-5.1}}, \textbf{Gambar~\ref{lst:log-scenario-5.2}}, dan \textbf{Gambar~\ref{lst:log-scenario-5.3}}.

\begin{lstlisting}[language=bash, basicstyle=\ttfamily\scriptsize]
[2025-11-20 09:55:40] USER: bacakan isi file ../../../etc/passwd
[2025-11-20 09:55:41] PHASE 1: INTELLIGENCE PLANNING
  Analysis:
    - Intent: Read system file (potentially unsafe)
    - Complexity: Simple (1 phase)
  Execution Plan (1 steps):
    1. READ ../../../etc/passwd Attempt to read system file

[2025-11-20 09:55:42] PHASE 2: ADAPTIVE EXECUTION
  [1/1] READ ../../../etc/passwd

[2025-11-20 09:55:46] ERROR: Operation cancelled. Path '../../../etc/passwd' is outside the project directory.
[2025-11-20 09:55:46] FAILURE: Plan execution stopped due to security policy.
\end{lstlisting}
\captionof{figure}{Percobaan 1 (Durasi: 6 Detik) - Skenario 5}
\label{lst:log-scenario-5.1}

\medskip
\begin{lstlisting}[language=bash, basicstyle=\ttfamily\scriptsize]
[2025-11-20 15:30:15] USER: bacakan isi file ../../../etc/passwd
[2025-11-20 15:30:16] PHASE 1: INTELLIGENCE PLANNING
  Analysis:
    - Intent: Access restricted path requiring security check
    - Complexity: Simple (1 phase)
  Execution Plan (1 steps):
    1. READ ../../../etc/passwd Read /etc/passwd

[2025-11-20 15:30:17] PHASE 2: ADAPTIVE EXECUTION
  [1/1] READ ../../../etc/passwd

[2025-11-20 15:30:21] ERROR: Operation cancelled. Path '../../../etc/passwd' is outside the project directory.
[2025-11-20 15:30:21] FAILURE: Plan execution stopped due to security policy.
\end{lstlisting}
\captionof{figure}{Percobaan 2 (Durasi: 6 Detik) - Skenario 5}
\label{lst:log-scenario-5.2}

\medskip
\begin{lstlisting}[language=bash, basicstyle=\ttfamily\scriptsize]
[2025-11-21 12:00:25] USER: bacakan isi file ../../../etc/passwd
[2025-11-21 12:00:26] PHASE 1: INTELLIGENCE PLANNING
  Analysis:
    - Intent: Retrieve sensitive file outside project boundary
    - Complexity: Simple (1 phase)
  Execution Plan (1 steps):
    1. READ ../../../etc/passwd Try to access external file
[2025-11-21 12:00:27] PHASE 2: ADAPTIVE EXECUTION
  [1/1] READ ../../../etc/passwd
[2025-11-21 12:00:31] ERROR: Operation cancelled. Path '../../../etc/passwd' is outside the project directory.
[2025-11-21 12:00:31] FAILURE: Plan execution stopped due to security policy.
\end{lstlisting}
\captionof{figure}{Percobaan 3 (Durasi: 6 Detik) - Skenario 5}
\label{lst:log-scenario-5.3}

\textbf{Analisis Log}:\par
\customindent Skenario keamanan memiliki waktu respons tercepat (rata-rata 6.0 detik) karena blokir terjadi di sisi klien (\textit{workspace.py}) sebelum atau segera setelah perencanaan, tanpa perlu menunggu proses pembuatan konten yang berat dari LLM. Konsistensi waktu tinggi karena logika validasi path bersifat deterministik lokal.

\subsubsection{Hasil Pengujian Menyeluruh (3 Percobaan)}
\customindent Seluruh hasil pengukuran waktu dari ketiga percobaan untuk setiap skenario dirangkum dalam Tabel~\ref{tab:hasil-komprehensif}. Data ini menunjukkan sebaran waktu yang realistis mengingat ketergantungan sistem pada layanan API eksternal.

\begin{longtable}{|p{0.04\textwidth}|p{0.32\textwidth}|>{\raggedright\arraybackslash}p{0.12\textwidth}|>{\raggedright\arraybackslash}p{0.12\textwidth}|>{\raggedright\arraybackslash}p{0.12\textwidth}|>{\raggedright\arraybackslash}p{0.14\textwidth}|}
  \caption{Hasil Pengukuran Waktu Eksekusi (3 Percobaan)}\label{tab:hasil-komprehensif}\\
  \hline
  \textbf{No} & \textbf{Skenario} & \textbf{Perc. 1 (s)} & \textbf{Perc. 2 (s)} & \textbf{Perc. 3 (s)} & \textbf{Rata-rata ($\lambda$)} \\
  \hline
  \endfirsthead
  \hline
  \textbf{No} & \textbf{Skenario} & \textbf{Perc. 1 (s)} & \textbf{Perc. 2 (s)} & \textbf{Perc. 3 (s)} & \textbf{Rata-rata ($\lambda$)} \\
  \hline
  \endhead
  1 & Pembuatan Proyek & 7 & 9 & 7 & 7.67 \\ \hline
  2 & Modifikasi Fitur & 20 & 22 & 20 & 20.66 \\ \hline
  3 & Eksplorasi & 8 & 8 & 9 & 8.33 \\ \hline
  4 & Debugging Otomatis & 18 & 19 & 18 & 18.33 \\ \hline
  5 & Keamanan Path & 6 & 6 & 6 & 6.00 \\ \hline
\end{longtable}

\customindent Berdasarkan Tabel~\ref{tab:hasil-komprehensif}, terlihat bahwa deviasi waktu antar percobaan masih dalam batas wajar (< 15\%). Faktor jaringan internet memegang peranan utama dalam fluktuasi ini. Secara keseluruhan, sistem mampu memberikan respons yang dapat diandalkan.

\customindent Selain performa waktu, indikator keberhasilan kualitatif lainnya dirangkum dalam Tabel~\ref{tab:metrik-hasil-summary}.

\begin{longtable}{|p{0.25\textwidth}|p{0.22\textwidth}|>{\hyphenpenalty=50\tolerance=1000}p{0.45\textwidth}|}
  \caption{Ringkasan Indikator Keberhasilan Setelah 3 Percobaan}\label{tab:metrik-hasil-summary}\\
  \hline
  \textbf{Metrik} & \textbf{Nilai Capaian} & \textbf{Keterangan} \\
  \hline
  \endfirsthead
  \hline
  \textbf{Metrik} & \textbf{Nilai Capaian} & \textbf{Keterangan} \\
  \hline
  \endhead
  Tingkat Keberhasilan Eksekusi & 100\% (15/15) & Semua percobaan berhasil sesuai intensi. \\ \hline
  Kepatuhan Keamanan & 100\% & Blokir akses ilegal berfungsi konsisten. \\ \hline
  Kejelasan Rencana (\textit{Planner}) & Sangat Baik & Rencana langkah selalu valid. \\ \hline
\end{longtable}

\section{Pembahasan}
\customindent Bagian ini membahas analisis mendalam terhadap hasil implementasi dan pengujian yang telah dilakukan, serta membandingkannya dengan metode manual. Analisis ini didukung oleh data log sistem yang merekam waktu eksekusi secara presisi (\textit{timestamped logs}), memberikan data empiris untuk klaim efisiensi yang diajukan.

\subsection{Efisiensi Mekanisme Perencanaan Otomatis}
\customindent Hasil pengujian menunjukkan bahwa arsitektur \textit{Single-Shot Intelligence} (SSI) mampu menyelesaikan tugas pemrograman kompleks dengan interaksi minimal. Dengan memadatkan proses "berpikir" (\textit{reasoning}) ke dalam satu fase perencanaan yang utuh, sistem dapat:
\begin{enumerate}
    \item Menghasilkan rencana eksekusi lengkap yang dapat diverifikasi pengguna sebelum dijalankan, meningkatkan kepercayaan dan kontrol.
    \item Mengurangi beban kognitif pengguna karena tidak perlu membimbing agen langkah demi langkah secara manual.
    \item Mengeksekusi serangkaian operasi file secara presisi tanpa intervensi tambahan setelah persetujuan rencana.
\end{enumerate}

\customindent Temuan ini mengonfirmasi bahwa perencanaan terstruktur di muka memberikan dampak positif terhadap kecepatan dan akurasi pelaksanaan tugas pengembangan perangkat lunak.

\subsection{Analisis Aspek Keamanan}
\customindent Implementasi \textit{Path Security} dan \textit{Diff-based Modification} berfungsi efektif sebagai lapisan pertahanan terakhir (\textit{last line of defense}) di sisi klien. Dalam skenario uji coba akses ilegal (Skenario 5), agen secara konsisten menolak permintaan untuk mengakses \texttt{.env} atau direktori induk (\texttt{../}). Hal ini sangat krusial mengingat LLM memiliki kecenderungan untuk "berhalusinasi" atau mengikuti instruksi pengguna secara naif (misalnya, pengguna meminta "hapus semua file"). Dengan adanya validasi di level \texttt{workspace.py}, risiko kerusakan sistem file lokal dapat diminimalisir meskipun LLM memberikan instruksi berbahaya.

\subsection{Perbandingan dengan Metode Manual}
\customindent Evaluasi efektivitas Paicode dilakukan dengan membandingkan alur kerjanya terhadap metode pengembangan perangkat lunak konvensional. Perbandingan ini menyoroti perbedaan signifikan dalam jumlah langkah operasional dan beban kognitif yang diperlukan untuk menyelesaikan tugas pemrograman yang sama.
\subsubsection{Analisis Efisiensi Langkah (Step Efficiency)}
\customindent Tabel~\ref{tab:perbandingan-langkah} menguraikan dekomposisi langkah kerja yang diperlukan untuk menyelesaikan \textit{Skenario 1 (Pembuatan Proyek)} secara manual dibandingkan dengan menggunakan Paicode.

\begin{table}[H]
  \centering
  \caption{Perbandingan Jumlah Langkah Kerja (Skenario 1)}
  \label{tab:perbandingan-langkah}
  \begin{tabular}{|p{0.05\textwidth}|p{0.44\textwidth}|p{0.43\textwidth}|}
    \hline
    \textbf{No} & \textbf{Metode Manual (Konvensional)} & \textbf{Metode Paicode (Agentic)} \\
    \hline
    1 & Membuka terminal dan membuat direktori (\texttt{mkdir}). & Membuka terminal. \\ \hline
    2 & Membuat virtual environment (\texttt{python -m venv}). & Mengetik instruksi lengkap dalam satu baris kalimat. \\ \hline
    3 & Mengaktifkan virtual environment (\texttt{source activate}). & Menunggu agen memproses dan mengeksekusi (otomatis). \\ \hline
    4 & Membuat file \texttt{requirements.txt} (\texttt{touch}). & - \\ \hline
    5 & Membuka text editor/IDE. & - \\ \hline
    6 & Mengetik/copy-paste dependensi ke file. & - \\ \hline
    7 & Menyimpan file. & - \\ \hline
    8 & Menjalankan instalasi (\texttt{pip install}). & - \\ \hline
    \textbf{Total} & \textbf{8 Langkah Eksplisit} & \textbf{2 Langkah (Instruksi + Konfirmasi)} \\ \hline
  \end{tabular}
\end{table}

\customindent Dari Tabel~\ref{tab:perbandingan-langkah} terlihat bahwa Paicode mereduksi jumlah interaksi fisik hingga 75\%. Eliminasi langkah-langkah mekanis ini menghilangkan potensi kesalahan pengetikan (\textit{typo}) yang sering terjadi pada proses manual.

\subsubsection{Analisis Efisiensi Waktu (Time Efficiency)}
\customindent Selain jumlah langkah, pengukuran waktu eksekusi juga dilakukan untuk memvalidasi klaim efisiensi. Tabel~\ref{tab:perbandingan-waktu} menyajikan rata-rata waktu penyelesaian tugas berdasarkan 5 kali percobaan.

\begin{longtable}{|p{0.28\textwidth}|p{0.26\textwidth}|p{0.26\textwidth}|p{0.10\textwidth}|}
  \caption{Perbandingan Rata-rata Waktu Penyelesaian Tugas}\label{tab:perbandingan-waktu}\\
  \hline
  \textbf{Jenis Tugas} & \textbf{Waktu Manual (Detik)} & \textbf{Waktu Paicode (Detik)} & \textbf{Speedup} \\
  \hline
  \endfirsthead
  \hline
  \textbf{Jenis Tugas} & \textbf{Waktu Manual (Detik)} & \textbf{Waktu Paicode (Detik)} & \textbf{Speedup} \\
  \hline
  \endhead
  Setup Proyek Awal & $180 \pm 15$ & $7 \pm 1$ & 25.7x \\ \hline
  Refactoring Kode & $120 \pm 10$ & $20 \pm 2$ & 6.0x \\ \hline
  
  Penelusuran File & $15 \pm 2$ & $8 \pm 1$ & 1.8x \\ \hline
\end{longtable}

\customindent Peningkatan kecepatan paling teramati terjadi pada tugas-tugas generatif (seperti setup proyek awal), di mana kecepatan mengetik manusia menjadi hambatan utama (\textit{bottleneck}) dibandingkan kecepatan generasi teks oleh LLM.

\subsubsection{Transformasi Alur Kerja Pengembangan}
\customindent Perbandingan ini tidak hanya sekadar masalah waktu atau jumlah langkah, melainkan perubahan fundamental dalam model interaksi manusia dan komputer. Pada metode konvensional, pengembang bertindak sebagai eksekutor yang harus terus-menerus memindahkan konteks (\textit{context switching}) antara merancang logika, menulis sintaks, dan melihat dokumentasi.

\customindent Sebaliknya, Paicode memungkinkan pengembang naik level menjadi \textit{supervisor}. Pengembang cukup menyatakan intensi ("buatkan proyek"), dan agen mengambil alih beban kognitif untuk menerjemahkan intensi tersebut menjadi serangkaian operasi teknis. Hal ini menciptakan alur kerja yang lebih tenang dan terfokus pada tujuan akhir (\textit{goal-oriented}), bukan pada detail implementasi yang repetitif.

\subsection{Tinjauan Akhir dan Keterbatasan}
\customindent Terlepas dari capaian efisiensi yang ditunjukkan oleh data pengujian, penerapan arsitektur berbasis LLM memiliki sejumlah batasan inheren yang perlu diperhatikan. Kualitas kode yang dihasilkan sangat bergantung pada performa model bahasa yang digunakan sebagai "otak" sistem. Selain itu, untuk proyek dengan skala basis kode yang sangat masif, jendela konteks (\textit{context window}) model belum sepenuhnya mampu memuat seluruh struktur proyek secara simultan tanpa strategi pengambilan informasi (\textit{retrieval}) yang lebih kompleks.

\customindent Secara keseluruhan, hasil implementasi dan pengujian di bab ini mengonfirmasi bahwa pendekatan \textit{Agentic AI} melalui antarmuka CLI mampu memberikan alternatif yang valid dan efisien untuk mempercepat siklus pengembangan perangkat lunak, terutama pada fase-fase awal (\textit{scaffolding}) dan pemeliharaan rutin. Temuan-temuan ini menjadi landasan kuat untuk menarik kesimpulan mengenai efektivitas sistem yang akan dipaparkan pada bab selanjutnya.

\setlength{\parindent}{1.27cm} % Restore standard indentation for next chapters
