% Chapter 4 - Implementasi dan Hasil
\chapter{Implementasi dan Hasil}
\label{chap:implementasi}

\section{Implementasi Paicode}
Implementasi dilakukan menggunakan Python dengan manajemen dependensi Poetry. Berkas \texttt{pyproject.toml} mendefinisikan paket yang dibutuhkan beserta titik masuk CLI. Langkah instalasi dan konfigurasi sebagai berikut.

\subsection{Instalasi}
\begin{enumerate}
  \item Pastikan Python (\texttt{\textgreater= 3.9}) dan Poetry terpasang.
  \item Masuk ke direktori \texttt{paicode/} dan jalankan:
\end{enumerate}

\begin{lstlisting}[language=bash,caption={Instalasi dependensi dengan Poetry}]
poetry install
\end{lstlisting}

\subsection{Konfigurasi API Key}
Paicode memerlukan API key Gemini untuk akses LLM. Kunci disimpan secara aman pada \texttt{~/.config/pai-code/credentials} dengan izin berkas 600.

\begin{lstlisting}[language=bash,caption={Set dan verifikasi API key Gemini}]
poetry run pai config --set <API_KEY_GEMINI>
poetry run pai config --show
\end{lstlisting}

\subsection{Menjalankan Agen}
Sesi interaktif dapat dimulai langsung:

\begin{lstlisting}[language=bash,caption={Menjalankan sesi agen interaktif}]
poetry run pai
\end{lstlisting}

\section{Alur Interaksi}
Alur kerja pada sesi interaktif meliputi: (i) pengguna memberikan tujuan tingkat tinggi; (ii) agen mengobservasi struktur proyek menggunakan perintah \texttt{TREE}/\texttt{LIST\_PATH}; (iii) agen membaca/menulis/memodifikasi berkas; (iv) hasil dievaluasi dan menjadi konteks untuk langkah berikutnya. Kebijakan keamanan jalur mencegah akses ke direktori sensitif seperti \texttt{.git}, \texttt{venv}, dan \texttt{.env}.

\section{Rencana Gambar Implementasi}
Bagian ini merinci rencana gambar/screenshot yang akan ditambahkan untuk memperkuat penjelasan implementasi dan hasil.

% Figure 4.1: Tampilan awal sesi agen di terminal
\begin{figure}[htbp]
  \centering
  \fbox{\parbox{0.95\textwidth}{\centering Placeholder gambar: `img/fig4-1-sesi-awal-cli.png`\\
  (Screenshot terminal: pembukaan sesi agen, panel "Interactive Auto Mode")}}
  \caption{Tampilan awal sesi agen di terminal.}
  \label{fig:sesi-awal-cli}
\end{figure}

Pada Gambar~\ref{fig:sesi-awal-cli} diperlihatkan antarmuka awal sesi agen yang akan menjadi konteks interaksi.

% Figure 4.2: Perintah TREE menampilkan struktur direktori
\begin{figure}[htbp]
  \centering
  \fbox{\parbox{0.95\textwidth}{\centering Placeholder gambar: `img/fig4-2-tree-output.png`\\
  (Screenshot hasil perintah TREE pada proyek uji)}}
  \caption{Output perintah \texttt{TREE} untuk observasi struktur proyek.}
  \label{fig:tree-output}
\end{figure}

Pada Gambar~\ref{fig:tree-output} ditunjukkan hasil observasi struktur direktori yang digunakan agen sebagai dasar perencanaan aksi.

% Figure 4.3: Perintah LIST_PATH menampilkan daftar path
\begin{figure}[htbp]
  \centering
  \fbox{\parbox{0.95\textwidth}{\centering Placeholder gambar: `img/fig4-3-list-path.png`\\
  (Screenshot hasil perintah LIST_PATH dengan format baris per baris)}}
  \caption{Output perintah \texttt{LIST\_PATH} untuk daftar path mesin-baca.}
  \label{fig:list-path}
\end{figure}

% Figure 4.4: Panel pembacaan berkas (READ) dengan penyorotan sintaks
\begin{figure}[htbp]
  \centering
  \fbox{\parbox{0.95\textwidth}{\centering Placeholder gambar: `img/fig4-4-read-panel.png`\\
  (Panel kode dengan line number dan syntax highlighting saat READ)}}
  \caption{Panel pembacaan berkas dengan penyorotan sintaks.}
  \label{fig:read-panel}
\end{figure}

% Figure 4.5: Modifikasi terarah (MODIFY) beserta ringkasan diff
\begin{figure}[htbp]
  \centering
  \fbox{\parbox{0.95\textwidth}{\centering Placeholder gambar: `img/fig4-5-modify-diff.png`\\
  (Cuplikan hasil MODIFY yang menampilkan ringkasan diff/lines changed)}}
  \caption{Contoh hasil perintah \texttt{MODIFY} dengan batasan perubahan berbasis \textit{diff}.}
  \label{fig:modify-diff}
\end{figure}

% Figure 4.6: Diagram alur evaluasi dan metrik
\begin{figure}[htbp]
  \centering
  \fbox{\parbox{0.95\textwidth}{\centering Placeholder gambar: `img/fig4-6-evaluasi-metrik.png`\\
  (Diagram alur evaluasi: skenario → eksekusi → pencatatan metrik → analisis)}}
  \caption{Diagram alur evaluasi dan metrik yang dikumpulkan.}
  \label{fig:evaluasi-metrik}
\end{figure}

% Figure 4.7: Visualisasi hasil (grafik waktu/langkah)
\begin{figure}[htbp]
  \centering
  \fbox{\parbox{0.95\textwidth}{\centering Placeholder gambar: `img/fig4-7-grafik-hasil.png`\\
  (Grafik batang/garis: perbandingan waktu dan jumlah langkah antar skenario)}}
  \caption{Contoh visualisasi hasil awal untuk metrik efisiensi.}
  \label{fig:grafik-hasil}
\end{figure}
\section{Contoh Sesi}
Cuplikan berikut menggambarkan pembuatan proyek sederhana dan pembacaan isi berkas.

\begin{lstlisting}[language=bash,caption={Contoh interaksi singkat}]
$ pai
> buatkan proyek python sederhana: BMI Calculator
# Agen mengeksekusi: MKDIR, TOUCH, WRITE
> tampilkan struktur
# Agen mengeksekusi: TREE
> tampilkan isi sumber kode
# Agen mengeksekusi: READ
\end{lstlisting}

% Example of including code from the repository (adjust path if needed):
% \lstinputlisting[language=Python, caption={Cuplikan kode agent}, label={lst:agent}]{../paicode/paicode/agent.py}

\section{Evaluasi}
Evaluasi dilakukan melalui skenario tugas representatif yang mencakup pembuatan struktur proyek, penulisan berkas sumber, pembacaan, dan modifikasi terarah. Metrik yang diukur meliputi:

\begin{itemize}
  \item Waktu penyelesaian tugas.
  \item Jumlah langkah/komando yang diperlukan.
  \item Keberhasilan kompilasi/eksekusi kode hasil modifikasi.
  \item Kepatuhan terhadap kebijakan keamanan jalur (kegagalan akses jalur sensitif).
\end{itemize}

Hasil awal menunjukkan bahwa pendekatan agen \textit{stateful} dengan batasan perubahan berbasis \textit{diff} memudahkan penulisan dan pengembangan bertahap sambil menekan risiko penimpaan berkas yang tidak diinginkan. Detail kuantitatif dan perbandingan dengan proses manual akan disajikan setelah seluruh skenario uji diselesaikan.

% (reserved for future detailed session transcripts and extended results)
