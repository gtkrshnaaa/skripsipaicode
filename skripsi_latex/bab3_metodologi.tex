% Chapter 3 - Metodologi Penelitian
\chapter{Metodologi Penelitian}
\label{chap:metodologi}

\section{Metode Pengembangan}
Penelitian ini menggunakan pendekatan \textit{Research and Development} (R\&D) dengan strategi \textit{prototyping} iteratif. Pendekatan tersebut dipilih karena kebutuhan eksplorasi desain agen AI yang bersifat \textit{stateful} dan interaktif, sehingga memerlukan siklus cepat: perancangan, implementasi, uji coba, dan perbaikan. Setiap iterasi menghasilkan artefak yang dapat diuji untuk memvalidasi asumsi dan menyempurnakan rancangan.

\section{Arsitektur Sistem}
Arsitektur Paicode dirancang modular dan berlapis, dengan pembagian tanggung jawab yang jelas:

\begin{itemize}
  \item \textbf{Antarmuka CLI (\texttt{cli.py})}: titik masuk perintah \texttt{pai} dan pengelola argumen (subperintah \texttt{auto}, \texttt{config}). Mendukung parameter \texttt{--model} dan \texttt{--temperature} untuk konfigurasi runtime LLM. Secara default, CLI memanggil sesi interaktif agen.
  \item \textbf{Agen (\texttt{agent.py})}: mengimplementasikan \textit{8-step workflow} yang mencakup: (1) klasifikasi intensi (\textit{chat} vs \textit{task}), (2) respon awal agen, (3) penjadwalan tugas berbasis JSON, (4-6) iterasi aksi dengan \textit{thinking phase} dan \textit{integrity check}, serta (7) ringkasan akhir. Menyediakan perintah: \texttt{MKDIR}, \texttt{TOUCH}, \texttt{READ}, \texttt{WRITE}, \texttt{MODIFY}, \texttt{RM}, \texttt{MV}, \texttt{TREE}, \texttt{LIST\_PATH}, \texttt{FINISH}. Mengelola memori percakapan dengan kompresi konteks dan pencatatan sesi ke \texttt{.pai\_history}.
  \item \textbf{Jembatan LLM (\texttt{llm.py})}: menangani konfigurasi API Gemini dengan dukungan multi-API key dan \textit{round-robin load balancing}. Membersihkan output dari markdown artifacts dan menyediakan status spinner saat LLM berpikir.
  \item \textbf{Manajemen Konfigurasi (\texttt{config.py})}: menyimpan dan mengelola beberapa API key dalam format JSON di \texttt{~/.config/pai-code/credentials} dengan izin berkas 600. Mendukung operasi: \texttt{add}, \texttt{list}, \texttt{show}, \texttt{remove}, \texttt{set-default}, dan \textit{round-robin} untuk load balancing.
  \item \textbf{Pengatur Workspace (\texttt{workspace.py})}: bertindak sebagai \emph{workspace controller} yang menyediakan fungsi-fungsi terpusat untuk menjalankan operasi tingkat-aplikasi pada ruang kerja proyek. Sebelum aksi dieksekusi, modul ini menegakkan kebijakan \textit{path security} (normalisasi, verifikasi akar, dan deny-list direktori sensitif). Fungsi \texttt{apply\_modification\_with\_patch} membatasi perubahan maksimal 500 baris (dapat dikonfigurasi via \texttt{PAI\_MODIFY\_THRESHOLD}) dengan atomic write menggunakan tempfile.
  \item \textbf{Tampilan Terminal (\texttt{ui.py})}: penyajian hasil eksekusi menggunakan \texttt{rich} (panel, warna, tabel, penyorotan sintaks, spinner status). Mendukung \texttt{prompt\_toolkit} (opsional) untuk input yang lebih baik.
\end{itemize}

Alur data tipikal dengan \textit{8-step workflow}: masukan pengguna (CLI) → klasifikasi intensi → konstruksi prompt (Agen) → panggilan LLM → penjadwalan tugas JSON → iterasi aksi (thinking + eksekusi + integrity check) → ringkasan akhir → pencatatan konteks sebagai memori percakapan.

\section{Visualisasi Metodologi}
Bagian ini menyajikan visualisasi konsep menggunakan tabel dan daftar terstruktur berbasis LaTeX.

% Tabel 3.1: Diagram modul dan dependensi
\begin{longtable}{@{}p{0.25\textwidth}p{0.70\textwidth}@{}}
  \caption{Modul dan Dependensi Komponen Paicode}\label{tab:diagram-modul}\\
  \toprule
  \textbf{Komponen} & \textbf{Deskripsi dan Dependensi Utama} \\
  \midrule
  \endfirsthead
  \toprule
  \textbf{Komponen} & \textbf{Deskripsi dan Dependensi Utama} \\
  \midrule
  \endhead
  CLI (\texttt{cli.py}) & Titik masuk perintah, parsing argumen (\texttt{--model}, \texttt{--temperature}); memanggil sesi agen. Bergantung pada modul \texttt{agent}, \texttt{config}, dan \texttt{llm}. \\
  Agen (\texttt{agent.py}) & Implementasi \textit{8-step workflow}: klasifikasi intensi, penjadwalan tugas JSON, iterasi aksi dengan \textit{thinking phase} dan \textit{integrity check}, ringkasan akhir. Mengelola memori percakapan dengan kompresi konteks, \textit{interrupt handling} (Ctrl+C), \textit{auto-continue}, dan pencatatan sesi ke \texttt{.pai\_history}. Memanggil \texttt{llm}, \texttt{workspace}, \texttt{ui}. \\
  LLM Bridge (\texttt{llm.py}) & Integrasi Gemini API (\texttt{google-generativeai}) dengan multi-API key dan \textit{round-robin load balancing}. Membersihkan markdown artifacts dari output LLM. Mengambil API key dari \texttt{config}. \\
  Konfigurasi (\texttt{config.py}) & Manajemen multi-API key dalam format JSON di \texttt{~/.config/pai-code/credentials}. Operasi: \texttt{add}, \texttt{list}, \texttt{show}, \texttt{remove}, \texttt{set-default}, \texttt{next\_api\_key} (round-robin). \\
  Pengatur Workspace (\texttt{workspace.py}) & \emph{Workspace controller} dengan fungsi operasi workspace (baca/tulis, buat/hapus/pindah, tree/list path). Fungsi \texttt{apply\_modification\_with\_patch} membatasi perubahan maksimal 500 baris dengan atomic write. Penegakan \textit{path security} (normalisasi, verifikasi akar, deny-list). \\
  Terminal UI (\texttt{ui.py}) & Komponen TUI berbasis \texttt{rich}: panel, tema, syntax highlighting, tabel, spinner. Dukungan opsional \texttt{prompt\_toolkit} untuk input yang lebih baik. \\
  \bottomrule
\end{longtable}

Pada Tabel~\ref{tab:diagram-modul} ditunjukkan komponen utama dan interkoneksinya, sebagai acuan implementasi.

% Tabel 3.2: Urutan interaksi (sequence) sesi agen dengan 8-step workflow
\begin{longtable}{@{}p{0.06\textwidth}p{0.18\textwidth}p{0.70\textwidth}@{}}
  \caption{Urutan Interaksi Sesi Agen dengan 8-Step Workflow}\label{tab:sequence-session}\\
  \toprule
  \textbf{No} & \textbf{Pelaku} & \textbf{Aksi/Peristiwa} \\
  \midrule
  \endfirsthead
  \toprule
  \textbf{No} & \textbf{Pelaku} & \textbf{Aksi/Peristiwa} \\
  \midrule
  \endhead
  1 & Pengguna & Memberikan tujuan/permintaan tingkat tinggi di terminal. \\
  2 & CLI & Meneruskan masukan ke agen; menyiapkan konteks sesi. \\
  3 & Agen & Melakukan klasifikasi intensi (\textit{chat} vs \textit{task}) menggunakan LLM. Jika \textit{chat}, langsung berikan respons dan kembali ke langkah 1. \\
  4 & Agen & \textbf{Step 1}: Memberikan respon awal singkat (1-2 kalimat) untuk mengakui permintaan pengguna. \\
  5 & LLM & \textbf{Step 2}: Menghasilkan penjadwalan tugas berbasis JSON dengan 2-6 langkah logis. \\
  6 & Agen & Menampilkan rencana tugas dalam bentuk tabel terstruktur. \\
  7 & Agen & \textbf{Steps 3-7}: Untuk setiap iterasi aksi (maksimal 5 iterasi): \\
    & & \quad a) \textit{Thinking Phase}: LLM melakukan penalaran internal tentang pendekatan terbaik (3-6 poin). \\
    & & \quad b) \textit{Action Execution}: Mengeksekusi perintah (\texttt{READ}, \texttt{WRITE}, \texttt{MODIFY}, dll.) dengan batasan 500 baris per modifikasi. \\
    & & \quad c) \textit{Integrity Check}: LLM mengevaluasi hasil dengan skor kualitas 1-10, alasan, dan saran perbaikan. \\
  8 & Workspace/UI & Menjalankan operasi berkas (dengan \textit{path security} dan \textit{diff}-aware) dan menampilkan hasil di terminal. \\
  9 & Agen & \textbf{Step 8}: Memberikan ringkasan akhir, 2-3 saran langkah berikutnya, dan pertanyaan konfirmasi. \\
  10 & Agen & Mencatat seluruh interaksi ke \texttt{.pai\_history/session\_YYYYMMDD\_HHMMSS.log} sebagai memori (\emph{stateful}). \\
  11 & Pengguna & Memberikan instruksi lanjutan atau konfirmasi (\textit{auto-continue}); siklus berulang sampai \texttt{exit/quit}. \\
  \bottomrule
\end{longtable}

Pada Tabel~\ref{tab:sequence-session} divisualisasikan aliran pesan yang terjadi selama satu putaran iterasi agen.

% Tabel/Daftar 3.3: Alur kebijakan keamanan path
\paragraph{Alur Kebijakan Keamanan \textit{Path}.} Langkah-langkah validasi \textit{path} diringkas berikut:
\begin{enumerate}[label=\arabic*.]
  \item Normalisasi \textit{path} target (\texttt{os.path.normpath}).
  \item Resolusi \textit{real path} relatif terhadap akar proyek; pastikan tetap berada di dalam akar proyek.
  \item Pemeriksaan \emph{deny-list} direktori/berkas sensitif: \texttt{.env}, \texttt{.git}, \texttt{venv}, \texttt{\_\_pycache\_\_}, \texttt{.pai\_history}, \texttt{.idea}, \texttt{.vscode}.
  \item Jika salah satu pemeriksaan gagal: batalkan operasi dan tampilkan pesan kesalahan.
\end{enumerate}

\begin{longtable}{@{}p{0.20\textwidth}p{0.74\textwidth}@{}}
  \caption{Rangkuman Validasi Keamanan \textit{Path}}\label{tab:policy-keamanan}\\
  \toprule
  \textbf{Tahap} & \textbf{Detail Pemeriksaan} \\
  \midrule
  \endfirsthead
  \toprule
  \textbf{Tahap} & \textbf{Detail Pemeriksaan} \\
  \midrule
  \endhead
  Normalisasi & Gunakan fungsi normalisasi untuk menyingkirkan segmen berlebih (mis. \texttt{..}, duplikasi pemisah). \\
  Verifikasi Root & Gabungkan terhadap akar proyek, lakukan \texttt{realpath}, dan validasi prefiks tetap di dalam akar proyek. \\
  Deny-list & Tolak bila salah satu segmen \textit{path} termasuk daftar sensitif (.env, .git, venv, dll.). \\
  Penanganan Error & Batalkan operasi dan tampilkan pesan kesalahan yang informatif melalui TUI. \\
  \bottomrule
\end{longtable}

Pada Tabel~\ref{tab:policy-keamanan} diperlihatkan langkah-langkah validasi \textit{path} sebagai pengaman operasi berkas proyek.

\section{Alat dan Lingkungan}
Lingkungan dan alat yang digunakan:

\begin{itemize}
  \item Sistem operasi: Ubuntu (Linux).
  \item Bahasa pemrograman: Python (\texttt{\textgreater= 3.10}).
  \item Manajer dependensi/kemasan: Poetry atau setuptools; titik masuk CLI didefinisikan pada \texttt{setup.cfg} dengan entry point \texttt{pai = paicode.cli:main}.
  \item LLM: Google Gemini (model \texttt{gemini-2.5-flash}, dapat dikonfigurasi via \texttt{PAI\_MODEL}) melalui paket \texttt{google-generativeai} versi $\geq$ 0.5.4.
  \item TUI: \texttt{rich} (versi $\geq$ 13.7.1) untuk panel, warna, tabel, penyorotan sintaks, dan spinner status; \texttt{prompt\_toolkit} (opsional) untuk input yang lebih baik.
  \item Dependensi tambahan: \texttt{python-dotenv} ($\geq$ 1.0.1), \texttt{Pygments} ($\geq$ 2.16.0).
  \item Variabel lingkungan: \texttt{PAI\_MODEL}, \texttt{PAI\_TEMPERATURE}, \texttt{PAI\_MAX\_CMDS\_PER\_STEP}, \texttt{PAI\_MODIFY\_THRESHOLD}, \texttt{PAI\_MODIFY\_MAX\_RATIO}.
  \item LaTeX: TeX Live (\texttt{texlive-latex-recommended}, \texttt{texlive-latex-extra}, dsb.) dan Makefile untuk kompilasi naskah.
  \item Kendali versi: Git dan GitHub.
\end{itemize}

\section{Prosedur Penelitian}
Prosedur penelitian dan evaluasi dirancang sebagai berikut:

\begin{enumerate}
  \item \textbf{Perancangan}: mendefinisikan skenario penggunaan, himpunan perintah agen, dan kebijakan keamanan \textit{path}.
  \item \textbf{Implementasi}: membangun modul-modul inti (CLI, Agen, LLM, Workspace, UI) berikut mekanisme \textit{diff}-aware untuk pembatasan perubahan.
  \item \textbf{Eksperimen}: menjalankan serangkaian skenario pemrograman (mis. pembuatan struktur proyek, pembuatan/ pembacaan/ modifikasi berkas, refaktorisasi sederhana) dalam sesi interaktif.
  \item \textbf{Pengumpulan Data}: merekam waktu penyelesaian tugas, jumlah langkah perintah, tingkat keberhasilan eksekusi, dan catatan kesalahan.
  \item \textbf{Evaluasi}: membandingkan hasil dengan proses manual atau alat pembanding bila relevan, menggunakan metrik: (i) efisiensi (waktu dan langkah), (ii) ketepatan hasil (kompilasi/eksekusi kode), (iii) keamanan (kegagalan akses \textit{path} sensitif), dan (iv) pengalaman pengguna (keterbacaan output).
  \item \textbf{Analisis}: mengidentifikasi kelebihan, kekurangan, dan peluang peningkatan (mis. dukungan multi-LLM, integrasi editor, perluasan kebijakan keamanan).
\end{enumerate}
