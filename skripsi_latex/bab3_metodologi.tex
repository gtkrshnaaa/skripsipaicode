% Chapter 3 - Metodologi Penelitian
\chapter{Metodologi Penelitian}
\label{chap:metodologi}

\section{Metode Pengembangan}
Penelitian ini menggunakan pendekatan \textit{Research and Development} (R\&D) dengan strategi \textit{prototyping} iteratif. Pendekatan tersebut dipilih karena kebutuhan eksplorasi desain agen AI yang bersifat \textit{stateful} dan interaktif, sehingga memerlukan siklus cepat: perancangan, implementasi, uji coba, dan perbaikan. Setiap iterasi menghasilkan artefak yang dapat diuji untuk memvalidasi asumsi dan menyempurnakan rancangan.

\section{Arsitektur Sistem}
Arsitektur Paicode dirancang modular dan berlapis, dengan pembagian tanggung jawab yang jelas:

\begin{itemize}
  \item \textbf{Antarmuka CLI (\texttt{cli.py})}: titik masuk perintah \texttt{pai} dan pengelola argumen (subperintah \texttt{auto}, \texttt{config}). Secara default, CLI memanggil sesi interaktif agen.
  \item \textbf{Agen (\texttt{agent.py})}: menyusun prompt, mengelola memori percakapan, dan mengeksekusi rencana aksi hasil LLM. Menyediakan perintah: \texttt{MKDIR}, \texttt{TOUCH}, \texttt{READ}, \texttt{WRITE}, \texttt{MODIFY}, \texttt{RM}, \texttt{MV}, \texttt{TREE}, \texttt{LIST\_PATH}, \texttt{FINISH}.
  \item \textbf{Jembatan LLM (\texttt{llm.py})}: menangani konfigurasi API Gemini dan penyederhanaan hasil keluaran.
  \item \textbf{Gerbang Sistem Berkas (\texttt{fs.py})}: menyediakan operasi berkas dengan kebijakan keamanan (\textit{path security}: validasi \textit{path}, blokir direktori sensitif), serta mekanisme \textit{diff}-aware untuk \texttt{MODIFY}.
  \item \textbf{Tampilan Terminal (\texttt{ui.py})}: penyajian hasil eksekusi menggunakan \texttt{rich} (panel, warna, penomoran baris).
\end{itemize}

Alur data tipikal: masukan pengguna (CLI) → konstruksi prompt (Agen) → panggilan LLM → rencana aksi → eksekusi tindakan (FS/UI) → pelaporan dan pencatatan konteks sebagai memori percakapan.

\section{Rencana Gambar Metodologi}
Bagian ini memuat rencana gambar untuk memperjelas metode dan arsitektur pada Bab~\ref{chap:metodologi}. Placeholder akan diganti dengan gambar final sesuai hasil perancangan.

% Figure 3.1: Diagram modul dan dependensi
\begin{figure}[htbp]
  \centering
  \fbox{\parbox{0.95\textwidth}{\centering Placeholder gambar: `img/fig3-1-diagram-modul.png`\\
  (Diagram modul/komponen: CLI, Agent, LLM, FS, UI; hubungan dependensi)}}
  \caption{Diagram modul dan dependensi komponen Paicode.}
  \label{fig:diagram-modul}
\end{figure}

Pada Gambar~\ref{fig:diagram-modul} ditunjukkan komponen utama dan interkoneksinya, sebagai acuan implementasi.

% Figure 3.2: Urutan interaksi (sequence diagram) sesi agen
\begin{figure}[htbp]
  \centering
  \fbox{\parbox{0.95\textwidth}{\centering Placeholder gambar: `img/fig3-2-sequence-session.png`\\
  (Sequence diagram: user → CLI → Agent → LLM → FS/UI → kembali ke user)}}
  \caption{Urutan interaksi sesi agen dari masukan pengguna hingga hasil.}
  \label{fig:sequence-session}
\end{figure}

Pada Gambar~\ref{fig:sequence-session} divisualisasikan aliran pesan yang terjadi selama satu putaran iterasi agen.

% Figure 3.3: Alur kebijakan keamanan path
\begin{figure}[htbp]
  \centering
  \fbox{\parbox{0.95\textwidth}{\centering Placeholder gambar: `img/fig3-3-policy-keamanan.png`\\
  (Flowchart validasi path: normalisasi → verifikasi root → deny-list direktori sensitif)}}
  \caption{Alur kebijakan keamanan path (path security) pada berkas proyek.}
  \label{fig:policy-keamanan}
\end{figure}

Pada Gambar~\ref{fig:policy-keamanan} diperlihatkan langkah-langkah validasi \textit{path} sebagai pengaman operasi berkas proyek.

\section{Alat dan Lingkungan}
Lingkungan dan alat yang digunakan:

\begin{itemize}
  \item Sistem operasi: Ubuntu (Linux).
  \item Bahasa pemrograman: Python (\texttt{\textgreater= 3.9}).
  \item Manajer dependensi/kemasan: Poetry; titik masuk CLI didefinisikan pada \texttt{pyproject.toml}.
  \item LLM: Google Gemini melalui paket \texttt{google-generativeai}.
  \item TUI: \texttt{rich} untuk panel, warna, dan penyorotan sintaks.
  \item LaTeX: TeX Live (\texttt{texlive-latex-recommended}, \texttt{texlive-latex-extra}, dsb.) dan Makefile untuk kompilasi naskah.
  \item Kendali versi: Git dan GitHub.
\end{itemize}

\section{Prosedur Penelitian}
Prosedur penelitian dan evaluasi dirancang sebagai berikut:

\begin{enumerate}
  \item \textbf{Perancangan}: mendefinisikan skenario penggunaan, himpunan perintah agen, dan kebijakan keamanan \textit{path}.
  \item \textbf{Implementasi}: membangun modul-modul inti (CLI, Agen, LLM, FS, UI) berikut mekanisme \textit{diff}-aware untuk pembatasan perubahan.
  \item \textbf{Eksperimen}: menjalankan serangkaian skenario pemrograman (mis. pembuatan struktur proyek, pembuatan/ pembacaan/ modifikasi berkas, refaktorisasi sederhana) dalam sesi interaktif.
  \item \textbf{Pengumpulan Data}: merekam waktu penyelesaian tugas, jumlah langkah perintah, tingkat keberhasilan eksekusi, dan catatan kesalahan.
  \item \textbf{Evaluasi}: membandingkan hasil dengan proses manual atau alat pembanding bila relevan, menggunakan metrik: (i) efisiensi (waktu dan langkah), (ii) ketepatan hasil (kompilasi/eksekusi kode), (iii) keamanan (kegagalan akses \textit{path} sensitif), dan (iv) pengalaman pengguna (keterbacaan output).
  \item \textbf{Analisis}: mengidentifikasi kelebihan, kekurangan, dan peluang peningkatan (mis. dukungan multi-LLM, integrasi editor, perluasan kebijakan keamanan).
\end{enumerate}
