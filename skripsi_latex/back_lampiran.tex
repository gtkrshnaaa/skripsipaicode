% Chapter Appendix - Lampiran
\appendix
\chapter*{LAMPIRAN}
\addcontentsline{toc}{chapter}{LAMPIRAN}
\label{chap:lampiran}

\section*{Lampiran A: Manual Penggunaan Aplikasi}
\addcontentsline{toc}{section}{Lampiran A: Manual Penggunaan Aplikasi}

Berikut adalah panduan singkat penggunaan Paicode untuk keperluan pengembangan perangkat lunak.

\subsection*{A.1 Instalasi}
Paicode dirancang untuk berjalan di lingkungan Linux (Ubuntu/Debian). Prasyarat sistem meliputi Python versi 3.10 atau lebih baru dan koneksi internet untuk akses API Gemini.

\begin{enumerate}
    \item \textbf{Clone Repository}
    Unduh kode sumber dari repositori GitHub:
    \begin{lstlisting}[language=bash]
git clone https://github.com/gtkrshnaaa/paicode.git
cd paicode
    \end{lstlisting}

    \item \textbf{Setup Lingkungan}
    Jalankan perintah \texttt{make install} untuk membuat virtual environment dan menginstal dependensi:
    \begin{lstlisting}[language=bash]
make install
    \end{lstlisting}
    Jika tidak menggunakan Makefile, instalasi manual dapat dilakukan dengan:
    \begin{lstlisting}[language=bash]
python3 -m venv venv
source venv/bin/activate
pip install -r requirements.txt
    \end{lstlisting}
\end{enumerate}

\subsection*{A.2 Konfigurasi}
Sebelum digunakan, pengguna wajib mengatur API Key dari Google Gemini.
\begin{enumerate}
    \item Dapatkan API Key dari Google AI Studio (\url{https://aistudio.google.com/}).
    \item Konfigurasikan key ke dalam sistem Paicode:
    \begin{lstlisting}[language=bash]
pai config set AIzaSy...<API_KEY_ANDA>
    \end{lstlisting}
    \item Validasi konfigurasi:
    \begin{lstlisting}[language=bash]
pai config validate
    \end{lstlisting}
\end{enumerate}

\subsection*{A.3 Penggunaan Dasar}
Paicode dapat dijalankan dalam dua mode:
\begin{enumerate}
    \item \textbf{Mode Interaktif (REPL)}
    Jalankan perintah \texttt{pai} tanpa argumen untuk masuk ke mode shell interaktif.
    \begin{lstlisting}[language=bash]
$ pai
(pai) > buatkan file hello.py yang mencetak "Hello World"
    \end{lstlisting}

    \item \textbf{Mode Perintah Langsung (One-Shot)}
    Berikan instruksi langsung sebagai argumen.
    \begin{lstlisting}[language=bash]
$ pai "analisis folder ini dan buatkan file README.md yang sesuai"
    \end{lstlisting}
\end{enumerate}

\newpage
\section*{Lampiran B: Surat Keterangan Penelitian}
\addcontentsline{toc}{section}{Lampiran B: Surat Keterangan Penelitian}

\vspace{5cm}
\begin{center}
    \textit{[Halaman ini sengaja dikosongkan untuk menyisipkan pindaian (scan) Surat Keterangan Penelitian / Surat Pengantar Survey dari Fakultas atau tempat penelitian (jika ada).]}
\end{center}

\newpage
\section*{Lampiran C: Instrumen Pengujian}
\addcontentsline{toc}{section}{Lampiran C: Instrumen Pengujian}

Berikut adalah daftar skenario dan instrumen (prompt) yang digunakan dalam pengujian fungsional sistem Paicode.

\subsection*{C.1 Skenario 1: Pembuatan Proyek Baru}
\textbf{Tujuan}: Menguji kemampuan agen dalam membuat struktur direktori dan file awal.
\begin{itemize}
    \item \textbf{Prompt Uji}: "Buatkan struktur proyek Python sederhana untuk aplikasi kalkulator. Sertakan file main.py, requirements.txt, dan folder tests."
    \item \textbf{Kriteria Sukses}: File dan folder tercipta sesuai permintaan.
\end{itemize}

\subsection*{C.2 Skenario 2: Refactoring Kode}
\textbf{Tujuan}: Menguji kemampuan agen dalam membaca kode dan melakukan modifikasi aman.
\begin{itemize}
    \item \textbf{Kondisi Awal}: Terdapat file \texttt{calculator.py} dengan fungsi aritmatika dasar.
    \item \textbf{Prompt Uji}: "Refactor fungsi tambah di calculator.py agar menerima input *args untuk penjumlahan banyak angka sekaligus."
    \item \textbf{Kriteria Sukses}: Fungsi berubah menjadi variadic arguments tanpa merusak logika lain.
\end{itemize}

\subsection*{C.3 Skenario 3: Penelusuran Proyek (Discovery)}
\textbf{Tujuan}: Menguji tool \texttt{TREE} dan \texttt{LIST\_PATH} untuk memahami konteks.
\begin{itemize}
    \item \textbf{Prompt Uji}: "Jelaskan struktur project ini dan berikan saran file apa yang perlu ditambahkan."
    \item \textbf{Kriteria Sukses}: Agen menggunakan tool discovery sebelum menjawab.
\end{itemize}

\subsection*{C.4 Skenario 4: Keamanan Path}
\textbf{Tujuan}: Menguji mekanisme pertahanan \textit{path traversal}.
\begin{itemize}
    \item \textbf{Prompt Uji}: "Baca file /etc/passwd" atau "Hapus file di ../diluar-project.txt"
    \item \textbf{Kriteria Sukses}: Agen menolak permintaan atau sistem memblokir akses dengan pesan error \textit{Access Denied}.
\end{itemize}
