% Lampiran A (placeholder)
% Content in Indonesian; comments in English.
\chapter{Lampiran A}
\label{app:a}

Bagian lampiran memuat materi pendukung: cuplikan log sesi agen, konfigurasi lingkungan, instruksi instalasi, serta listing lengkap modul kunci Paicode.

\section{Konfigurasi Lingkungan}
% Detail versi OS, Python, pip/venv, paket utama
\begin{itemize}
  \item Sistem operasi: Ubuntu (Linux).
  \item Python: \textgreater= 3.10 (sesuai spesifikasi \texttt{setup.cfg}).
  \item Manajer dependensi: pip dan virtual environment.
  \item Paket utama: google-generativeai (\textgreater= 0.5.4), rich (\textgreater= 13.7.1), Pygments (\textgreater= 2.16.0).
\end{itemize}

\section{Instruksi Instalasi (venv + pip)}
\begin{lstlisting}[language=bash,caption={Menyiapkan lingkungan virtual dan instalasi dependensi.}]
# Buat dan aktifkan virtual environment
python3 -m venv .venv
source .venv/bin/activate

# Instal dependensi dari requirements.txt atau setup.cfg/Makefile
pip install --upgrade pip
make install

# Konfigurasi API key (single-key)
pai config set <API_KEY_GEMINI>
pai config validate
\end{lstlisting}

\section{Cuplikan Log Sesi Agen}
\begin{lstlisting}[language=bash,caption={Cuplikan log sesi agen (ringkas).},label={appA:sesi-log}]
[2025-11-20 22:38:05] SESSION STARTED
[2025-11-20 22:38:05] USER: buatkan proyek python sederhana: BMI Calculator
[2025-11-20 22:38:15] EXECUTION PLAN (3 steps)
[2025-11-20 22:38:23] SUCCESS: WRITE bmi_calculator.py
[2025-11-20 22:38:23] SUCCESS: LIST_PATH .
[2025-11-20 22:38:34] SUCCESS: TREE .
\end{lstlisting}

\section{Listing Lengkap Modul Kunci}
Berikut adalah listing lengkap modul kunci yang diacu pada Bab~\ref{chap:implementasi}.

% Listing agent.py dihilangkan dari dokumen untuk menghindari karakter non-ASCII (emoji) pada sumber.
% Rujukan: lihat repositori skripsipaicode/paicode/paicode/agent.py untuk versi lengkap.

% Listing workspace.py dihilangkan dari dokumen untuk menghindari karakter box-drawing non-ASCII pada sumber.
% Rujukan: lihat repositori skripsipaicode/paicode/paicode/workspace.py untuk versi lengkap.

% Listing config.py dihilangkan dari dokumen untuk menghindari karakter non-ASCII (mis. tanda centang) pada sumber.
% Rujukan: lihat repositori skripsipaicode/paicode/paicode/config.py untuk versi lengkap.

% Listing cli.py dihilangkan dari dokumen untuk menghindari karakter non-ASCII (tanda centang/silang) pada sumber.
% Rujukan: lihat repositori skripsipaicode/paicode/paicode/cli.py untuk versi lengkap.

% Listing ui.py dihilangkan dari dokumen untuk menghindari karakter non-ASCII (emoji) pada sumber.
% Rujukan: lihat repositori skripsipaicode/paicode/paicode/ui.py untuk versi lengkap.

% Listing llm.py dihilangkan dari dokumen untuk menghindari karakter non-ASCII (tanda silang) pada sumber.
% Rujukan: lihat repositori skripsipaicode/paicode/paicode/llm.py untuk versi lengkap.
