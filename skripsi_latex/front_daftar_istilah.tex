% Daftar Istilah (campuran Inggris-Indonesia sesuai kebutuhan teknis)
% Keep comments in English; content in Indonesian.
\begin{longtable}{p{0.25\textwidth}p{0.70\textwidth}}
\textbf{CLI} & Command Line Interface; antarmuka baris perintah pada terminal. \\
\textbf{LLM} & Large Language Model; model bahasa berskala besar untuk inferensi teks/kode. \\
\textbf{API} & Application Programming Interface; antarmuka pemrograman untuk mengakses layanan (mis. LLM). \\
\textbf{control/data flow} & Pola arus kontrol dan data antar komponen dalam arsitektur sistem yang menggambarkan urutan eksekusi dan pertukaran informasi. \\
\textbf{workspace controller} & Modul pengatur workspace yang memusatkan fungsi-fungsi operasi tingkat-aplikasi pada workspace proyek, termasuk validasi \textit{path}, pelarangan \textit{path} sensitif, dan modifikasi berbasis \textit{diff}. \\
\textbf{path} & Jalur berkas/direktori pada workspace proyek (contoh: \texttt{/home/user/project/main.py}). \\
\textbf{path security} & Kebijakan keamanan terkait path: normalisasi, validasi root, dan blokir direktori sensitif untuk mencegah akses yang tidak sah. \\
\textbf{path traversal} & Teknik atau upaya mengakses direktori/berkas di luar cakupan yang diizinkan dengan memanipulasi path (mis. menggunakan segmen \texttt{..}). \\
\textbf{deny-list} & Daftar path/pola yang dilarang untuk diakses atau dimodifikasi (mis. \texttt{.env}, \texttt{.git}, \texttt{venv/}, \texttt{\_\_pycache\_\_/}, \texttt{.vscode/}). \\
\textbf{project files (berkas proyek)} & Berkas-berkas aplikasi dalam workspace proyek yang dapat dibaca/ditulis/dimodifikasi oleh Paicode (mis. kode sumber, konfigurasi proyek, README). \\
\textbf{diff} & Representasi perubahan antar versi berkas (baris ditambah/diubah/dihapus). \\
\textbf{stateful} & Menjaga konteks/riwayat interaksi agar mempengaruhi langkah berikutnya. \\
\textbf{guardrail} & Pembatas/safeguard untuk mengurangi tindakan berisiko (mis. pembatasan ruang perubahan). \\
\textbf{workspace} & Direktori/lingkungan kerja proyek aktif tempat berkas proyek dikelola dan dimanipulasi. \\
\textbf{repository root} & Direktori akar dari repository proyek; menjadi patokan validasi dan normalisasi \textit{path}. \\
\textbf{rate limit} & Batas kuota/kecepatan permintaan API dalam jangka waktu tertentu yang ditetapkan penyedia layanan. \\
\textbf{tokenization} & Proses memecah teks menjadi unit-unit token yang diproses LLM; mempengaruhi biaya dan \textit{context window}. \\
\textbf{prompt} & Instruksi atau masukan yang diberikan ke LLM untuk menghasilkan keluaran. \\
\textbf{context window} & Batas panjang konteks (jumlah token) yang dapat dipertimbangkan LLM pada satu permintaan. \\
\textbf{API key} & Kredensial rahasia untuk mengakses layanan API; harus disimpan aman (jangan ditulis di repository publik). \\
\textbf{Single-Shot Intelligence} & Arsitektur agen AI yang mengoptimalkan efisiensi dengan sistem panggilan API terbatas: klasifikasi intensi, acknowledgment dinamis, perencanaan JSON, eksekusi adaptif 1-3 subfase, dan saran langkah berikutnya. \\
\textbf{agentic AI} & Sistem kecerdasan buatan yang mampu bertindak secara otonom dengan kemampuan observasi, perencanaan, dan eksekusi dalam lingkungan tertentu. \\
\textbf{acknowledgment dinamis} & Respons konfirmasi yang diberikan agen untuk mengakui dan memahami permintaan pengguna sebelum memulai perencanaan. \\
\textbf{interrupt handling} & Mekanisme penanganan interupsi (Ctrl+C) yang memungkinkan pengguna menghentikan respons AI tanpa keluar dari sesi. \\
\textbf{atomic write} & Teknik penulisan berkas yang menggunakan file sementara (tempfile) untuk memastikan operasi tulis berhasil sepenuhnya atau gagal total, mencegah korupsi data. \\
\textbf{threshold ganda} & Sistem pembatasan modifikasi berkas dengan dua kriteria: batas absolut (500 baris) dan batas relatif (50\% dari total baris berkas). \\
\textbf{SENSITIVE\_PATTERNS} & Daftar 7 pola direktori/berkas sensitif yang diblokir akses: .env, .git, venv, \_\_pycache\_\_, .pai\_history, .idea, .vscode. \\
\textbf{noise suppression} & Teknik menekan log yang berisik dari library gRPC/absl menggunakan environment variables khusus. \\
\textbf{entry point} & Titik masuk aplikasi yang didefinisikan dalam setup.cfg sebagai console script (pai = paicode.cli:main). \\
\textbf{prototyping iteratif} & Metode pengembangan dengan siklus berulang: perancangan, implementasi, uji coba, dan perbaikan untuk validasi asumsi dan penyempurnaan rancangan. \\
\textbf{markdown artifacts} & Sisa-sisa format markdown (seperti \texttt{```}, \texttt{**bold**}) dalam output LLM yang perlu dibersihkan sebelum ditampilkan. \\
\textbf{spinner status} & Indikator visual berputar yang menunjukkan bahwa sistem sedang memproses (misalnya saat LLM berpikir). \\
\textbf{multiline input} & Kemampuan input teks multi-baris dengan dukungan key bindings khusus (Alt+Enter untuk baris baru, Enter untuk submit). \\
\end{longtable}
