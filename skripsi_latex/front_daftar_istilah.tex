% Daftar Istilah (campuran Inggris-Indonesia sesuai kebutuhan teknis)
% Keep comments in English; content in Indonesian.
\begin{longtable}{p{0.25\textwidth}p{0.70\textwidth}}
\textbf{CLI} & Command Line Interface; antarmuka baris perintah pada terminal. \\
\textbf{LLM} & Large Language Model; model bahasa berskala besar untuk inferensi teks/kode. \\
\textbf{API} & Application Programming Interface; antarmuka pemrograman untuk mengakses layanan (mis. LLM). \\
\textbf{path} & Jalur berkas/direktori pada workspace proyek (contoh: \texttt{/home/user/project/main.py}). \\
\textbf{path security} & Kebijakan keamanan terkait path: normalisasi, validasi root, dan blokir direktori sensitif untuk mencegah akses yang tidak sah. \\
\textbf{project files (berkas proyek)} & Berkas-berkas aplikasi dalam workspace proyek yang dapat dibaca/ditulis/dimodifikasi oleh Paicode (mis. kode sumber, konfigurasi proyek, README). Tidak mencakup manajemen \textit{file system} OS. \\
\textbf{diff} & Representasi perubahan antar versi berkas (baris ditambah/diubah/dihapus). \\
\textbf{stateful} & Menjaga konteks/riwayat interaksi agar mempengaruhi langkah berikutnya. \\
\textbf{guardrail} & Pembatas/safeguard untuk mengurangi tindakan berisiko (mis. pembatasan ruang perubahan). \\
\end{longtable}
