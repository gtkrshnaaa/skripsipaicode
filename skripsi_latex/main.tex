% Main LaTeX file for the thesis
% All comments are in English as per coding convention, but content is Indonesian.

\documentclass[12pt,a4paper]{report}

% --- Packages ---
\usepackage[T1]{fontenc}
\usepackage[utf8]{inputenc}
\usepackage[indonesian]{babel}
\usepackage{mathptmx}
\usepackage{geometry}
\usepackage{setspace}
\usepackage{graphicx}

% --- Capitalize List Titles (Babel Override) ---
\addto\captionsindonesian{
  \renewcommand{\contentsname}{DAFTAR ISI}
  \renewcommand{\listfigurename}{DAFTAR GAMBAR}
  \renewcommand{\listtablename}{DAFTAR TABEL}
  \renewcommand{\listtablename}{DAFTAR TABEL}
}
\usepackage{tocloft} % Enhance TOC formatting
\renewcommand{\cftchapdotsep}{\cftdotsep} % Add dots for chapters
\renewcommand{\cftsecdotsep}{\cftdotsep} % Add dots for sections
% Center and Bold 14pt for List Titles
\renewcommand{\cfttoctitlefont}{\hfill\bfseries\fontsize{14pt}{17pt}\selectfont}
\renewcommand{\cftaftertoctitle}{\hfill}
\renewcommand{\cftloftitlefont}{\hfill\bfseries\fontsize{14pt}{17pt}\selectfont}
\renewcommand{\cftafterloftitle}{\hfill}
\renewcommand{\cftlottitlefont}{\hfill\bfseries\fontsize{14pt}{17pt}\selectfont}
\renewcommand{\cftafterlottitle}{\hfill}

\usepackage{hyperref}
\usepackage{caption}
\usepackage{subcaption}
\usepackage{float}
\usepackage{longtable}
\usepackage{booktabs}
\usepackage{array}
\usepackage{enumitem}
\usepackage{listings}
\usepackage{listingsutf8}
\usepackage{xcolor}
\usepackage{titlesec}
\usepackage{csquotes}
\usepackage{nameref}
\usepackage{indentfirst} % Ensure first paragraph is indented
\usepackage{fancyhdr} % For custom headers and footers
% Bibliography using BibTeX (simple and Ubuntu-friendly)
\usepackage[round]{natbib}
\renewcommand{\bibname}{DAFTAR PUSTAKA}
% Support for Unicode box-drawing characters in monospaced text
\usepackage{pmboxdraw}
% String manipulation for dynamic text
\usepackage{xstring}
% Native diagram and plotting packages (no external images)
\usepackage{tikz}
\usetikzlibrary{arrows.meta,positioning,shapes.geometric,fit}
\usepackage{pgfplots}
\pgfplotsset{compat=1.18}

% --- Sub-figure label formatting (e.g. Gambar 4.7.1) ---
\renewcommand{\thesubfigure}{\arabic{subfigure}}
\DeclareCaptionLabelFormat{mysublab}{\figurename\ \thefigure.\thesubfigure}
\captionsetup[subfigure]{labelformat=mysublab, labelsep=colon}


% --- Page setup ---
\geometry{a4paper, top=3cm, left=4cm, bottom=3cm, right=3cm, headheight=15pt}
\onehalfspacing

% --- Disable Hyphenation (Justified but no word breaking) ---
\tolerance=1
\emergencystretch=\maxdimen
\hyphenpenalty=10000
\hbadness=10000

% --- Table Padding ---
\setlength{\tabcolsep}{4pt}


% --- Remove extra paragraph spacing for consistency ---
\setlength{\parskip}{0pt} % No extra space between paragraphs
\setlength{\parindent}{1.27cm} % Standard indentation (0.5 inch / approx 1.27cm)

% --- Configure list spacing to match body text ---
\setlist{nosep, topsep=0pt, partopsep=0pt, itemsep=0pt, parsep=0pt}

% --- Hyperref setup ---
\hypersetup{
  colorlinks=true,
  linkcolor=black,
  citecolor=black,
  urlcolor=black,
  pdfauthor={I PUTU GEDE GILANG TEJA KRISHNA},
  pdftitle={PAICODE: AGENTIC AI BERBASIS CLI UNTUK OTOMASI AKTIVITAS PEMROGRAMAN DAN PENGEMBANGAN PERANGKAT LUNAK DI LINUX YANG DITENAGAI LLM MELALUI API}
}

% --- Listings (code) setup ---
\definecolor{codebg}{RGB}{245,245,245}
\lstdefinestyle{code}{
  backgroundcolor=\color{codebg},
  basicstyle=\ttfamily\small,
  keywordstyle=\color{blue},
  stringstyle=\color{teal},
  commentstyle=\color{gray},
  numbers=left,
  numberstyle=\tiny, 
  stepnumber=1,
  numbersep=8pt,
  showstringspaces=false,
  tabsize=2,
  breaklines=true,
  frame=single,
  framerule=0.3pt
}
\lstset{style=code}

% --- Chapter format (optional, cleaner look) ---
\titleformat{\chapter}[display]
  {\bfseries\fontsize{14pt}{17pt}\selectfont}
  {\centering\MakeUppercase{Bab}~\Roman{chapter}}
  {1ex}
  {\vspace{1ex}\centering}
\titlespacing{\chapter}{0pt}{0pt}{20pt}

% --- Section/Subsection format (Force 12pt as per requirement) ---
\titleformat{\section}
  {\bfseries\normalsize} % 12pt Bold
  {\thesection}
  {1em}
  {}
\titlespacing{\section}{0pt}{0pt}{0pt}

\titleformat{\subsection}
  {\bfseries\normalsize} % 12pt Bold
  {\thesubsection}
  {1em}
  {}
\titlespacing{\subsection}{0pt}{0pt}{0pt}

\titleformat{\subsubsection}
  {\bfseries\normalsize} % 12pt Bold
  {\thesubsubsection}
  {1em}
  {}
\titlespacing{\subsubsection}{0pt}{0pt}{0pt}

% --- Page styles with fancyhdr ---
% Style for frontmatter: page number at bottom center (modified as requested)
\fancypagestyle{frontmatterstyle}{
  \fancyhf{} % Clear all headers and footers
  \fancyfoot[C]{\thepage} % Page number at bottom center
  \renewcommand{\headrulewidth}{0pt} % No header rule
  \renewcommand{\footrulewidth}{0pt} % No footer rule
}

% Redefine plain to be same as frontmatterstyle (for TOC, LOF, LOT)
\fancypagestyle{plain}{
  \fancyhf{} % Clear all headers and footers
  \fancyfoot[C]{\thepage} % Page number at bottom center
  \renewcommand{\headrulewidth}{0pt} % No header rule
  \renewcommand{\footrulewidth}{0pt} % No footer rule
}

% Style for mainmatter: page number at Top Right (regular pages)
\fancypagestyle{mainmatterstyle}{
  \fancyhf{} % Clear all headers and footers
  \fancyhead[R]{\thepage} % Page number at top right
  \renewcommand{\headrulewidth}{0pt} % No header rule
  \renewcommand{\footrulewidth}{0pt} % No footer rule
}

% Style for chapter pages in mainmatter (optional custom name, though standard uses 'plain')
\fancypagestyle{chapterplain}{
  \fancyhf{} % Clear all headers and footers  
  \fancyfoot[C]{\thepage} % Page number at bottom center
  \renewcommand{\headrulewidth}{0pt} % No header rule
  \renewcommand{\footrulewidth}{0pt} % No footer rule
}

% --- Document meta (edit these) ---
\newcommand{\JudulSkripsi}{PAICODE: AGENTIC AI BERBASIS CLI UNTUK OTOMASI AKTIVITAS PEMROGRAMAN DAN PENGEMBANGAN PERANGKAT LUNAK DI LINUX YANG DITENAGAI LLM MELALUI API}
\newcommand{\NamaMahasiswa}{I PUTU GEDE GILANG TEJA KRISHNA}
\newcommand{\NIM}{225410001}
\newcommand{\Prodi}{Informatika}
\newcommand{\Fakultas}{Teknologi Informasi}
\newcommand{\Universitas}{Universitas Teknologi Digital Indonesia}
\newcommand{\Tahun}{2025}
\newcommand{\TahunAngkatan}{2022}

% --- Watermark Setup ---
\newcommand{\WatermarkBackground}{%
  \begin{tikzpicture}[remember picture, overlay]
    \node[opacity=1, inner sep=0pt] at ([xshift=0.5cm]current page.center) {
      \includegraphics[width=8cm]{assets/img/yellowscaleutdilogo.png}
    };
  \end{tikzpicture}%
}

\begin{document}
\setlength{\parindent}{1.27cm} % Force indentation inside document body
\newcommand{\customindent}{\leavevmode\hspace{1.27cm}} % Explicit indentation command


% --- Cover (Sampul Depan) ---
% Front Cover (Sampul Depan)
\begin{titlepage}
  \centering
  \vspace*{0.5cm}
  
  \textbf{\large TUGAS AKHIR}
  
  \vspace{0.3cm}
  \textbf{\large SKEMA SKRIPSI}
  
  \vspace{1.5cm}
  
  \textbf{\large \JudulSkripsi}
  
  \vfill
  
  % Logo Universitas Normal (Standard)
  \includegraphics[width=5cm]{assets/img/utdilogo.png}
  
  \vfill
  
  \textbf{\NamaMahasiswa}
  
  \textbf{NIM : \NIM}
  
  \vspace{1.5cm}
  
  \textbf{PROGRAM STUDI \Prodi}
  
  \textbf{PROGRAM SARJANA}
  
  \textbf{FAKULTAS \Fakultas}
  
  \textbf{UNIVERSITAS TEKNOLOGI DIGITAL INDONESIA}
  
  \textbf{YOGYAKARTA}
  
  \textbf{\Tahun}
  
\end{titlepage}


% --- Title Page (Halaman Judul) ---
% Start Roman numbering from here (i)
\pagenumbering{roman}
\pagestyle{frontmatterstyle} % Apply frontmatter style (page number top right)
% Front Title Page (Halaman Judul)
\begin{titlepage}
  \centering
  \WatermarkBackground % Add the yellow watermark background
  \addcontentsline{toc}{chapter}{HALAMAN JUDUL}
  
  \vspace*{0.5cm}
  
  \textbf{\fontsize{14pt}{17pt}\selectfont TUGAS AKHIR}
  
  \vspace{0.3cm}
  \textbf{\fontsize{14pt}{17pt}\selectfont SKEMA SKRIPSI}
  
  \vspace{1.5cm}
  
  \textbf{\normalsize \JudulSkripsi}
  
  \vfill
  
  \textbf{Diajukan sebagai salah satu syarat untuk menyelesaikan studi pada}
  
  \textbf{Program Sarjana}
  
  \textbf{Program Studi \Prodi}
  
  \textbf{Fakultas \Fakultas}
  
  \textbf{Universitas Teknologi Digital Indonesia}
  
  \vfill
  
  \textbf{Disusun Oleh}
  
  \vspace{0.5cm}
  \textbf{\NamaMahasiswa}
  
  \textbf{NIM : \NIM}
  
  \vspace{1.5cm}
  
  \textbf{PROGRAM STUDI \Prodi}
  
  \textbf{PROGRAM SARJANA}
  
  \textbf{FAKULTAS \Fakultas}
  
  \textbf{UNIVERSITAS TEKNOLOGI DIGITAL INDONESIA}
  
  \textbf{YOGYAKARTA}
  
  \textbf{\Tahun}
  
\end{titlepage}


% Skip page ii (blank/hidden) so Approval starts at iii
\stepcounter{page}

% --- Approval Pages (iii, iv) ---
% Front Approval Page (Halaman Persetujuan)
\thispagestyle{empty}
\WatermarkBackground

\begin{center}
  \textbf{\fontsize{14pt}{17pt}\selectfont HALAMAN PERSETUJUAN UJIAN TUGAS AKHIR}
\end{center}

\noindent
\begin{tabular}{@{}ll@{}}
  Judul & : \parbox[t]{10cm}{\JudulSkripsi} \\[0.5cm]
  Nama & : \NamaMahasiswa \\
  NIM & : \NIM \\
  Program Studi & : \Prodi \\
  Program & : Sarjana \\
  Semester & : Gasal \\
  Tahun Akademik & : 2025/2026 \\
\end{tabular}

\vspace{1cm}

\begin{center}
  Telah diperiksa dan disetujui untuk diujikan \\
  di hadapan Dewan Penguji Tugas Akhir
  
  \vspace{1cm}
  
  Yogyakarta, 24 November 2025
  
  Dosen Pembimbing,
  
  \vspace{2cm}
  
  \underline{Dr. Bambang Purnomosidi Dwi Putranto, S.E., Akt., S.Kom., MMSI} \\
  NIDN: 0505058801
\end{center}

% Front matter: Lembar Pengesahan
\thispagestyle{frontmatterstyle}
\addcontentsline{toc}{chapter}{HALAMAN PENGESAHAN}
\WatermarkBackground

\begin{center}
  \textbf{\fontsize{14pt}{17pt}\selectfont HALAMAN PENGESAHAN}
  
  \vspace{0.5cm}
  
  \textbf{\JudulSkripsi}
  
  \vspace{0.5cm}
  
  Telah dipertahankan di depan Dewan Penguji dan dinyatakan diterima untuk \\
  memenuhi sebagian persyaratan guna memperoleh \\
  Gelar Sarjana Komputer \\
  Program Studi \Prodi \\
  Fakultas \Fakultas \\
  Universitas Teknologi Digital Indonesia
  
  \vspace{0.5cm}
  
  Yogyakarta, 15 Desember 2025
  
\end{center}

\vspace{0.5cm}

\noindent
\begin{tabular}{@{} >{\raggedright\arraybackslash}p{7.9cm} p{2.5cm} p{3cm} @{}}
  Dewan Penguji & NIDN & Tandatangan \\[0.3cm]
  1. Wagito, S.T., M.T. (Ketua) & 0522126901 & ............. \\[0.5cm]
  2. Dr. Bambang Purnomosidi Dwi Putranto, S.E., Akt., S.Kom., MMSI (Sekretaris) & ............. & ............. \\[0.5cm]
  3. Ariesta Damayanti, S.Kom., M.Cs. (Anggota) & 0020047801 & ............. \\[0.3cm]
\end{tabular}

\vspace{0.5cm}

\begin{center}
  Mengetahui \\
  Ketua Program Studi \Prodi
  
  \vspace{2cm}
  
  Dini Fakta Sari, S.T., M.T. \\
  NIDN: 0507108401
\end{center}


% --- Declaration of Authenticity (v) ---
% Front matter: Pernyataan Keaslian (placeholder)
% Content in Indonesian; comments in English.
\chapter*{Pernyataan Keaslian}
\addcontentsline{toc}{chapter}{Pernyataan Keaslian}

Saya yang bertanda tangan di bawah ini, menyatakan bahwa skripsi ini adalah hasil karya sendiri dan tidak memuat karya orang lain yang pernah diajukan untuk memperoleh gelar akademik di perguruan tinggi manapun, kecuali bagian-bagian tertentu yang dirujuk sebagaimana tercantum dalam daftar pustaka.

\vspace{2cm}
Yogyakarta, .............. \\
\textbf{I PUTU GEDE GILANG TEJA KRISHNA} \\
NIM: 225410001


% --- Dedication (vi) ---
% Front matter: Persembahan
\thispagestyle{frontmatterstyle}
\addcontentsline{toc}{chapter}{HALAMAN PERSEMBAHAN}

\begin{center}
  \textbf{\fontsize{14pt}{17pt}\selectfont HALAMAN PERSEMBAHAN}
\end{center}

\vspace{1cm}


\indent Tugas Akhir ini saya persembahkan kepada kedua orang tua tercinta yang telah memberikan doa, dukungan, dan kasih sayang yang tiada henti; seluruh keluarga besar yang senantiasa memberikan motivasi dan semangat; para guru dan dosen yang telah membimbing dan memberikan ilmu yang bermanfaat; serta seluruh teman-teman di kampus dan rekan seperjuangan UTDI THE ARCADE.


% --- Preface (vii) ---
% Front matter: Kata Pengantar (placeholder)
% Content in Indonesian; comments in English.
\chapter*{Kata Pengantar}
\addcontentsline{toc}{chapter}{Kata Pengantar}

Puji syukur ke hadirat Tuhan Yang Maha Esa atas limpahan rahmat dan karunia-Nya sehingga skripsi ini dapat diselesaikan. Naskah ini disusun untuk memenuhi salah satu syarat memperoleh gelar Sarjana pada Program Studi \Prodi, \Fakultas, \Universitas.

Ucapan terima kasih disampaikan kepada:
\begin{enumerate}
  \item Orang tua dan keluarga atas doa, dukungan, serta motivasi yang tiada henti.
  \item Dosen pembimbing I dan II atas bimbingan, arahan, dan waktu yang dicurahkan selama proses penyusunan.
  \item Para dosen dan staf di \Fakultas{} atas ilmu, kesempatan, dan fasilitas yang diberikan.
  \item Rekan-rekan mahasiswa yang telah memberikan bantuan, masukan, dan semangat selama penelitian ini berlangsung.
  \item Pihak-pihak lain yang tidak dapat disebutkan satu per satu, yang turut berkontribusi dalam bentuk apa pun.
\end{enumerate}

Penulis menyadari bahwa skripsi ini masih memiliki keterbatasan. Oleh karena itu, kritik dan saran yang membangun sangat diharapkan demi perbaikan pada penelitian selanjutnya.

\vspace{1cm}
\begin{tabular}{p{0.55\textwidth}p{0.4\textwidth}}
& Yogyakarta, ................................ \\
& \vspace{1.6cm} \\
& \textbf{\NamaMahasiswa} \\
& NIM: \NIM \\
\end{tabular}


% --- Table of contents & lists (viii, ix, x) ---
\cleardoublepage
\addcontentsline{toc}{chapter}{DAFTAR ISI}
\tableofcontents

\cleardoublepage
\addcontentsline{toc}{chapter}{DAFTAR GAMBAR}
\listoffigures

\cleardoublepage
\addcontentsline{toc}{chapter}{DAFTAR TABEL}
\listoftables

% --- Abstract (xi, xii) ---
\chapter*{INTISARI}
\addcontentsline{toc}{chapter}{INTISARI}
Penelitian ini mengusulkan \textbf{Paicode}, sebuah agen AI berbasis Command Line Interface (CLI) untuk membantu proses pengembangan perangkat lunak secara interaktif dengan arsitektur \textit{Single-Shot Intelligence}. Sistem berjalan pada lingkungan terminal lokal dan melakukan \textbf{operasi berkas tingkat-aplikasi di ruang kerja proyek (project workspace)}; namun \textbf{mengirimkan cuplikan kode/konteks ke layanan LLM (Gemini) melalui API} untuk keperluan inferensi. Oleh karena itu, aspek privasi dan kerahasiaan kode \textbf{bergantung pada kebijakan penyedia API}, sedangkan pengamanan lokal difokuskan pada kebijakan \textit{path security}. Himpunan perintah yang disediakan (mis. \texttt{READ}, \texttt{WRITE}, \texttt{MODIFY}, \texttt{TREE}, \texttt{LIST\_PATH}) memungkinkan agen mengobservasi proyek, memanipulasi berkas, dan memodifikasi kode secara terarah dengan sistem perubahan berbasis \textit{diff}.

Arsitektur \textit{Single-Shot Intelligence} meningkatkan efisiensi dengan sistem panggilan API yang terdiri dari: (1) klasifikasi intensi, (2) acknowledgment dinamis, (3) fase perencanaan untuk analisis mendalam dan perencanaan komprehensif dalam format JSON, (4) fase eksekusi adaptif yang dapat berjalan dalam 1-3 subfase berdasarkan kompleksitas tugas, dan (5) saran langkah berikutnya. Sistem mencakup manajemen API key tunggal, \textit{interrupt handling} (Ctrl+C), dan pencatatan sesi ke \texttt{.pai\_history}.

Metode yang digunakan adalah \textit{Research and Development} (R\&D) dengan pendekatan \textit{prototyping} iteratif. Evaluasi dilakukan melalui skenario tugas representatif, dengan metrik efisiensi (jumlah panggilan API), ketepatan hasil (kompilasi/eksekusi), dan kepatuhan keamanan \textit{path}. Hasil menunjukkan bahwa agen \textit{stateful} dengan arsitektur \textit{Single-Shot Intelligence} dan pembatasan perubahan berbasis \textit{diff} dengan threshold ganda (500 baris absolut dan 50\% ratio maksimal) memudahkan pengembangan bertahap. Sistem eksekusi adaptif dengan 1-3 subfase menunjukkan efisiensi waktu operasional dibandingkan pendekatan tradisional yang memerlukan banyak panggilan API berulang, dengan tetap mempertahankan kualitas hasil yang baik.

\textbf{Kata kunci}: AI, agen, CLI, LLM, API, keamanan, pengembangan, perangkat lunak.

\chapter*{ABSTRACT}
\addcontentsline{toc}{chapter}{ABSTRACT}
This thesis presents \textbf{Paicode}, an agentic AI for the Command Line Interface (CLI) that assists software development through interactive, stateful workflows with a \textit{Single-Shot Intelligence} architecture. The system runs on a local terminal and performs \textbf{application-level file operations within the project workspace}, while \textbf{sending code/context snippets to an external LLM (Gemini) via API} for inference. Consequently, privacy and confidentiality \textbf{depend on the provider's policy}, whereas local safeguards focus on path-security policies. A compact set of commands (e.g., \texttt{READ}, \texttt{WRITE}, \texttt{MODIFY}, \texttt{TREE}, \texttt{LIST\_PATH}) enables the agent to observe the project, manipulate files, and apply targeted code modifications with \textit{diff}-based change system.

The \textit{Single-Shot Intelligence} architecture improves efficiency through an API call system consisting of: (1) intent classification, (2) dynamic acknowledgment, (3) planning phase for deep analysis and comprehensive JSON-based planning, (4) adaptive execution phase that can run in 1-3 sub-phases based on task complexity, and (5) next-step suggestions. The system includes single API key management, \textit{interrupt handling} (Ctrl+C), and session logging to \texttt{.pai\_history}.

We adopt a Research and Development approach with iterative prototyping. The evaluation uses representative programming scenarios and measures efficiency (API call count), correctness (build/run), and security compliance. Results indicate that a stateful agent with \textit{Single-Shot Intelligence} and \textit{diff}-based change constraints with dual thresholds (500-line absolute and 50\% maximum ratio) facilitates incremental development while reducing the risk of unintended overwrites. The adaptive execution system with 1-3 sub-phases proves more efficient than traditional approaches requiring multiple repetitive API calls, while maintaining high result quality.

\textbf{Keywords}: AI, agent, CLI, LLM, API, security, software, development.


% --- Switch to arabic numbering for main content ---
\cleardoublepage
\pagenumbering{arabic}
% Redefine plain for chapter pages in mainmatter
\fancypagestyle{plain}{
  \fancyhf{}
  \fancyfoot[C]{\thepage}
  \renewcommand{\headrulewidth}{0pt}
  \renewcommand{\footrulewidth}{0pt}
}
\pagestyle{mainmatterstyle} % Apply mainmatter style (page number bottom center)

% --- Chapters ---
% Chapter 1 - Pendahuluan
% Keep comments in English; content in Indonesian.
\chapter{Pendahuluan}
\label{chap:pendahuluan}

\section{Latar Belakang}
Perkembangan \textit{Large Language Model} (LLM) telah mendorong lahirnya beragam asisten pemrograman yang mampu membantu pengembang perangkat lunak dalam menulis, meninjau, dan memodifikasi kode. Meskipun demikian, sebagian besar asisten tersebut beroperasi sebagai ekstensi editor atau layanan berbasis \textit{cloud} yang menyimpan, memproses, atau melatih dari data pengguna. Kondisi ini menimbulkan kekhawatiran terkait privasi, kendali atas data, serta ketergantungan pada antarmuka tertentu.

Di sisi lain, \textit{Command Line Interface} (CLI) tetap menjadi lingkungan kerja yang penting bagi banyak pengembang karena sifatnya yang ringan, dapat diotomasi, dan mudah diintegrasikan dengan beragam alat. Integrasi kemampuan agen cerdas yang \textit{stateful} dan \textit{proactive} ke dalam CLI berpotensi mempercepat proses pengembangan perangkat lunak. Dalam konteks Paicode, sistem berjalan pada terminal lokal dan mengeksekusi tindakan langsung pada \textbf{berkas proyek di workspace}; namun, cuplikan kode/konteks \textbf{dikirim ke layanan LLM melalui API} untuk keperluan inferensi \cite{brown2020gpt3,openai2023gpt4,gemini2023}. Dengan demikian, aspek privasi/kerahasiaan kode \textbf{bergantung pada kebijakan penyedia API}, sementara pengamanan di sisi lokal difokuskan pada kebijakan \textit{path security} (keamanan \textit{path}) dan pembatasan perubahan berbasis \textit{diff}.

Penelitian ini menghadirkan \textbf{Paicode}, sebuah agen AI berbasis CLI yang dirancang untuk membantu proses pengembangan perangkat lunak secara interaktif dengan arsitektur 8-langkah (\textit{8-step workflow}). Paicode mampu: (i) mengobservasi struktur proyek (\texttt{TREE}, \texttt{LIST\_PATH}); (ii) membaca dan menulis berkas proyek (\texttt{READ}, \texttt{WRITE}); (iii) memodifikasi kode secara terarah dengan batasan perubahan berbasis diff (\texttt{MODIFY}) hingga maksimal 500 baris per modifikasi; (iv) menegakkan kebijakan keamanan path pada berkas proyek (memblokir akses ke direktori sensitif seperti \texttt{.git}, \texttt{venv}, dan \texttt{.env}); (v) melakukan klasifikasi intensi pengguna (\textit{chat} vs \textit{task}); (vi) menjalankan fase berpikir (\textit{thinking phase}) sebelum eksekusi; serta (vii) melakukan pemeriksaan integritas (\textit{integrity check}) pasca-eksekusi dengan skor kualitas 1-10. Sistem diimplementasikan pada lingkungan Ubuntu dengan bahasa pemrograman Python, pengelolaan dependensi melalui Poetry, mendukung manajemen multi-API key dengan \textit{round-robin load balancing}, dan menggunakan API Gemini sebagai LLM.

\section{Rumusan Masalah}
Berdasarkan latar belakang tersebut, rumusan masalah yang diajukan adalah sebagai berikut:

\begin{enumerate}
  \item Bagaimana merancang arsitektur agen AI berbasis CLI dengan \textit{8-step workflow} yang mencakup klasifikasi intensi, perencanaan tugas, fase berpikir, eksekusi aksi, dan pemeriksaan integritas, disertai integrasi LLM melalui API dan manajemen multi-API key dengan \textit{round-robin load balancing}?
  \item Bagaimana mengimplementasikan kebijakan keamanan \textit{path} dan pembatasan perubahan berbasis \textit{diff} dengan threshold maksimal 500 baris per modifikasi untuk mencegah penimpaan berkas yang tidak diinginkan?
  \item Bagaimana mengintegrasikan kemampuan observasi proyek, manipulasi berkas, serta modifikasi kode terarah dengan mekanisme \textit{interrupt handling} (Ctrl+C) dan fitur \textit{auto-continue} untuk pengalaman interaktif yang lebih baik?
  \item Bagaimana mengevaluasi efektivitas Paicode dalam membantu tugas-tugas pemrograman dengan metrik efisiensi, ketepatan hasil, dan skor kualitas dari sistem \textit{integrity check}?
\end{enumerate}

\section{Batasan Masalah}
Agar fokus penelitian terjaga dan implementasi dapat dilakukan secara terukur, batasan-batasan berikut ditetapkan:

\begin{itemize}
  \item Lingkungan target adalah sistem operasi Ubuntu (Linux) dengan antarmuka CLI.
  \item Bahasa pemrograman utama adalah Python; contoh dan skenario uji berfokus pada ekosistem Python/Unix.
  \item Layanan LLM eksternal menggunakan API Gemini; kualitas respons bergantung pada model dan tidak menjadi ruang lingkup untuk dioptimasi ulang.
  \item Dukungan multi-pengguna, kolaborasi real-time, dan integrasi langsung dengan editor tidak dibahas pada versi ini.
  \item Aspek visual seperti diagram dan ilustrasi antarmuka ditunda pada tahap akhir; fokus laporan adalah narasi dan hasil teknis.
\end{itemize}

\section{Tujuan Penelitian}
Tujuan penelitian ini adalah membangun dan mengevaluasi sebuah agen AI berbasis CLI yang dapat membantu pengembang dalam proses pemrograman secara interaktif dengan arsitektur 8-langkah. Secara khusus, penelitian menargetkan:

\begin{enumerate}
  \item Merancang arsitektur Paicode yang mencakup modul agen dengan \textit{8-step workflow} (klasifikasi intensi, perencanaan tugas, fase berpikir, eksekusi aksi, pemeriksaan integritas, dan ringkasan akhir), jembatan LLM dengan manajemen multi-API key, antarmuka CLI dengan \textit{interrupt handling}, lapisan keamanan path pada berkas proyek, serta komponen tampilan terminal.
  \item Mengimplementasikan kemampuan observasi proyek, manipulasi berkas, dan modifikasi kode terarah dengan mekanisme \textit{patch/diff} yang membatasi perubahan maksimal 500 baris per modifikasi (dapat dikonfigurasi melalui variabel lingkungan).
  \item Mengintegrasikan fitur-fitur interaktif seperti \textit{auto-continue} untuk melanjutkan saran otomatis, pencatatan sesi ke \texttt{.pai\_history}, dan sistem penilaian kualitas dengan skor 1-10 pada setiap eksekusi.
  \item Menyusun prosedur evaluasi dengan skenario tugas pemrograman yang representatif dan mengukur peningkatan produktivitas, kualitas hasil, serta efektivitas sistem \textit{integrity check}.
\end{enumerate}

\section{Manfaat Penelitian}
Manfaat yang diharapkan dari penelitian ini meliputi:

\begin{itemize}
  \item \textbf{Akademis:} menyediakan studi kasus dan arsitektur rujukan untuk pengembangan agen AI berbasis CLI dengan integrasi LLM melalui API, serta memperkaya literatur mengenai integrasi LLM dalam alur kerja rekayasa perangkat lunak.
  \item \textbf{Praktis:} menghadirkan alat bantu yang \textit{privacy-aware} dan mudah diintegrasikan dengan berbagai IDE karena beroperasi langsung pada berkas proyek di ruang kerja (workspace); memfasilitasi pembuatan struktur proyek, pembacaan, dan modifikasi kode secara cepat dan terarah.
\end{itemize}

% Chapter 2 - Tinjauan Pustaka
\chapter{Tinjauan Pustaka}
\label{chap:tinjauan}

\section{Teori Dasar}
Bahas konsep-konsep: Command Line Interface (CLI), AI Agent, Large Language Model (LLM), local-first software, manajemen dependensi (Poetry), dan Rich TUI.

\section{Penelitian Terkait}
Ringkas penelitian/alat terkait AI coding assistant atau agentic AI. Bandingkan pendekatan dan hasil.

\section{Posisi Penelitian}
Posisikan kontribusi paicode dibandingkan karya-karya sebelumnya.

% Example citation usage (keep, then replace with real keys):
% Menurut \citet{goodfellow2016deep}, ... atau penelitian terkait \citep{somepaper2020}.

% Chapter 3 - Metodologi Penelitian
\chapter{Metodologi Penelitian}
\label{chap:metodologi}

\section{Metode Pengembangan}
Jelaskan model pengembangan (prototyping, waterfall, agile, R\&D) yang digunakan dan alasan pemilihan.

\section{Arsitektur Sistem}
Jelaskan arsitektur paicode: modul \texttt{agent.py}, \texttt{llm.py}, \texttt{fs.py}, \texttt{cli.py}, dan \texttt{ui.py}. Sertakan diagram alir atau class diagram bila perlu.

\section{Alat dan Lingkungan}
Sebutkan Python, Poetry, Git, Google Gemini API, serta lingkungan Ubuntu.

\section{Prosedur Penelitian}
Jabarkan langkah-langkah eksperimen, pengumpulan data, dan kriteria evaluasi.

% Chapter 4 - Implementasi dan Hasil
\chapter{Implementasi dan Hasil}
\label{chap:implementasi}

\section{Implementasi Paicode}
Implementasi dilakukan menggunakan Python dengan manajemen dependensi pip dan virtual environment. Berkas \texttt{setup.cfg} mendefinisikan paket yang dibutuhkan beserta titik masuk CLI. Instalasi otomatis melalui Makefile. Langkah instalasi dan konfigurasi sebagai berikut.

\subsection{Instalasi}
\begin{enumerate}
  \item Pastikan Python (\texttt{\textgreater= 3.10}) terpasang sesuai spesifikasi \texttt{setup.cfg}.
  \item Masuk ke direktori \texttt{paicode/} dan jalankan:
\end{enumerate}

\begin{lstlisting}[language=bash,caption={Instalasi dependensi dengan Makefile}]
make install
\end{lstlisting}

\subsection{Konfigurasi API Key}
Paicode menggunakan manajemen API key tunggal dengan migrasi otomatis dari sistem multi-key. Kunci disimpan secara aman dalam format JSON pada \texttt{~/.config/pai-code/credentials.json} dengan izin berkas 0o600.

\begin{lstlisting}[language=bash,caption={Manajemen API key tunggal Gemini}]
# Mengatur API key
pai config set <API_KEY_GEMINI>

# Melihat API key saat ini (masked)
pai config show

# Validasi API key
pai config validate

# Menghapus API key
pai config remove
\end{lstlisting}

Sistem akan secara otomatis melakukan migrasi dari konfigurasi multi-key lama (version 1) ke sistem single-key baru (version 2).

\subsection{Menjalankan Agen}
Sesi interaktif dapat dimulai langsung dengan berbagai opsi konfigurasi:

\begin{lstlisting}[language=bash,caption={Menjalankan sesi agen interaktif}]
# Menjalankan dengan konfigurasi default
pai

# Menjalankan dengan model dan temperature tertentu
pai auto --model gemini-2.5-flash-lite --temperature 0.3

# Menggunakan variabel lingkungan untuk konfigurasi
export PAI_MODEL="gemini-2.5-flash-lite"
export PAI_TEMPERATURE="0.3"
export PAI_MODIFY_THRESHOLD="500"
export PAI_MODIFY_MAX_RATIO="0.5"
pai
\end{lstlisting}

Selama sesi, pengguna dapat:
\begin{itemize}
  \item Menekan Ctrl+C sekali untuk menghentikan respons AI (sesi tetap aktif)
  \item Menekan Ctrl+C dua kali untuk keluar dari sesi
  \item Mengetik \texttt{exit} atau \texttt{quit} untuk mengakhiri sesi
\end{itemize}

\section{Alur Interaksi dengan Single-Shot Intelligence}
Alur kerja pada sesi interaktif mengikuti arsitektur \textit{Single-Shot Intelligence}:

\begin{enumerate}
  \item \textbf{Klasifikasi Intensi}: Agen mengklasifikasikan input pengguna sebagai \textit{chat} (diskusi/pertanyaan) atau \textit{task} (tugas pemrograman). Untuk mode \textit{chat}, agen langsung memberikan respons tanpa eksekusi perintah.
  \item \textbf{Acknowledgment Dinamis}: Agen memberikan konfirmasi pemahaman terhadap permintaan pengguna sebelum memulai perencanaan.
  \item \textbf{Fase Perencanaan}: LLM melakukan analisis mendalam dan menghasilkan perencanaan komprehensif dalam format JSON yang terstruktur.
  \item \textbf{Fase Eksekusi Adaptif}: Eksekusi perintah dalam 1-3 subfase berdasarkan kompleksitas tugas, menggunakan perintah workspace (\texttt{READ}, \texttt{WRITE}, \texttt{MODIFY}, \texttt{TREE}, \texttt{LIST\_PATH}, \texttt{MKDIR}, \texttt{TOUCH}, \texttt{RM}, \texttt{MV}, \texttt{FINISH}) dengan batasan threshold ganda (500 baris absolut dan 50\% ratio maksimal).
  \item \textbf{Saran Langkah Berikutnya}: Agen memberikan saran untuk langkah selanjutnya berdasarkan hasil eksekusi.
\end{enumerate}

Operasi berkas dieksekusi melalui \textbf{Workspace Controller} (\texttt{workspace.py}) dengan penegakan kebijakan \textit{path security} yang mencegah akses ke 7 pola direktori sensitif: \texttt{.env}, \texttt{.git}, \texttt{venv}, \texttt{\_\_pycache\_\_}, \texttt{.pai\_history}, \texttt{.idea}, \texttt{.vscode}. Seluruh interaksi dicatat ke \texttt{.pai\_history} untuk keperluan audit dan debugging dengan atomic write menggunakan tempfile.

\section{Gambar Implementasi}
Bagian ini menampilkan screenshot implementasi aktual yang memperkuat penjelasan implementasi dan hasil.

% Figure 4.1: Tampilan awal sesi agen di terminal
\begin{figure}[htbp]
  \centering
  \fbox{\parbox{0.95\textwidth}{\centering Placeholder gambar: `img/fig4-1-sesi-awal-cli.png`\\
  (Screenshot terminal: pembukaan sesi agen, panel "Interactive Auto Mode")}}
  \caption{Tampilan awal sesi agen di terminal.}
  \label{fig:sesi-awal-cli}
\end{figure}

Pada Gambar~\ref{fig:sesi-awal-cli} diperlihatkan antarmuka awal sesi agen yang akan menjadi konteks interaksi.

% Figure 4.2: Perintah TREE menampilkan struktur direktori
\begin{figure}[htbp]
  \centering
  \fbox{\parbox{0.95\textwidth}{\centering Placeholder gambar: `img/fig4-2-tree-output.png`\\
  (Screenshot hasil perintah TREE pada proyek uji)}}
  \caption{Output perintah \texttt{TREE} untuk observasi struktur proyek.}
  \label{fig:tree-output}
\end{figure}

Pada Gambar~\ref{fig:tree-output} ditunjukkan hasil observasi struktur direktori yang digunakan agen sebagai dasar perencanaan aksi.

% Figure 4.3: Perintah LIST_PATH menampilkan daftar path
\begin{figure}[htbp]
  \centering
  \fbox{\parbox{0.95\textwidth}{\centering Placeholder gambar: `img/fig4-3-list-path.png`\\
  (Screenshot hasil perintah \texttt{LIST\_PATH} dengan format baris per baris)}}
  \caption{Output perintah \texttt{LIST\_PATH} untuk daftar path mesin-baca.}
  \label{fig:list-path}
\end{figure}

% Figure 4.4: Panel pembacaan berkas (READ) dengan penyorotan sintaks
\begin{figure}[htbp]
  \centering
  \fbox{\parbox{0.95\textwidth}{\centering Placeholder gambar: `img/fig4-4-read-panel.png`\\
  (Panel kode dengan line number dan syntax highlighting saat READ)}}
  \caption{Panel pembacaan berkas dengan penyorotan sintaks.}
  \label{fig:read-panel}
\end{figure}

% Figure 4.5: Modifikasi terarah (MODIFY) beserta ringkasan diff
\begin{figure}[htbp]
  \centering
  \fbox{\parbox{0.95\textwidth}{\centering Placeholder gambar: `img/fig4-5-modify-diff.png`\\
  (Cuplikan hasil MODIFY yang menampilkan ringkasan diff/lines changed)}}
  \caption{Contoh hasil perintah \texttt{MODIFY} dengan batasan perubahan berbasis \textit{diff}.}
  \label{fig:modify-diff}
\end{figure}

% Figure 4.6: Diagram alur evaluasi dan metrik
\begin{figure}[htbp]
  \centering
  \fbox{\parbox{0.95\textwidth}{\centering Placeholder gambar: `img/fig4-6-evaluasi-metrik.png`\\
  (Diagram alur evaluasi: skenario → eksekusi → pencatatan metrik → analisis)}}
  \caption{Diagram alur evaluasi dan metrik yang dikumpulkan.}
  \label{fig:evaluasi-metrik}
\end{figure}

% Figure 4.7: Visualisasi hasil (grafik waktu/langkah)
\begin{figure}[htbp]
  \centering
  \fbox{\parbox{0.95\textwidth}{\centering Placeholder gambar: `img/fig4-7-grafik-hasil.png`\\
  (Grafik batang/garis: perbandingan waktu dan jumlah langkah antar skenario)}}
  \caption{Contoh visualisasi hasil awal untuk metrik efisiensi.}
  \label{fig:grafik-hasil}
\end{figure}

\section{Tabel Skenario Pengujian}
Tabel~\ref{tab:skenario-uji} merangkum skenario uji yang digunakan untuk mengevaluasi Paicode.

\begin{longtable}{@{}p{0.26\textwidth}p{0.52\textwidth}p{0.16\textwidth}@{}}
  \caption{Skenario Pengujian Paicode}\label{tab:skenario-uji}\\
  \toprule
  \textbf{Skenario} & \textbf{Deskripsi} & \textbf{Artefak Bukti} \\
  \midrule
  \endfirsthead
  \toprule
  \textbf{Skenario} & \textbf{Deskripsi} & \textbf{Artefak Bukti} \\
  \midrule
  \endhead
  Pembuatan Proyek & Agen membuat struktur proyek Python sederhana (direktori, file, \texttt{README}) & SS: \texttt{TREE} \\
  Pembacaan Kode & Agen menampilkan isi file sumber dan menjelaskan ringkas & SS: panel \texttt{READ} \\
  Modifikasi Terarah & Agen menerapkan perubahan kecil pada fungsi (\textit{diff}-based) & SS: \texttt{MODIFY} + diff \\
  Refactoring Ringan & Agen memecah fungsi panjang menjadi beberapa fungsi kecil & SS: diff + build \\
  Dokumentasi & Agen menulis docstring/README singkat & SS: panel \texttt{WRITE} \\
  \bottomrule
\end{longtable}

\section{Tabel Metrik Evaluasi}
Tabel~\ref{tab:metrik-evaluasi} mendeskripsikan metrik dan cara pengukurannya.

\begin{longtable}{@{}p{0.24\textwidth}p{0.56\textwidth}p{0.16\textwidth}@{}}
  \caption{Metrik Evaluasi dan Definisi Operasional}\label{tab:metrik-evaluasi}\\
  \toprule
  \textbf{Metrik} & \textbf{Definisi} & \textbf{Satuan} \\
  \midrule
  \endfirsthead
  \toprule
  \textbf{Metrik} & \textbf{Definisi} & \textbf{Satuan} \\
  \midrule
  \endhead
  Waktu & Durasi dari awal perintah sampai hasil akhir pada setiap skenario & detik \\
  Langkah & Jumlah aksi agen (\texttt{READ}, \texttt{WRITE}, dsb.) per skenario & langkah \\
  Keberhasilan Build/Run & Status eksekusi program/kompilasi setelah perubahan & biner/rasio \\
  Ukuran Perubahan & Banyaknya baris yang ditambah/ubah/hapus berdasarkan \textit{diff} & baris \\
  Kepatuhan Path & Tidak ada akses ke direktori sensitif; validasi path terpenuhi & biner/rasio \\
  \bottomrule
\end{longtable}

\section{Tabel Konfigurasi Lingkungan}
Tabel~\ref{tab:konfigurasi} menampilkan konfigurasi lingkungan yang digunakan selama pengujian.

\begin{longtable}{@{}p{0.30\textwidth}p{0.64\textwidth}@{}}
  \caption{Konfigurasi Lingkungan Uji}\label{tab:konfigurasi}\\
  \toprule
  \textbf{Komponen} & \textbf{Spesifikasi} \\
  \midrule
  \endfirsthead
  \toprule
  \textbf{Komponen} & \textbf{Spesifikasi} \\
  \midrule
  \endhead
  Sistem Operasi & Ubuntu (Linux) \\
  Python & \texttt{\textgreater= 3.10} (sesuai spesifikasi \texttt{setup.cfg}) \\
  Manajer Dependensi & pip dan virtual environment; titik masuk CLI pada \texttt{setup.cfg} \\
  LLM Provider & Gemini melalui \texttt{google-generativeai} (API) \\
  TUI & \texttt{rich} untuk panel dan penyorotan sintaks \\
  LaTeX & TeX Live; kompilasi via Makefile \\
  Perangkat Keras & CPU x86\_64; RAM minimal 8 GB (contoh) \\
  \bottomrule
\end{longtable}
\section{Contoh Sesi}
Cuplikan berikut menggambarkan pembuatan proyek sederhana dan pembacaan isi berkas.

\begin{lstlisting}[language=bash,caption={Contoh interaksi singkat}]
$ pai
> buatkan proyek python sederhana: BMI Calculator
# Agen mengeksekusi: MKDIR, TOUCH, WRITE
> tampilkan struktur
# Agen mengeksekusi: TREE
> tampilkan isi kode sumber
# Agen mengeksekusi: READ
\end{lstlisting}

% Example of including code from the repository (adjust path if needed):
% \lstinputlisting[language=Python, caption={Cuplikan kode agent}, label={lst:agent}]{../paicode/paicode/agent.py}

\section{Evaluasi}
Evaluasi dilakukan melalui skenario tugas representatif yang mencakup pembuatan struktur proyek, penulisan berkas sumber, pembacaan, dan modifikasi terarah. Metrik yang diukur meliputi:

\begin{itemize}
  \item Waktu penyelesaian tugas.
  \item Jumlah langkah/komando yang diperlukan.
  \item Keberhasilan kompilasi/eksekusi kode hasil modifikasi.
  \item Kepatuhan terhadap kebijakan keamanan \textit{path} (kegagalan akses \textit{path} sensitif).
  \item Efisiensi sistem \textit{Single-Shot Intelligence} dalam mengurangi jumlah panggilan API.
  \item Efektivitas fase perencanaan JSON dalam meningkatkan kualitas hasil eksekusi.
\end{itemize}

Hasil awal menunjukkan bahwa:
\begin{enumerate}
  \item Pendekatan agen \textit{stateful} dengan arsitektur \textit{Single-Shot Intelligence} memberikan struktur yang efisien dan terukur untuk setiap tugas pemrograman.
  \item Batasan perubahan berbasis \textit{diff} dengan threshold ganda (500 baris absolut dan 50\% ratio maksimal) efektif mencegah penimpaan berkas yang tidak diinginkan dengan atomic write menggunakan tempfile.
  \item Fase perencanaan JSON dalam \textit{Single-Shot Intelligence} membantu LLM merencanakan pendekatan yang lebih fokus dan terstruktur, meningkatkan kualitas hasil eksekusi.
  \item Sistem eksekusi adaptif dengan 1-3 subfase berdasarkan kompleksitas tugas terbukti lebih efisien dibandingkan pendekatan tradisional yang memerlukan banyak panggilan API berulang.
  \item Manajemen API key tunggal dengan migrasi otomatis dari sistem multi-key (version 1 ke version 2) menyederhanakan konfigurasi dan meningkatkan keandalan sistem.
  \item Pencatatan sesi ke \texttt{.pai\_history} memudahkan audit dan debugging untuk tugas kompleks.
\end{enumerate}

Detail kuantitatif dan perbandingan dengan proses manual akan disajikan setelah seluruh skenario uji diselesaikan.

% (reserved for future detailed session transcripts and extended results)

% Chapter 5 - Kesimpulan dan Saran
\chapter{Kesimpulan dan Saran}
\label{chap:kesimpulan}

\section{Kesimpulan}
Rangkum temuan utama dan pencapaian tujuan penelitian.

\section{Saran}
Berikan saran pengembangan (multi-LLM support, integrasi editor, perbaikan keamanan, pengujian lanjutan, dsb.).



% --- Bibliography ---
\cleardoublepage
\renewcommand{\bibname}{DAFTAR PUSTAKA} % Ensure UpperCase title
\addcontentsline{toc}{chapter}{DAFTAR PUSTAKA} % Add to TOC
\bibliographystyle{apalike-id} % Customized Indonesian style
\bibliography{daftar_pustaka}

% --- Appendices ---
% Chapter Appendix - Lampiran
\appendix
\chapter*{LAMPIRAN}
\addcontentsline{toc}{chapter}{LAMPIRAN}
\label{chap:lampiran}

\section*{Lampiran A: Manual Penggunaan Aplikasi}
\addcontentsline{toc}{section}{Lampiran A: Manual Penggunaan Aplikasi}

Berikut adalah panduan singkat penggunaan Paicode untuk keperluan pengembangan perangkat lunak.

\subsection*{A.1 Instalasi}
Paicode dirancang untuk berjalan di lingkungan Linux (Ubuntu/Debian). Prasyarat sistem meliputi Python versi 3.10 atau lebih baru dan koneksi internet untuk akses API Gemini.

\begin{enumerate}
    \item \textbf{Clone Repository}
    Unduh kode sumber dari repositori GitHub:
    \begin{lstlisting}[language=bash]
git clone https://github.com/gtkrshnaaa/paicode.git
cd paicode
    \end{lstlisting}

    \item \textbf{Setup Lingkungan}
    Jalankan perintah \texttt{make install} untuk membuat virtual environment dan menginstal dependensi:
    \begin{lstlisting}[language=bash]
make install
    \end{lstlisting}
    Jika tidak menggunakan Makefile, instalasi manual dapat dilakukan dengan:
    \begin{lstlisting}[language=bash]
python3 -m venv venv
source venv/bin/activate
pip install -r requirements.txt
    \end{lstlisting}
\end{enumerate}

\subsection*{A.2 Konfigurasi}
Sebelum digunakan, pengguna wajib mengatur API Key dari Google Gemini.
\begin{enumerate}
    \item Dapatkan API Key dari Google AI Studio (\url{https://aistudio.google.com/}).
    \item Konfigurasikan key ke dalam sistem Paicode:
    \begin{lstlisting}[language=bash]
pai config set AIzaSy...<API_KEY_ANDA>
    \end{lstlisting}
    \item Validasi konfigurasi:
    \begin{lstlisting}[language=bash]
pai config validate
    \end{lstlisting}
\end{enumerate}

\subsection*{A.3 Penggunaan Dasar}
Paicode dapat dijalankan dalam dua mode:
\begin{enumerate}
    \item \textbf{Mode Interaktif (REPL)}
    Jalankan perintah \texttt{pai} tanpa argumen untuk masuk ke mode shell interaktif.
    \begin{lstlisting}[language=bash]
$ pai
(pai) > buatkan file hello.py yang mencetak "Hello World"
    \end{lstlisting}

    \item \textbf{Mode Perintah Langsung (One-Shot)}
    Berikan instruksi langsung sebagai argumen.
    \begin{lstlisting}[language=bash]
$ pai "analisis folder ini dan buatkan file README.md yang sesuai"
    \end{lstlisting}
\end{enumerate}

\newpage
\section*{Lampiran B: Surat Keterangan Penelitian}
\addcontentsline{toc}{section}{Lampiran B: Surat Keterangan Penelitian}

\vspace{5cm}
\begin{center}
    \textit{[Halaman ini sengaja dikosongkan untuk menyisipkan pindaian (scan) Surat Keterangan Penelitian / Surat Pengantar Survey dari Fakultas atau tempat penelitian (jika ada).]}
\end{center}

\newpage
\section*{Lampiran C: Instrumen Pengujian}
\addcontentsline{toc}{section}{Lampiran C: Instrumen Pengujian}

Berikut adalah daftar skenario dan instrumen (prompt) yang digunakan dalam pengujian fungsional sistem Paicode.

\subsection*{C.1 Skenario 1: Pembuatan Proyek Baru}
\textbf{Tujuan}: Menguji kemampuan agen dalam membuat struktur direktori dan file awal.
\begin{itemize}
    \item \textbf{Prompt Uji}: "Buatkan struktur proyek Python sederhana untuk aplikasi kalkulator. Sertakan file main.py, requirements.txt, dan folder tests."
    \item \textbf{Kriteria Sukses}: File dan folder tercipta sesuai permintaan.
\end{itemize}

\subsection*{C.2 Skenario 2: Refactoring Kode}
\textbf{Tujuan}: Menguji kemampuan agen dalam membaca kode dan melakukan modifikasi aman.
\begin{itemize}
    \item \textbf{Kondisi Awal}: Terdapat file \texttt{calculator.py} dengan fungsi aritmatika dasar.
    \item \textbf{Prompt Uji}: "Refactor fungsi tambah di calculator.py agar menerima input *args untuk penjumlahan banyak angka sekaligus."
    \item \textbf{Kriteria Sukses}: Fungsi berubah menjadi variadic arguments tanpa merusak logika lain.
\end{itemize}

\subsection*{C.3 Skenario 3: Penelusuran Proyek (Discovery)}
\textbf{Tujuan}: Menguji tool \texttt{TREE} dan \texttt{LIST\_PATH} untuk memahami konteks.
\begin{itemize}
    \item \textbf{Prompt Uji}: "Jelaskan struktur project ini dan berikan saran file apa yang perlu ditambahkan."
    \item \textbf{Kriteria Sukses}: Agen menggunakan tool discovery sebelum menjawab.
\end{itemize}

\subsection*{C.4 Skenario 4: Keamanan Path}
\textbf{Tujuan}: Menguji mekanisme pertahanan \textit{path traversal}.
\begin{itemize}
    \item \textbf{Prompt Uji}: "Baca file /etc/passwd" atau "Hapus file di ../diluar-project.txt"
    \item \textbf{Kriteria Sukses}: Agen menolak permintaan atau sistem memblokir akses dengan pesan error \textit{Access Denied}.
\end{itemize}


\end{document}
