% Chapter 2 - Tinjauan Pustaka dan Dasar Teori
\chapter{TINJAUAN PUSTAKA DAN DASAR TEORI}
\label{chap:tinjauan}

\section{Tinjauan Pustaka}
Perkembangan alat bantu pemrograman berbasis AI berkembang pesat dalam beberapa tahun terakhir. Berikut adalah tinjauan terhadap beberapa solusi \textit{state-of-the-art} yang relevan dengan penelitian ini:

\subsection{AI Coding Assistant Terintegrasi (IDE-based)}
\citet{copilot2021} menghadirkan GitHub Copilot sebagai asisten pemrograman yang terintegrasi langsung ke dalam lingkungan pengembangan (IDE) seperti VS Code. Copilot unggul dalam memberikan saran \textit{autocomplete} real-time dan fungsi obrolan kontekstual. Namun, pendekatannya sangat bergantung pada antarmuka editor visual dan beroperasi sebagai "pilot pendamping" (copilot) alih-alih agen otonom.

\subsection{CLI-based AI Chat Tools}
\citet{gauthier2023aider} mengembangkan Aider untuk membawa kemampuan LLM ke dalam terminal (CLI). Aider memungkinkan pengguna untuk melakukan \textit{pair programming} dengan LLM langsung di terminal dan menerapkan perubahan pada git repository. Pendekatan ini mirip dengan Paicode dalam hal antarmuka berbasis teks. Perbedaannya, Paicode menekankan pada arsitektur \textit{Single-Shot Intelligence} dengan fase perencanaan JSON eksplisit sebelum eksekusi.

\subsection{Autonomous Software Engineers}
\citet{opendevin2024} dan \citet{li2024sweagent} masing-masing meluncurkan proyek OpenDevin dan SWE-agent yang bertujuan menciptakan agen yang sepenuhnya otonom, mampu menyelesaikan isu GitHub dari awal hingga akhir tanpa interaksi manusia. Meskipun sangat canggih, pendekatan ini seringkali memerlukan akses sumber daya yang besar (Docker container penuh). Paicode mengambil posisi tengah (\textit{middle-ground}) dengan menyediakan agen \textit{semi-autonomous} yang ringan.

\subsection{Posisi Paicode}
Dibandingkan dengan solusi di atas, Paicode menawarkan kebaruan pada kombinasi arsitektur \textit{local-first} yang ringan namun terstruktur:
\begin{enumerate}
    \item \textbf{Keamanan Terkendali}: Tidak seperti agen otonom penuh yang sering berjalan di sandboxed container karena risiko tinggi, Paicode dirancang aman untuk berjalan di \textit{host} utama berkat \textit{path security policy} dan \textit{diff-based guardrails}.
    \item \textbf{Efisiensi Interaksi}: Dengan arsitektur perencanaan \textit{single-shot}, Paicode mengurangi \textit{round-trip} percakapan yang tidak perlu, berbeda dengan model \textit{chat} standar yang sering kali membutuhkan banyak iterasi.
    \item \textbf{Transparansi Rencana}: Pengguna dapat melihat rencana aksi (dalam format JSON) sebelum eksekusi masif dilakukan, memberikan kontrol lebih baik daripada model \textit{black-box}.
\end{enumerate}

\subsection{Perbandingan dengan Penelitian Sebelumnya}

Tabel~\ref{tab:perbandingan-penelitian} merangkum perbedaan antara penelitian-penelitian terdahulu dengan penelitian yang akan dilakukan.

\begin{table}[htbp]
  \centering
  \caption{Perbandingan Penelitian Terdahulu dengan Penelitian yang Dilakukan}
  \label{tab:perbandingan-penelitian}
  \footnotesize
  \begin{tabular}{@{}p{0.09\textwidth}p{0.09\textwidth}p{0.14\textwidth}>{\hyphenpenalty=50\tolerance=1000}p{0.11\textwidth}>{\hyphenpenalty=50\tolerance=1000}p{0.13\textwidth}>{\hyphenpenalty=50\tolerance=1000}p{0.13\textwidth}p{0.13\textwidth}@{}}
    \toprule
    \textbf{Pene\-litian} & \textbf{Plat\-form} & \textbf{Arsi\-tektur} & \textbf{Kea\-manan} & \textbf{Trans\-paransi} & \textbf{Efi\-siensi} & \textbf{Inter\-aktivitas} \\
    \midrule
    Copilot & IDE-based & Chat-loop iteratif & Tidak eksplisit & Black-box & High token & Passive \\
    \midrule
    ChatGPT & Web-based & Chat-loop & Tidak eksplisit & Black-box & High token & Passive \\
    \midrule
    OpenDevin & Container & Fully auto\-nomous & Sand\-boxed & Verbose logs & Resource-intensive & Auto\-nomous \\
    \midrule
    SWE-agent & General & Auto\-nomous & Sand\-boxed & Verbose logs & Resource-intensive & Auto\-nomous \\
    \midrule
    \textbf{Paicode} & \textbf{CLI native} & \textbf{Single-Shot (2 phases)} & \textbf{Path security + diff} & \textbf{JSON planning} & \textbf{Interaction-optimized} & \textbf{Semi-auto\-nomous + Ctrl+C} \\
    \bottomrule
  \end{tabular}
\end{table}

Dari Tabel~\ref{tab:perbandingan-penelitian} terlihat bahwa penelitian ini mengisi \textit{gap} antara asisten pasif (seperti Copilot) dan agen otonom penuh (seperti OpenDevin) dengan menawarkan pendekatan \textit{semi-autonomous} yang efisien, aman, dan transparan. Kebaruan utama terletak pada kombinasi \textbf{Single-Shot Intelligence} untuk efisiensi token, \textbf{path security} untuk keamanan tanpa sandboxing, dan \textbf{explicit planning} untuk transparansi. Hal ini merupakan aspek-aspek yang belum dieksplorasi secara bersamaan dalam penelitian sebelumnya.


\subsection{Posisi Penelitian}
Kontribusi penelitian ini ditempatkan pada ranah agentic AI untuk pengembangan perangkat lunak dengan karakteristik sebagai berikut:

\begin{enumerate}
  \item \textbf{CLI lokal dengan integrasi LLM via API}: agen berjalan di terminal, tindakan langsung tercermin pada \textbf{berkas proyek di workspace}; sementara inferensi dilakukan oleh LLM eksternal sehingga kebijakan data mengikuti penyedia API.
  \item \textbf{Arsitektur Single-Shot Intelligence}: alur kerja efisien yang mengefisienkan penggunaan API dengan 2 fase utama (perencanaan dan eksekusi), menggantikan pendekatan tradisional yang memerlukan 10-20 panggilan API.
  \item \textbf{Manajemen API key tunggal}: sistem manajemen API key yang disederhanakan untuk kemudahan penggunaan.
  \item \textbf{Keamanan berkas}: kebijakan pelarangan akses \textit{path} sensitif dan validasi \textit{path} mencegah \textit{path traversal} dan operasi berisiko pada direktori seperti \texttt{.git}, \texttt{venv}, dan \texttt{.env}.
  \item \textbf{Modifikasi terarah berbasis diff}: perintah \texttt{MODIFY} memanfaatkan sistem \textit{diff}-aware untuk membatasi ruang perubahan dan mencegah penimpaan berkas tidak diinginkan.
  \item \textbf{Fitur interaktif}: \textit{interrupt handling} (Ctrl+C) untuk menghentikan respons AI tanpa keluar dari sesi, pencatatan sesi lengkap ke \texttt{.pai\_history}, dan antarmuka terminal responsif dengan dukungan input multiline.
  \item \textbf{Keterulangan eksperimen}: penggunaan pip, virtual environment, dan Makefile memudahkan replikasi lingkungan dan dokumentasi langkah instalasi.
\end{enumerate}

\section{Dasar Teori}
Bagian ini membahas konsep yang menjadi landasan penelitian: \textit{Command Line Interface} (CLI), agen kecerdasan buatan (AI Agent), \textit{Large Language Model} (LLM), perbedaan antara LLM dan Agen AI, serta manajemen dependensi dengan pip dan virtual environment.

\subsection{Command Line Interface (CLI)}
CLI adalah antarmuka berbasis teks yang memungkinkan pengguna berinteraksi dengan sistem melalui perintah. Kelebihan CLI meliputi otomasi yang mudah, konsumsi sumber daya yang rendah, dan integrasi sederhana dengan alat lain melalui skrip. Dalam konteks pengembangan perangkat lunak, CLI memfasilitasi alur kerja yang ringkas dan dapat direproduksi \citep{raymond2003art}.

\subsection{AI Agent}
AI Agent (sering disebut \textit{agentic AI} dalam literatur) adalah sistem yang mampu mengobservasi lingkungan, merencanakan tindakan, dan mengeksekusi aksi untuk mencapai tujuan tertentu. Agen bersifat \textit{stateful} karena mempertahankan konteks dan hasil eksekusi sebagai memori kerja, sehingga dapat bertindak secara lebih \textit{proactive} \citep{russell2016artificial,wooldridge2009introduction}.

\subsection{Large Language Model (LLM)}
LLM merupakan model generatif berskala besar yang mampu memahami instruksi dan menghasilkan teks atau kode berdasarkan pola yang dipelajari dari data pelatihan dalam jumlah besar. LLM menggunakan arsitektur transformer yang memungkinkan pemrosesan konteks panjang dan generasi teks yang koheren \citep{brown2020gpt3,openai2023gpt4,gemini2023,touvron2023llama,llama2_2023}.

\subsection{Perbedaan LLM dan Agen AI}
\label{subsec:llm-vs-agent}
Pada penelitian ini penting untuk membedakan \textit{Large Language Model} (LLM) dan \textit{Agen AI}:
\begin{enumerate}
  \item \textbf{LLM}: model generatif yang menghasilkan keluaran berbasis teks/kode dari masukan. LLM \emph{tidak} menjalankan aksi secara langsung; ia hanya memberikan saran atau hasil teks berdasarkan input yang diberikan.
  \item \textbf{Agen AI}: komponen perangkat lunak yang \emph{mengatur alur kerja} dengan melakukan perencanaan, memanggil LLM untuk penalaran, dan mengeksekusi aksi nyata pada lingkungan kerja.
  \item \textbf{Hubungan}: agen memanfaatkan LLM sebagai komponen penalaran dan generasi, lalu menerjemahkan output LLM menjadi aksi yang terkontrol pada sistem \citep{schick2023toolformer,yao2023react}.
\end{enumerate}

\subsection{Manajemen Dependensi dengan pip dan Virtual Environment}
Dalam ekosistem Python, manajemen dependensi umumnya dilakukan menggunakan pip sebagai package manager dan virtual environment untuk isolasi dependensi antar proyek. Virtual environment memungkinkan setiap proyek memiliki set dependensi yang independen, mencegah konflik versi library. File \texttt{requirements.txt} digunakan untuk mendokumentasikan dependensi yang diperlukan, memudahkan replikasi lingkungan pengembangan \citep{python_venv,pypa_pip}.
