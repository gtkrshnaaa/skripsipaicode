% Chapter 5 - Kesimpulan dan Saran
\chapter{PENUTUP}
\label{chap:kesimpulan}

\section{SIMPULAN}
Penelitian ini menghasilkan prototipe \textbf{Paicode}, sebuah agen AI berbasis CLI yang mendukung proses pengembangan perangkat lunak secara interaktif dengan memanfaatkan LLM eksternal melalui API. Sistem beroperasi pada terminal lokal dan melakukan \textbf{operasi berkas tingkat-aplikasi di ruang kerja proyek}, dilengkapi kebijakan \textit{path security} untuk mencegah akses ke direktori sensitif. Himpunan perintah yang disediakan (\texttt{MKDIR}, \texttt{TOUCH}, \texttt{READ}, \texttt{WRITE}, \texttt{MODIFY}, \texttt{RM}, \texttt{MV}, \texttt{TREE}, \texttt{LIST\_PATH}, \texttt{FINISH}) memungkinkan agen untuk mengobservasi, memanipulasi, dan memodifikasi berkas secara terarah.

Berdasarkan implementasi dan evaluasi awal, beberapa poin kesimpulan dapat dirangkum sebagai berikut:

\begin{enumerate}
  \item Arsitektur \textit{Single-Shot Intelligence} dengan 5 komponen (klasifikasi intensi, acknowledgment dinamis, fase perencanaan JSON, fase eksekusi adaptif 1-3 subfase, dan saran langkah berikutnya) memberikan struktur yang efisien dan terukur untuk setiap tugas pemrograman.
  \item Integrasi agen \textit{stateful} di lingkungan CLI efektif dalam mempercepat beberapa tugas rekayasa perangkat lunak berulang (pembuatan struktur proyek, pembuatan dan pembacaan berkas, serta modifikasi terarah) dengan tetap menjaga keterlacakan langkah.
  \item Mekanisme pembatasan perubahan berbasis \textit{diff} pada perintah \texttt{MODIFY} dengan threshold ganda (500 baris absolut dan 50\% ratio maksimal, dapat dikonfigurasi via \texttt{PAI\_MODIFY\_THRESHOLD} dan \texttt{PAI\_MODIFY\_MAX\_RATIO}) membantu mengurangi risiko penimpaan besar yang tidak diinginkan dengan atomic write menggunakan tempfile.
  \item Fase perencanaan JSON dalam \textit{Single-Shot Intelligence} membantu LLM merencanakan pendekatan yang lebih fokus dan terstruktur, meningkatkan kualitas hasil eksekusi.
  \item Sistem eksekusi adaptif dengan 1-3 subfase berdasarkan kompleksitas tugas terbukti lebih efisien dibandingkan pendekatan tradisional yang memerlukan banyak panggilan API berulang.
  \item Manajemen API key tunggal dengan migrasi otomatis dari sistem multi-key (version 1 ke version 2) menyederhanakan konfigurasi dan meningkatkan keandalan sistem.
  \item Fitur interaktif seperti \textit{interrupt handling} (Ctrl+C) dan pencatatan sesi ke \texttt{.pai\_history} meningkatkan pengalaman pengguna dan memudahkan debugging.
  \item Kebijakan keamanan path berhasil memblokir akses ke direktori sensitif (mis. \texttt{.git}, \texttt{venv}, \texttt{.env}) dan mencegah \textit{path traversal}, mendukung aspek privasi dan kendali lokal.
  \item Pemakaian pip/venv, Makefile, dan LaTeX mendukung keterulangan eksperimen serta dokumentasi terstruktur untuk keperluan akademik.
\end{enumerate}

Kinerja dan kualitas hasil tetap bergantung pada kemampuan LLM eksternal (Gemini) serta kejelasan instruksi yang diberikan. Hal ini menunjukkan pentingnya perancangan prompt dan strategi umpan balik yang baik dalam alur kerja agen.

\section{SARAN}
Beberapa saran pengembangan lanjutan yang dapat dilakukan antara lain:

\begin{itemize}
  \item \textbf{Dukungan multi-LLM}: menambahkan opsi pemilihan model dan penyedia LLM alternatif (OpenAI GPT, Anthropic Claude, Llama, dll.) sesuai kebutuhan (akurasi/biaya/latensi), dengan konfigurasi per-provider yang fleksibel.
  \item \textbf{Optimasi fase perencanaan}: mengembangkan mekanisme caching untuk hasil perencanaan JSON yang serupa, mengurangi waktu respons untuk tugas berulang.
  \item \textbf{Peningkatan validasi hasil}: menambahkan automated testing (unit test, integration test) sebagai bagian dari validasi hasil eksekusi untuk verifikasi kualitas yang lebih objektif.
  \item \textbf{Integrasi editor}: menyediakan jembatan ringan ke IDE (mis. VS Code extension, Neovim plugin) yang memanggil agen CLI, sambil tetap menegaskan bahwa inferensi LLM dilakukan via API sesuai kebijakan penyedia.
  \item \textbf{Peningkatan keamanan}: memperluas kebijakan \textit{allow/deny list} \textit{path}, menambah konfirmasi eksplisit untuk operasi berisiko (mis. \texttt{RM}), dan memperketat validasi konten sebelum penulisan berkas.
  \item \textbf{Memori jangka panjang}: menambahkan ringkasan sesi dan penyimpanan konteks terkurasi (vector database) agar agen dapat mempelajari preferensi proyek pengguna secara berkelanjutan.
  \item \textbf{Fitur kolaborasi}: menambahkan dukungan untuk sesi multi-user dengan shared context, memungkinkan tim untuk bekerja bersama dengan agen.
  \item \textbf{Adaptive threshold}: mengembangkan sistem yang secara otomatis menyesuaikan threshold modifikasi (\texttt{PAI\_MODIFY\_THRESHOLD}) berdasarkan ukuran file dan kompleksitas perubahan.
  \item \textbf{Evaluasi kuantitatif}: melakukan pengujian terstandardisasi dengan skenario lebih beragam, termasuk proyek nyata berskala kecil-menengah, untuk memperoleh gambaran dampak produktivitas yang lebih komprehensif.
  \item \textbf{Dashboard monitoring}: menambahkan dashboard web untuk memantau penggunaan API key, statistik sesi, skor kualitas rata-rata, dan metrik performa lainnya.
\end{itemize}
