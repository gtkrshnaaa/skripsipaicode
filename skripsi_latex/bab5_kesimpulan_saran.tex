% Chapter 5 - Kesimpulan dan Saran
\chapter{Kesimpulan dan Saran}
\label{chap:kesimpulan}

\section{Kesimpulan}
Penelitian ini menghasilkan prototipe \textbf{Paicode}, sebuah agen AI berbasis CLI yang mendukung proses pengembangan perangkat lunak secara interaktif dengan memanfaatkan LLM eksternal. Sistem dirancang dengan prinsip \textit{local-first} dan dilengkapi kebijakan keamanan jalur untuk mencegah akses ke direktori sensitif. Himpunan perintah yang disediakan (\texttt{MKDIR}, \texttt{TOUCH}, \texttt{READ}, \texttt{WRITE}, \texttt{MODIFY}, \texttt{RM}, \texttt{MV}, \texttt{TREE}, \texttt{LIST\_PATH}, \texttt{FINISH}) memungkinkan agen untuk mengobservasi, memanipulasi, dan memodifikasi berkas secara terarah.

Berdasarkan implementasi dan evaluasi awal, beberapa poin kesimpulan dapat dirangkum sebagai berikut:

\begin{enumerate}
  \item Integrasi agen \textit{stateful} di lingkungan CLI efektif dalam mempercepat beberapa tugas rekayasa perangkat lunak berulang (pembuatan struktur proyek, pembuatan dan pembacaan berkas, serta modifikasi terarah) dengan tetap menjaga keterlacakan langkah.
  \item Mekanisme pembatasan perubahan berbasis \textit{diff} pada perintah \texttt{MODIFY} membantu mengurangi risiko penimpaan besar yang tidak diinginkan, sehingga sejalan dengan prinsip perubahan minimal.
  \item Kebijakan keamanan jalur berhasil memblokir akses ke direktori sensitif (mis. \texttt{.git}, \texttt{venv}, \texttt{.env}) dan mencegah \textit{path traversal}, mendukung aspek privasi dan kendali lokal.
  \item Pemakaian Poetry, Makefile, dan LaTeX mendukung keterulangan eksperimen serta dokumentasi terstruktur untuk keperluan akademik.
\end{enumerate}

Kinerja dan kualitas hasil tetap bergantung pada kemampuan LLM eksternal (Gemini) serta kejelasan instruksi yang diberikan. Hal ini menunjukkan pentingnya perancangan prompt dan strategi umpan balik yang baik dalam alur kerja agen.

\section{Saran}
Beberapa saran pengembangan lanjutan yang dapat dilakukan antara lain:

\begin{itemize}
  \item \textbf{Dukungan multi-LLM}: menambahkan opsi pemilihan model dan penyedia LLM alternatif sesuai kebutuhan (akurasi/biaya/latensi).
  \item \textbf{Integrasi editor}: menyediakan jembatan ringan ke IDE (mis. VS Code) tanpa mengorbankan sifat \textit{local-first}, misalnya melalui ekstensi yang memanggil agen CLI.
  \item \textbf{Peningkatan keamanan}: memperluas kebijakan \textit{allow/deny list} jalur, menambah konfirmasi eksplisit untuk operasi berisiko, dan memperketat validasi konten sebelum penulisan berkas.
  \item \textbf{Memori jangka panjang}: menambahkan ringkasan sesi dan penyimpanan konteks terkurasi agar agen dapat mempelajari preferensi proyek pengguna secara berkelanjutan.
  \item \textbf{Evaluasi kuantitatif}: melakukan pengujian terstandardisasi dengan skenario lebih beragam, termasuk proyek nyata berskala kecil-menengah, untuk memperoleh gambaran dampak produktivitas yang lebih komprehensif.
\end{itemize}
