% Chapter 5 - Kesimpulan dan Saran
\chapter{PENUTUP}
\label{chap:kesimpulan}

\section{SIMPULAN}
Penelitian ini menghasilkan prototipe \textbf{Paicode}, sebuah agen AI berbasis CLI yang mendukung proses pengembangan perangkat lunak secara interaktif dengan memanfaatkan LLM eksternal melalui API. Sistem beroperasi pada terminal lokal dan melakukan \textbf{operasi berkas tingkat-aplikasi di ruang kerja proyek}, dilengkapi kebijakan \textit{path security} untuk mencegah akses ke direktori sensitif. Himpunan perintah yang disediakan (\texttt{MKDIR}, \texttt{TOUCH}, \texttt{READ}, \texttt{WRITE}, \texttt{MODIFY}, \texttt{RM}, \texttt{MV}, \texttt{TREE}, \texttt{LIST\_PATH}, \texttt{FINISH}) memungkinkan agen untuk mengobservasi, memanipulasi, dan memodifikasi berkas secara terarah.

Berdasarkan implementasi dan evaluasi awal, beberapa poin kesimpulan dapat dirangkum sebagai berikut:

\begin{enumerate}
  \item \textbf{Desain Arsitektur Sistem}: Penelitian ini berhasil merancang arsitektur \textit{Single-Shot Intelligence} (SSI) yang memadukan klasifikasi intensi, perencanaan terstruktur dalam format JSON, dan eksekusi adaptif. Arsitektur ini terbukti mampu memberikan kerangka kerja yang jelas bagi agen AI untuk menyelesaikan tugas pemrograman multi-langkah di lingkungan CLI secara otonom namun tetap terkendali.
  \item \textbf{Implementasi Kapabilitas dan Keamanan}: Sistem Paicode telah berhasil diimplementasikan dengan kemampuan observasi proyek (\texttt{TREE}, \texttt{LIST\_PATH}), manipulasi berkas, dan modifikasi kode terarah. Penerapan kebijakan \textit{Path Security} dan pembatasan modifikasi berbasis \textit{diff} (threshold ganda) terbukti efektif dalam memblokir akses ke direktori sensitif dan mencegah perubahan destruktif masif, menjawab kebutuhan akan keamanan agen yang berjalan di mesin lokal.
  \item \textbf{Fitur Interaktif}: Integrasi fitur-fitur interaktif seperti \textit{interrupt handling} (Ctrl+C) untuk penghentian darurat dan pencatatan sesi otomatis ke \texttt{.pai\_history} berhasil meningkatkan pengalaman pengguna dan transparansi proses, memungkinkan pengguna untuk mempertahankan kendali penuh atas tindakan agen.
  \item \textbf{Evaluasi Kinerja}: Berdasarkan hasil pengujian fungsional dengan skenario pemrograman representatif, Paicode menunjukkan peningkatan efisiensi yang signifikan (hingga 25x lebih cepat untuk tugas \textit{scaffolding}) dan tingkat keberhasilan eksekusi 100\% pada skenario yang diuji dengan kondisi jaringan stabil. Sistem terbukti andal dalam menerjemahkan intensi pengguna menjadi aksi nyata dengan overhead minimal dibandingkan metode manual.
\end{enumerate}

Secara arsitektural, Paicode memiliki karakteristik keunggulan dan batasan sebagai berikut:
\begin{enumerate}
    \item \textbf{Keunggulan}: Paicode menawarkan otonomi eksekusi multi-langkah di terminal lokal yang memberikan transparansi penuh melalui mekanisme rencana eksekusi (\textit{JSON planning}) dan pencatatan log aktivitas yang terstruktur. Desainnya yang minimalis dan berbasis CLI memungkinkan aplikasi ini berjalan di lingkungan Linux tanpa antarmuka grafis (\textit{headless}).
    \item \textbf{Batasan}: Sebagai aplikasi berbasis CLI yang berfokus pada otonomi, Paicode belum menyediakan fitur debugging visual interaktif seperti pada Integrated Development Environment (IDE). Selain itu, ketergantungan penuh pada konektivitas API LLM eksternal mengharuskan adanya koneksi internet aktif selama penggunaan.
\end{enumerate}

Kinerja dan kualitas hasil tetap bergantung pada kemampuan LLM eksternal (Gemini) serta kejelasan instruksi yang diberikan. Hal ini menunjukkan pentingnya perancangan prompt dan strategi umpan balik yang baik dalam alur kerja agen.

\section{SARAN}
Beberapa saran pengembangan lanjutan yang dapat dilakukan antara lain:

\begin{enumerate}
  \item \textbf{Dukungan multi-LLM}: menambahkan opsi pemilihan model dan penyedia LLM alternatif (OpenAI GPT, Anthropic Claude, Llama, dll.) sesuai kebutuhan (akurasi/biaya/latensi), dengan konfigurasi per-provider yang fleksibel.
  \item \textbf{Optimasi fase perencanaan}: mengembangkan mekanisme caching untuk hasil perencanaan JSON yang serupa, mengurangi waktu respons untuk tugas berulang.
  \item \textbf{Peningkatan validasi hasil}: menambahkan automated testing (unit test, integration test) sebagai bagian dari validasi hasil eksekusi untuk verifikasi kualitas yang lebih objektif.
  \item \textbf{Integrasi editor}: menyediakan jembatan ringan ke IDE (mis. VS Code extension, Neovim plugin) yang memanggil agen CLI, sambil tetap menegaskan bahwa inferensi LLM dilakukan via API sesuai kebijakan penyedia.
  \item \textbf{Peningkatan keamanan}: memperluas kebijakan \textit{allow/deny list} \textit{path}, menambah konfirmasi eksplisit untuk operasi berisiko (mis. \texttt{RM}), dan memperketat validasi konten sebelum penulisan berkas.
  \item \textbf{Memori jangka panjang}: menambahkan ringkasan sesi dan penyimpanan konteks terkurasi (vector database) agar agen dapat mempelajari preferensi proyek pengguna secara berkelanjutan.
  \item \textbf{Fitur kolaborasi}: menambahkan dukungan untuk sesi multi-user dengan shared context, memungkinkan tim untuk bekerja bersama dengan agen.
  \item \textbf{Adaptive threshold}: mengembangkan sistem yang secara otomatis menyesuaikan threshold modifikasi (\texttt{PAI\_MODIFY\_THRESHOLD}) berdasarkan ukuran file dan kompleksitas perubahan.
  \item \textbf{Evaluasi lanjutan}: melakukan pengujian terstandardisasi dengan skenario lebih beragam, termasuk proyek nyata berskala kecil-menengah, untuk memperoleh gambaran dampak produktivitas yang lebih komprehensif.
  \item \textbf{Dashboard monitoring}: menambahkan dashboard web untuk memantau penggunaan API key, statistik sesi, skor kualitas rata-rata, dan metrik performa lainnya.
\end{enumerate}
